%% ================================================================================
%% This LaTeX file was created by AbiWord.                                         
%% AbiWord is a free, Open Source word processor.                                  
%% More information about AbiWord is available at http://www.abisource.com/        
%% ================================================================================

\documentclass[letterpaper,portrait,12pt]{article}
\usepackage[latin1]{inputenc}
\usepackage{calc}
\usepackage{setspace}
\usepackage{fixltx2e}
\usepackage{graphicx}
\usepackage{multicol}
\usepackage[normalem]{ulem}
%% Please revise the following command, if your babel
%% package does not support en-US
\usepackage[en]{babel}
\usepackage{color}
\usepackage{hyperref}
 
\begin{document}

\setlength{\oddsidemargin}{1.2654in-1in}
\setlength{\textwidth}{\paperwidth - 1.1228in-1.2654in}

M\'{e}moire M2














Aspects formels





--- La bibliographie n'est pas une annexe. Il aurait \'{e}t\'{e} en outre envisageable de la classer pour distinguer


les travaux sur la SF (avec ou sans num\'{e}rique) des \'{e}tudes essentiellement


m\'{e}thodologiques.





--- Sur un point assez secondaire: le m\'{e}moire pr\'{e}sente quelques probl\`{e}mes


formels et typographiques (espacement dans les notes, points \`{a} la fin


des notes manquants, placement des appels de note, pr\'{e}sence de guillemets


anglaises, citations en bloc mais sans retrait)







































































Table des mati\`{e}res





R\'{e}sum\'{e}





Introduction











I) \uline{M\'{e}thodologie}





\textbf{1. }Du questionnement d'historien\ldots{}





2. ...\`{a} la constitution d'un support d'investigation








\textbf{II) }\textbf{\uline{La construction d'une m\'{e}thode computationnelle pour explorer notre corpus}}





1. Le \emph{topic modeling}





\textbf{2.} Construction d'un module





3. Cr\'{e}ation d'un mod\`{e}le adapt\'{e} \`{a} nos donn\'{e}es et entra\^{i}nement sur nos donn\'{e}es











III) \uline{R\'{e}sultats et retours m\'{e}thodologiques}





1. Pr\'{e}sentation des r\'{e}sultats, graphiques





2. Conclusions li\'{e}es aux r\'{e}sultats





3. Explorer d'autres pistes








Conclusion








Bibliographie








Annexes


{\huge R\'{e}sum\'{e}}





Dans ce m\'{e}moire, nous cherchons \`{a} d\'{e}montrer l'apport herm\'{e}neutique de la science-fiction en tant qu'objet de recherche. Nous montrons notamment son int\'{e}r\^{e}t pour \'{e}tudier l'histoire des repr\'{e}sentations et projections associ\'{e}es au futur. Nous commen\c{c}ons par d\'{e}finir clairement la litt\'{e}rature de science fiction, afin de construire un objet de recherche adapt\'{e} \`{a} notre \'{e}tude. Puis nous pr\'{e}sentons la notion de futurs envisag\'{e}s : nous montrons ici la pertinence d'utiliser la litt\'{e}rature de science-fiction « anticipatrice » comme support d'investigation pour appr\'{e}hender l'histoire des futurs envisag\'{e}s. Nous proposons ensuite d'avoir recours \`{a} des algorithmes informatiques de \emph{topic modeling} pour construire une m\'{e}thode d'analyse. Nous poursuivons en appliquant cette m\'{e}thode sur  un prototype exp\'{e}rimental r\'{e}duit, afin d'\'{e}valuer nos m\'{e}thodes et de les perfectionner. Enfin, nous analysons les r\'{e}sultats, et esquissons une r\'{e}flexion \'{e}pist\'{e}mologique sur notre d\'{e}marche. Cette r\'{e}flexion doit permettre de parfaire notre m\'{e}thode d'analyse, ainsi que d'explorer les perspectives de recherches qui s'offrent \`{a} nous.





Abstract





In this thesis, I am trying to demonstrate the hermeneutic contribution of science fonction as a research \emph{topic}. Notably, I show its interest to study the history of futuristic representations and projections. Firstly, I define clearly the choseb science fiction literature, in order to build up an appropriate research \emph{topic}. Then, I present the notion of expected futures as an appropriate one to study representations and projections among SF literature. I also suggest using digital humanities methods, particularly \emph{topic modeling}, to build up an analysis conduct. Finally, I apply \emph{topic modeling} to an experimental prototype, in order to evaluate and optimize my method. I analyse the results thus obtained, and sketch out an epistemological thought on my conduct. This thought offers a solid basis to improve our analysis method, and explore new research perspectives.
















































































INTRODUCTION
































\textbf{	}\emph{{``}L'Histoire est un roman qui a \'{e}t\'{e} ; le roman est de l'histoire qui aurait pu \^{e}tre''}, \'{e}crivaient d'une m\^{e}me plume Jules et Edmond de Goncourt, relevant toute la socialit\'{e} du texte litt\'{e}raire et de ses ressors narratifs. La litt\'{e}rature r\'{e}sulte, en effet, d'une formalisation probl\'{e}matique de l'imaginaire social, et entretient de fait une relation privil\'{e}gi\'{e}e autant qu'ambig\"{u}e avec les imaginaires collectifs : {``}\emph{Un roman est en interaction dynamique majeure avec la narrativit\'{e} ambiante, c'est-\`{a}-dire avec les fa\c{c}ons de raconter l'histoire de la soci\'{e}t\'{e} et les \'{e}v\'{e}nements qu'elle vit}''\footnote{	https://journals.openedition.org/pratiques/1762\#tocto2n6$\rightarrow$ Voir ref 93}  . Ainsi, le roman, en ce qu'il est {`}une histoire qui aurait pu \^{e}tre', est profod\'{e}ment ancr\'{e} dans  la r\'{e}alit\'{e} qui constitue son contexte de production, et int\'{e}resse particuli\`{e}rement l'histoire des imaginaires sociaux. Mais qu'en est-il, plus particuli\`{e}rement, de la litt\'{e}rature qui raconte des histoires qui pourraient advenir ? De mani\`{e}re plus pr\'{e}cise, j'ai souhait\'{e} m'int\'{e}resser \`{a} la litt\'{e}rature d'anticipation, qui s'attache \`{a} d\'{e}crire des futurs possibles, des mondes envisageables, en ce qu'elle constitue un mat\'{e}riau int\'{e}ressant pour acc\'{e}der aux imaginaires projectionnistes. En d\'{e}calant ainsi notre perspective, on s'int\'{e}resse non plus \`{a} l'histoire des imaginaires sociaux, mais \`{a} celle des projections futuristes, des diff\'{e}rents sc\'{e}narios envisag\'{e}s pour l'avenir.  Ce sont ces imaginaires projectionnistes que je proposerai d'appeler {`}futurs envisag\'{e}s' tout au long de ce m\'{e}moire. 





	Commen\c{c}ons donc par d\'{e}finir clairement cette notion, puisqu'elle est au c\oe{}ur de notre d\'{e}marche. Les futurs envisag\'{e}s d\'{e}signent excplicitement les sc\'{e}narios consid\'{e}r\'{e}s lorsqu'on essaye de se projeter dans l'avenir, et les multiples possibles et imaginaires qu'ils \'{e}voquent. Deux notions relativement proches mais bien distinctes de ce que nous entendons par l'expression {``}futurs envisag\'{e}s'' doivent ici nous permettre de la clarifier.





	D'abord, les futurs envisag\'{e}s se distinguent de la prospective, cette derni\`{e}re consistant en la recherche du, ou des sc\'{e}nario(s) le(s) plus plausible(s), d'un point de vue pragmatique. A cet \'{e}gard, de nombreux travaux int\'{e}ressants ont \'{e}t\'{e} publi\'{e}s sur l'histoire de la prospective\footnote{\textsuperscript{\newpage
}\textsuperscript{	GAUDIN,\ Thierry.\ La\ prospective.\ 2005,\ Presses\ Universitaires\ de\ France,\ «\ Que\ sais-je\ ?\ »\ }} . Or, ces travaux s'attachent davantage \`{a} d\'{e}crire la mani\`{e}re dont les hommes ont interrog\'{e} ces sc\'{e}narios pour \'{e}clairer la prise de d\'{e}cision, qu'\`{a} sonder les imaginaires qui entourent ces sc\'{e}narios. La prospective, dans une pure logique de l'action, consiste \`{a} objectiver ces sc\'{e}narios, \`{a} en d\'{e}gager des \'{e}l\'{e}ments de pr\'{e}diction. Au contraire, lorsque nous \'{e}tudions les futurs envisag\'{e}s, nous souhaitons sonder les imaginaires auxquels ils renvoient. En ce sens, il n'est pas exclu de prendre en compte les sc\'{e}narios les plus improbables, tant qu'ils restent concevables. C'est cette diff\'{e}rence qui est fondamentale, et qui explique l'int\'{e}r\^{e}t de cet objet pour notre \'{e}tude : la science-fiction est prospective, mais ne cherche pas n\'{e}cessairement \`{a} \^{e}tre pr\'{e}dictive. La question du probable concerne la discussion scientifique, tandis que le possible, nettement plus ouvert, est un espace d'expression privil\'{e}gi\'{e} pour l'imagination. 





	Ensuite, les futurs envisag\'{e}s se distinguent de ce qu'on peut appeler le champ des possibles. Cette notion, invent\'{e}e par Arlette Farge, et d\'{e}velopp\'{e}e par les tenants d'une histoire contrefactuelle\footnote{\textsuperscript{\newpage
}\textsuperscript{	DELUERMOZ,\ Quentin,\ SINGARAVELOU,\ Pierre.\ op.\ cit.\ }} , vise \`{a} resituer les \'{e}v\'{e}nements historiques dans un champ des possibles plus large, c'est-\`{a}-dire \`{a} envisager les diff\'{e}rentes possibilit\'{e}s qui s'offraient \`{a} l'Histoire, \`{a} un moment pr\'{e}cis. Ainsi, \'{e}tudier le champ des possibles, c'est avant tout d\'{e}gager l'\'{e}v\'{e}nement historique d'une explication lin\'{e}aire, parfois t\'{e}l\'{e}ologique ; c'est, en somme, rendre compte de l'existence d'autres chemins qu'aurait pu emprunter l'histoire. S'int\'{e}resser au champ des possibles, c'est donc questionner les possibles historiques, les futurs alternatifs plausibles qui auraient pu suivre un \'{e}v\'{e}nement historique donn\'{e}. Lorsque nous \'{e}voquons les futurs envisag\'{e}s, la d\'{e}marche est toute autre : nous cherchons \`{a} appr\'{e}hender les imaginaires auxquels renvoient ces tentatives de projection, peu importe qu'elles soient plausibles ou non. En ce sens, il n'est pas exclu de prendre en compte les sc\'{e}narios les plus improbables, tant qu'ils restent concevables. La litt\'{e}rature d'anticipation est prospective, mais ne cherche pas n\'{e}cessairement \`{a} \^{e}tre pr\'{e}dictive. La question du probable concerne la discussion scientifique, tandis que le possible, nettement plus ouvert, est un espace d'expression privil\'{e}gi\'{e} pour l'imagination. 





	Mais comment appr\'{e}hender l'histoire de ces futurs envisag\'{e}s ? C'est cette question qui marque le point de d\'{e}part de ces travaux de recherche. D'abord, comment caract\'{e}riser et comprendre ces futurs envisag\'{e}s? Ensuite, comment \'{e}tudier leur \'{e}volution historique ? Quel mat\'{e}riau utiliser, et quels m\'{e}thodes employer ? En proposant un support d'investigation, la litt\'{e}rature d'anticipation, et en d\'{e}veloppant une m\'{e}thode computationnelle pour explorer th\'{e}matiquement cet objet, ce m\'{e}moire participe \`{a} la construction d'un proc\'{e}d\'{e} d'investigation devant permettre d'\'{e}tudier les imaginaires projectionnistes, et leur \'{e}volution historique.





\textbf{\textcolor[rgb]{0.000,0.000,0.000}{	}}\textcolor[rgb]{0.000,0.000,0.000}{A quels \'{e}l\'{e}ments un auteur se r\'{e}f\`{e}re-t-il lorsqu'il imagine un futur possible ? Quelles th\'{e}matiques r\'{e}currentes observe-t-on lorsque les auteurs d'une m\^{e}me \'{e}poque d\'{e}crivent le futur ? Que nous disent ces th\'{e}matiques de cette \'{e}poque, et en particulier des imaginaires associ\'{e}s \`{a} l'avenir ? Lorsqu'on analyse les situations d\'{e}crites par le narrateur, et }\emph{\textcolor[rgb]{0.000,0.000,0.000}{a fortiori}}\textcolor[rgb]{0.000,0.000,0.000}{, les mots choisis pour d\'{e}crire cette situation, on sonde les r\'{e}f\'{e}rences communes, les fantasmes et angoisses qui structurent sa mani\`{e}re de penser l'avenir. La litt\'{e}rature d'anticipation appara\^{i}t, d\`{e}s lors, comme un objet litt\'{e}raire de choix pour \'{e}tudier la question des futurs envisag\'{e}s. Non seulement car les \oe{}uvres d'anticipation sont profond\'{e}ment ancr\'{e}es dans la r\'{e}alit\'{e} des auteurs qui les composent, mais aussi et surtout, en ce qu'elles permettent d'explorer les imaginaires collectifs auxquels se r\'{e}f\`{e}rent ces m\^{e}mes auteurs. Ainsi, lorsqu'un auteur \'{e}crit, il rend compte de craintes collectives, d'espoirs communs, d'interrogations propres \`{a} son \'{e}poque. R\'{e}ciproquement, la litt\'{e}rature d'anticipation contribue \`{a} nourrir ces imaginaires, \`{a} alimenter ces craintes, ces fantasmes}\textcolor[rgb]{0.000,0.000,0.000}{.}\emph{\textcolor[rgb]{0.000,0.000,0.000}{ }}\textcolor[rgb]{0.000,0.000,0.000}{Quelles angoisses, quels fantasmes caract\'{e}risent cette \'{e}poque ? Certaines technologies futures inspirent-elles la crainte ? Au contraire, certaines avanc\'{e}es techniques nourrissent-elles de grands espoirs ? En ce qu'elle est \`{a} la fois un r\'{e}servoir d'exp\'{e}riences de pens\'{e}e et des formes de probl\'{e}matisations, la litt\'{e}rature d'anticipation offre un mat\'{e}riau pertinent pour tenter de r\'{e}pondre \`{a} de telles questions, d\'{e}montrant tout son potentiel heuristique}\footnote{\textsuperscript{\textcolor[rgb]{0.000,0.000,0.000}{\newpage
}}\textsuperscript{\textcolor[rgb]{0.000,0.000,0.000}{	RUMPALA,\ Yannick.\ «\ Litt\'{e}rature\ \`{a}\ potentiel\ heuristique\ pour\ temps\ incertains\ »,\ in\ Methodos\ ,\ 15\ |\ 2015\ URL:\ http://journals.openedition.org/methodos/4178}}} \textcolor[rgb]{0.000,0.000,0.000}{. }





Nous allons maintenant r\'{e}fl\'{e}chir aux m\'{e}thodes par lesquelles nous analyserons ces futurs envisag\'{e}s. Commen\c{c}ons par \'{e}voquer le caract\`{e}re historique de cette \'{e}tude. Il s'agit bien ici de faire de l'histoire en utilisant la science-fiction comme support d'investigation, et en ayant recours \`{a} des m\'{e}thodes num\'{e}riques, comme nous l'exposerons par la suite. A travers la litt\'{e}rature de science-fiction que nous avons d\'{e}finie, nous cherchons donc \`{a} appr\'{e}hender l'\'{e}volution des futurs envisag\'{e}s sur un plan historique. On cherchera vraiment \`{a} adopter un regard d'historien sur la question, en mettant ces futurs envisag\'{e}s et les imaginaires auxquels ils renvoient en perspective avec le contexte de leur production. De plus, notre \'{e}tude pourra interroger, outre la mani\`{e}re dont ces imaginaires \'{e}voluent dans le temps, la mani\`{e}re dont ils \'{e}voluent dans l'espace, en comparant de la litt\'{e}rature issue de diff\'{e}rentes r\'{e}gions du monde, et empreinte de probl\'{e}matiques propres \`{a} l'espace culturel au sein duquel elles ont \'{e}t\'{e} produites. 





	Puisque nous \'{e}tudions uniquement des \oe{}uvres de litt\'{e}rature, nous conviendrons que les repr\'{e}sentations associ\'{e}es aux futurs envisag\'{e}s y sont d\'{e}crites essentiellement par des mots. Le choix des mots est donc d'une importance consid\'{e}rable : c'est par leur interm\'{e}diaire que le lecteur acc\`{e}de \`{a} des images, qu'il associe \`{a} des imaginaires. C'est donc \'{e}galement par leur interm\'{e}diaire que nous proposons d'interroger les futurs envisag\'{e}s. Ainsi, nous aurons recours \`{a} des m\'{e}thodes d'analyse s\'{e}mantique. Ce que l'on cherche \`{a} faire, c'est donc analyser les mots choisis par les auteurs, en fonction des \'{e}poques, pour d\'{e}crire ces futurs envisag\'{e}s, afin de comprendre leur sens et leur \'{e}volution. Ici, c'est au moyen de l'analyse th\'{e}matique que nous souhaitons explorer ces mots. Sur les m\'{e}thodes d'analyse th\'{e}matique et leur int\'{e}r\^{e}t, on pourra se r\'{e}f\'{e}rer \`{a} l'article de Pierre Zweigenbaum et Beno\^{i}t Habert, \emph{Acc\`{e}s mesur\'{e}s aux sens}\footnote{\textsuperscript{\newpage
}\textsuperscript{	ZWEIGENBAUM,\ Pierre,\ HABERT,\ Beno\^{i}t.\ {``}Acc\`{e}s\ mesur\'{e}s\ aux\ sens'',\ Mots.\ Les\ langages\ du\ politique,\ 74\ |\ 2004}} . Analyser les mots au moyen de m\'{e}thodes th\'{e}matiques, c'est essayer de d\'{e}terminer des th\`{e}mes r\'{e}currents, et des mots qui reviennent fr\'{e}quemment pour explorer ces th\'{e}matiques. On s'attend ainsi \`{a} observer des diff\'{e}rences th\'{e}matiques en fonction des \'{e}poques de production des oeuvres, et surtout, au sein m\^{e}me de ces th\'{e}matiques, des ensembles de mots tr\`{e}s vari\'{e}s, rendant compte des diff\'{e}rents futurs envisag\'{e}s. 





	Il nous faut donc trouver un moyen de traiter un volume tr\`{e}s important de donn\'{e}es textuelles, correspondant aux mots des \oe{}uvres \'{e}tudi\'{e}s, de mani\`{e}re syst\'{e}matique. On comprend, en effet, que l'\'{e}tude a beaucoup plus d'int\'{e}r\^{e}t si elle s'appuie sur un grand nombre de textes. Plus la pluralit\'{e} des textes et des auteurs est cons\'{e}quente, plus les r\'{e}sultats seront significatifs. Plus on compte de textes diff\'{e}rents pour une m\^{e}me \'{e}poque, plus on sera en mesure d'appr\'{e}cier finement les futurs envisag\'{e}s \`{a} cette \'{e}poque, les imaginaires desquels ils proc\`{e}dent. Justement, l'utilisation d'une m\'{e}thode computationnelle permet ici d'envisager un traitement massif des donn\'{e}es; si l'on trouve une m\'{e}thode efficace et probante, on n'aura donc aucun mal \`{a} la rendre syst\'{e}matique et \`{a} l'appliquer \`{a} un large corpus. De surcro\^{i}t, nous avons acc\`{e}s \`{a} des donn\'{e}es gratuites de mani\`{e}re quasi illimit\'{e}e : lorsqu'ils sont assez vieux, ces ouvrages sont tomb\'{e}s dans le domaine public et le texte est facilement accessible en ligne; lorsque ce n'est pas le cas, il suffit de trouver une version num\'{e}ris\'{e}e du texte en question.





	Nous proposons donc d'avoir recours \`{a} une m\'{e}thode relevant du champ des humanit\'{e}s num\'{e}riques, nous permettant de syst\'{e}matiser notre analyse gr\^{a}ce \`{a} un proc\'{e}d\'{e} computationnel, et \`{a} terme, de traiter un tr\`{e}s grand nombre de donn\'{e}es. On l'a dit, dans le cadre de notre \'{e}tude s\'{e}mantique, la capacit\'{e} \`{a} traiter un grand nombre de donn\'{e}es fait vraiment tout l'int\'{e}r\^{e}t de la m\'{e}thode, d'autant plus qu'on dispose d'un grand nombre de donn\'{e}es accessibles relativement facilement. Or, c'est pr\'{e}cis\'{e}ment l'int\'{e}r\^{e}t des m\'{e}thodes informatiques que nous proposons. D'abord, son faible co\^{u}t et sa r\'{e}employabilit\'{e} nous permettent de pr\'{e}tendre \`{a} \'{e}tendre l'analyse \`{a} un tr\`{e}s grand nombre d'oeuvres. L\`{a} où l'analyse litt\'{e}raire demande pour chaque oeuvre un travail fastidieux, les m\'{e}thodes num\'{e}riques, une fois bien \'{e}tablies et fonctionnelles, sont facilement r\'{e}utilisables sur d'autres textes. La capacit\'{e} d'analyse ne d\'{e}pend alors que de la disponibilit\'{e} des oeuvres et du temps de traitement des donn\'{e}es. 





	Ensuite, les m\'{e}thodes informatiques permettent une analyse rigoureusement syst\'{e}matique. L\`{a}, il faut \'{e}voquer la notion de \emph{distant reading}, d\'{e}velopp\'{e}e par Franco Moretti en 2000, dans un article qui a fait date\footnote{\textsuperscript{\newpage
}\textsuperscript{	Pour\ l'article\ en\ question,\ voir\ MORETTI,\ Franco.\ {``}Conjectures\ on\ World\ Literature'',\ New\ Left\ Review,\ 2000.\ On\ note\ aussi\ que\ cette\ publication,\ ainsi\ que\ d'autres\ articles\ dans\ lequel\ l'auteur\ poursuit\ sa\ r\'{e}flexion,\ ont\ \'{e}t\'{e}\ collect\'{e}s\ dans\ un\ ouvrage\ d\'{e}di\'{e}\ \`{a}\ la\ question\ du\ distant\ reading.\ Voir\ MORETTI,\ Franco.\ Distant\ Reading,\ Verso,\ 2013}} . Dans ce dernier, le professeur propose d'employer des m\'{e}thodes informatiques et statistiques pour analyser la litt\'{e}rature mondiale, et d\'{e}finit l'id\'{e}e de \emph{distant reading}, par opposition au \emph{close reading} que constitue l'analyse litt\'{e}raire par le corps du texte. Initialement con\c{c}u comme une m\'{e}thode compl\'{e}mentaire pour traiter un volume important de litt\'{e}rature hors du corpus de l'auteur, le \emph{distant reading} a progressivement d\'{e}velopp\'{e} des outils puissants, permettant de traiter \emph{per se} un corpus d'oeuvres destin\'{e} \`{a} l'analyse. Le \emph{distant reading} appara\^{i}t donc aujourd'hui comme une approche \`{a} part enti\`{e}re des \'{e}tudes litt\'{e}raires, se d\'{e}finissant par l'application de m\'{e}thodes computationnelles \`{a} des donn\'{e}es litt\'{e}raires. De mani\`{e}re int\'{e}ressante, cette lecture \`{a} distance permet aussi de s'affranchir de la subjectivit\'{e} du chercheur et des erreurs d'attention \'{e}ventuelles auquel s'expose n\'{e}cessairement le \emph{close reading}. C'est l\`{a} un apport majeur de ces m\'{e}thodes informatiques, qui permettent de syst\'{e}matiser l'analyse, c'est-\`{a}-dire de traiter toutes les donn\'{e}es uniform\'{e}ment et infailliblement. Bien entendu, l'interpr\'{e}tation des r\'{e}sultats obtenus via ces m\'{e}thodes rel\`{e}ve enti\`{e}rement du chercheur, et ce travail d'intelligibilit\'{e} doit rester un espace de subjectivit\'{e}; mais le recours \`{a} une approche informatique permet d'objectiver l'analyse des donn\'{e}es textuelles.





L'analyse statistique de donn\'{e}es textuelles, qu'on appelle parfois textom\'{e}trie, regroupe un ensemble de m\'{e}thodes aux usages tr\`{e}s vari\'{e}s. On distinguera, pour ce qui est du champ des sciences sociales et des humanit\'{e}s, deux principaux types d'applications. Le premier, tourn\'{e} vers l'analyse de la forme linguistique et du style, s'illustre lors de la datation ou de l'attribution de textes : \`{a} ce propos, l'\'{e}tude stylom\'{e}trique men\'{e}e en 2013 par Mike Kestemont, Sara Moens et Jeroen Deploige sur l'oeuvre d'Hildegarde de Bingen d\'{e}montre tout l'int\'{e}r\^{e}t de ces m\'{e}thodes\footnote{\textsuperscript{\newpage
}\textsuperscript{	KESTEMONT,\ Mike,\ MOENS,\ Sara,\ DEPLOIGE,\ Jeroen.\ {``}Collaborative\ authorship\ in\ the\ twelfth\ century\ :\ A\ stylometric\ study\ of\ Hildegard\ of\ Bingen\ and\ Guibert\ of\ Gembloux'',\ Literary\ and\ Linguistic\ Computing,\ 2013}} . Le deuxi\`{e}me s'int\'{e}resse davantage au contenu lexical et th\'{e}matique des textes, et correspond mieux \`{a} l'usage que nous souhaitons faire de ces m\'{e}thodes. Nous pensons ici plus pr\'{e}cis\'{e}ment aux m\'{e}thodes expos\'{e}es par les champs de la lexicom\'{e}trie, qui propose une \'{e}tude quantitative du lexique, et de la litt\'{e}rom\'{e}trie, qui fournit des outils num\'{e}riques plus sp\'{e}cifiquement adapt\'{e}s \`{a} l'analyse des textes litt\'{e}raires\footnote{\textsuperscript{\newpage
}\textsuperscript{	Pour\ une\ synth\`{e}se\ en\ fran\c{c}ais\ des\ m\'{e}thodes\ existantes,\ voir\ BERNARD,\ Michel,\ BOHET,\ Baptiste.\ Litt\'{e}rom\'{e}trie\ :\ outils\ num\'{e}riques\ pour\ l'analyse\ des\ textes\ litt\'{e}raires,\ Les\ fondamentaux\ de\ la\ Sorbonne\ Nouvelle,\ 2017}} . Parmi les nombreuses approches existantes, nous retiendrons principalement celles du \emph{topic modeling,} un ensemble de m\'{e}thodes de mod\'{e}lisation statistiques permettant la recherche de th\`{e}mes r\'{e}currents dans un corpus de documents. Fondant nos analyses sur ces mod\`{e}les, nous reviendrons plus longuement sur cette notion. 





	Pour pr\'{e}senter nos travaux, nous commencerons donc par expliquer plus en profondeur le choix de la litt\'{e}rature d'anticipation comme support d'investigation, en partant du questionnement d'historien initial et en expliquant le cheminement de pens\'{e}e qui a men\'{e} au choix de ce mat\'{e}riau, pour pr\'{e}senter finalement le corpus que nous avons constitu\'{e} pour cette \'{e}tude. Ensuite, nous reviendrons sur la m\'{e}thode du \emph{topic modeling}, en expliquant son fonctionnement, en pr\'{e}sentant le module que nous avons d\'{e}velopp\'{e} pour faciliter sa mise en oeuvre, et en rendant compte rigoureusement du proc\'{e}d\'{e} suivi pour appliquer cette m\'{e}thode \`{a} notre corpus. Enfin, nous pr\'{e}senterons les r\'{e}sultats ainsi obtenus, que nous commenterons en d\'{e}tail, avant d'ouvrir la discussion \'{e}pist\'{e}mologique sur notre m\'{e}thode et les diff\'{e}rentes perspectives d'enrichissement qui s'offrent \`{a} nous pour poursuivre en ce sens.


































































































































































































































































































\textbf{{\LARGE I) }}\textbf{{\LARGE \uline{La litt\'{e}rature d'anticipation comme support d'investigation}}}











\textbf{1. }Du questionnement d'historien\ldots{}








	Tout d'abord, revenons sur le questionnement d'historien qui constitue le point de d\'{e}part de ces recherches : comment appr\'{e}hender l'histoire des futurs envisag\'{e}s, c'est-\`{a}-dire des repr\'{e}sentations associ\'{e}es \`{a} ces projections ? Si cet objet d'\'{e}tudes appara\^{i}t comme encore relativement singulier et peu \'{e}tudi\'{e} dans la litt\'{e}rature acad\'{e}mique, du moins dans le monde francophone, notre questionnement, lui, s'inscrit dans une longue tradition historiographique de mise en perspective des repr\'{e}sentations et des imaginaires. Il convient donc de revenir sur les discussions scientifiques qui ont pr\'{e}c\'{e}d\'{e} notre r\'{e}flexion, pour mieux comprendre le cheminement de pens\'{e}e qui a pr\'{e}sid\'{e} \`{a} notre d\'{e}marche, ainsi que de dresser un \'{e}tat de l'art des recherches en la mati\`{e}re.








\textbf{	Vers une histoire des repr\'{e}sentations}





	L'id\'{e}e de s'int\'{e}resser \`{a} l'histoire des repr\'{e}sentations n'est pas n\'{e}e dans les ann\'{e}es 1980 avec le courant historiographique du m\^{e}me nom ; d\'{e}j\`{a}, l'histoire des mentalit\'{e}s, apparue dans les ann\'{e}es 1960 sous la houlette de Georges Duby et Robert Mandrou, avait port\'{e} un regard privil\'{e}gi\'{e} sur les imaginaires, en esquissant l'histoire de toutes formes de pens\'{e}es, sentiments ou croyances. Mais cette derni\`{e}re subissait de lourdes critiques concernant sa mani\`{e}re d'attribuer des \'{e}tats mentaux \`{a} des classes ou des cat\'{e}gories sociales enti\`{e}res, et son approche quantitative et s\'{e}rielle de ces \'{e}tats mentaux. En outre, cette histoire des mentalit\'{e}s reposait sur des concepts flous, des objets encore mal cern\'{e}s, comme en t\'{e}moigne l'expression consacr\'{e}e de Robert Mandrou qui d\'{e}finissait l'histoire des mentalit\'{e}s comme une {``}histoire des visions du monde''. Dans un article consacr\'{e} \`{a} la transition des mentalit\'{e}s aux repr\'{e}sentations, Michel Vovelle l'\'{e}voque d'ailleurs comme une d\'{e}finition {``}belle mais vague''. Pour d'autres historiens dont Vovelle, l'histoire des mentalit\'{e}s \'{e}tait en fait l'histoire des repr\'{e}sentations collectives. On voit, \`{a} travers ces multiples tentatives de d\'{e}finition, qu'il s'agissait encore d'un champs de recherche balbutiant, h\'{e}sitant sur les concepts m\^{e}me\footnote{	 Vovelle, Michel, et Christian-Marc Boss\'{e}no. « Des mentalit\'{e}s aux repr\'{e}sentations », \emph{Soci\'{e}t\'{e}s \& Repr\'{e}sentations}, vol. 12, no. 2, 2001, pp. 15-28.} . 





	Lorsque Jacques Le Goff coordonne la publication d'un ouvrage sur la {``}nouvelle histoire'' en 1978, et m\^{e}me si le terme de repr\'{e}sentation n'appara\^{i}t pas encore, il est bel et bien question de repr\'{e}sentations collectives, abord\'{e}es \`{a} travers l'histoire de l'amour, de la sexualit\'{e} ou de la mort. Il y a une forme de rupture avec l'histoire politique et sociale telle qu'on la pratiquait. On note aussi, quoiqu'avec un certain d\'{e}calage, l'influence du monde acad\'{e}mique anglo-saxon et en particulier des \emph{cultural }studies. On citera notamment les analyses classiques de Richard Hoggart sur la culture populaire, qui posent de solides bases m\'{e}thodologiques pour appr\'{e}hender les repr\'{e}sentations, et int\'{e}ressent tout particuli\`{e}rement nos recherches. Un de ses ouvrages majeurs,\emph{ La culture du pauvre -- The Uses of Literacy} en langue originale,\emph{ }dont on note qu'il ne sera publi\'{e} en France qu'en 1970, soit 13 ans apr\`{e}s sa publication en Grande-Bretagne, ouvre quelques pistes de r\'{e}flexion d'int\'{e}r\^{e}t pour l'analyse de la science-fiction, qui est une litt\'{e}rature fondamentalement populaire.





	Finalement, on observe dans la deuxi\`{e}me moiti\'{e} des ann\'{e}es 1980 en France une ru.pture marquant le passage de l'histoire des mentalit\'{e}s \`{a} l'histoire des repr\'{e}sentations. Une des figures embl\'{e}matiques de cette rupture est Roger Chartier, qui propose dans son article de r\'{e}f\'{e}rence de 1989 de {``}faire une histoire avec un retour du social \`{a} partir des repr\'{e}sentations que l'on s'en fait et qui le conditionnent''\footnote{	\textcolor[rgb]{0.000,0.000,0.000}{{\small  Chartier Roger. Le monde comme repr\'{e}sentation. In: }}\emph{\textcolor[rgb]{0.000,0.000,0.000}{{\small Annales. \'{E}conomies, Soci\'{e}t\'{e}s, Civilisations}}}\textcolor[rgb]{0.000,0.000,0.000}{{\small . 44ᵉ ann\'{e}e, N. 6, 1989. pp. 1505-1520.}}} . Les repr\'{e}sentations deviennent alors un moyen de faire de l'histoire sociale. Elles sont, comme l'exprimaient joliment Chartier, la mani\`{e}re par laquelle on rend pr\'{e}sent ce qui est absent. 





	D\'{e}finissons ici clairement ce que l'on entend par histoire des repr\'{e}sentations, puisque le terme fran\c{c}ais de \emph{repr\'{e}sentation} peut pr\^{e}ter \`{a} confusion -- on note sa polys\'{e}mie.  Alain Corbin en donne une d\'{e}finition claire et pr\'{e}cise, expliquant qu'elle se concentre sur l'histoire {``}[des] repr\'{e}sentations qu'un individu se fait du monde, d'un \'{e}ventuel au-del\`{a}, de lui-m\^{e}me et de l'autre. Ces repr\'{e}sentations qui r\`{e}glent le jeu du d\'{e}sir et de la r\'{e}pulsion ; qui d\'{e}cident des figures de l'angoisse et de l'horreur. Le syst\`{e}me de repr\'{e}sentations ne fait pas qu'ordonner le syst\`{e}me d'appr\'{e}ciation ; il d\'{e}termine les modalit\'{e}s de l'observation du monde, de la soci\'{e}t\'{e} »\footnote{	Corbin A., « {``}Le vertige des foisonnements'' Esquisse panoramique d'une histoire sans nom », dans \emph{Revue d'histoire moderne et contemporaine}, vol. 39, n°1, janvier-mars 1992, p. 117} . En d'autres termes, il s'agit d'une histoire des id\'{e}es, des images, des symboles, des mani\`{e}res dont on a pu se repr\'{e}senter le monde et la soci\'{e}t\'{e}. Pour faire le lien avec notre \'{e}tude, nous nous int\'{e}ressons \`{a} ces id\'{e}es, ces images, et  ces symboles, \`{a} ces {`}repr\'{e}sentations' en somme, lorsqu'elles concernent des projections du monde et de la soci\'{e}t\'{e} dans l'avenir.





\textcolor[rgb]{0.000,0.000,0.000}{	Par quels m\'{e}thodes, par quels proc\'{e}d\'{e}s construit-on cette histoire des repr\'{e}sentations ? Dans un premier temps, les historiens des repr\'{e}sentations se sont fond\'{e}s sur une analyse quantitative afin de visualiser l'\'{e}volution et les inflexions significatives des repr\'{e}sentations dans le temps. Ce qui est primordial lorsque l'on travaille sur l'histoire des repr\'{e}sentations, c'est l'\'{e}tude de ces derni\`{e}res avec une vision chronologique de celles-ci. Cette m\'{e}thode vient de l'influence des travaux d'}\href{https://fr.wikipedia.org/wiki/Ernest_Labrousse}{\textcolor[rgb]{0.000,0.000,0.000}{Ernest Labrousse}}\textcolor[rgb]{0.000,0.000,0.000}{ et }\href{https://fr.wikipedia.org/wiki/François_Simiand}{\textcolor[rgb]{0.000,0.000,0.000}{Fran\c{c}ois Simiand notamment}}\textcolor[rgb]{0.000,0.000,0.000}{, sur l'}\href{https://fr.wikipedia.org/wiki/Histoire_sociale}{\textcolor[rgb]{0.000,0.000,0.000}{histoire sociale}}\textcolor[rgb]{0.000,0.000,0.000}{. La mani\`{e}re d'\'{e}tudier les repr\'{e}sentations a ensuite \'{e}volu\'{e} vers l'\'{e}tude de cas concrets, par opposition avec l'approche quantitative. Cette \'{e}volution dans la m\'{e}thodologie de l'histoire des repr\'{e}sentations pousse les historiens \`{a} \'{e}tudier les individus repr\'{e}sentatifs de l'« exception normale », th\'{e}oris\'{e}e par }\href{https://fr.wikipedia.org/wiki/Carlo_Ginzburg}{\textcolor[rgb]{0.000,0.000,0.000}{Carlo Ginzburg}}\textcolor[rgb]{0.000,0.000,0.000}{. N\'{e}anmoins, selon }\href{https://fr.wikipedia.org/wiki/Michel_Vovelle}{Michel }\href{https://fr.wikipedia.org/wiki/Michel_Vovelle}{Vovelle}\textcolor[rgb]{0.000,0.000,0.000}{, ces deux approches ne sont pas n\'{e}cessairement oppos\'{e}es, mais peuvent \^{e}tre compl\'{e}mentaires. }





	Les sources disponibles pour l'histoire des repr\'{e}sentations peuvent \^{e}tre de diff\'{e}rents types. Dans un premier temps, dans la perspective d'analyses quantitatives, elles sont principalement compos\'{e}es de testaments, ex-voto, retables, etc. Aujourd'hui, elles se constituent principalement de t\'{e}moignages litt\'{e}raires ou de documents iconographiques\hypertarget{cite_ref-24}{}. L'histoire des repr\'{e}sentations s'appuie principalement sur les mots, les images et les symboles.. L'usage des mots est donc un moyen particuli\`{e}rement appr\'{e}ci\'{e} lorsqu'on s'int\'{e}resse \`{a} l'histoire des repr\'{e}sentations associ\'{e}es au futur, et il nous faut ici poser la question fastidieuse du lien existant entre litt\'{e}rature et repr\'{e}sentations.





\textbf{	}\textbf{Litt\'{e}rature et repr\'{e}sentations}





	Il existe un lien fort, ou plus pr\'{e}cis\'{e}ment, une forme de circularit\'{e} entre litt\'{e}rature et imaginaire social, repr\'{e}sentations collectives. La litt\'{e}rature, en ce qu'elle est le r\'{e}sultat d'une formalisation probl\'{e}matique de l'imaginaire social, entretient une relation privil\'{e}gi\'{e}e autant qu'ambig\"{u}e avec les imaginaires collectifs : {``}\emph{Un roman est en interaction dynamique majeure avec la narrativit\'{e} ambiante, c'est-\`{a}-dire avec les fa\c{c}ons de raconter l'histoire de la soci\'{e}t\'{e} et les \'{e}v\'{e}nements qu'elle vit}''\footnote{	https://journals.openedition.org/pratiques/1762} . Ainsi,   l'objet litt\'{e}raire constitue un mat\'{e}riau int\'{e}ressant pour \'{e}tudier l'histoire des imaginaires sociaux, et comprendre leur \'{e}volution,





	Ici, il nous faut mobiliser les travaux de la sociocritique pour comprendre la nature de ces liens r\'{e}ciproques, et la mani\`{e}re dont on peut acc\'{e}der \`{a} des repr\'{e}sentations \`{a} travers la litt\'{e}rature. Un article synth\'{e}tique de Pierre Popovic permet de bien comprendre les enjeux de la sociocritique, en m\^{e}me temps qu'il en donne une d\'{e}finition claire\footnote{	\textcolor[rgb]{0.000,0.200,0.200}{{\scriptsize Pierre Popovic}}\textcolor[rgb]{0.000,0.200,0.200}{{\scriptsize , « La sociocritique. D\'{e}finition, histoire, concepts, voies d'avenir », }}\emph{\textcolor[rgb]{0.000,0.200,0.200}{{\scriptsize Pratiques}}}\textcolor[rgb]{0.000,0.200,0.200}{{\scriptsize , 151-152 | 2011, 7-38.}}}  :





\emph{\textcolor[rgb]{0.000,0.000,0.000}{	{``}}}\emph{\textcolor[rgb]{0.000,0.000,0.000}{La sociocritique n'est ni une discipline ni une th\'{e}orie. Elle n'est pas non plus une sociologie, de quelque sorte qu 'elle soit, encore moins une m\'{e}thode. Elle constitue une perspective. A ce titre, elle pose comme principe fondateur une proposition heuristique g\'{e}n\'{e}rale de laquelle peuvent d\'{e}river de nombreuses probl\'{e}matiques individuellement coh\'{e}rentes et mutuellement compatibles. Cette proposition se pr\'{e}sente comme suit : Le but de la sociocritique est de d\'{e}gager la socialit\'{e} des textes. Celle-ci est }}\emph{\textcolor[rgb]{0.000,0.000,0.000}{analysable}}\emph{\textcolor[rgb]{0.000,0.000,0.000}{ dans les caract\'{e}ristiques de leurs mises en forme, lesquelles }}\emph{\textcolor[rgb]{0.000,0.000,0.000}{se comprennent}}\emph{\textcolor[rgb]{0.000,0.000,0.000}{ rapport\'{e}es \`{a} la semiosis sociale environnante prise en partie ou dans sa totalit\'{e}. L'\'{e}tude de ce rapport de commutation s\'{e}miotique permet d'}}\emph{\textcolor[rgb]{0.000,0.000,0.000}{expliquer}}\emph{\textcolor[rgb]{0.000,0.000,0.000}{ la forme-sens (th\'{e}matisations, contradictions, apories, d\'{e}rives s\'{e}mantiques, polys\'{e}mie, etc.) des textes, }}\emph{\textcolor[rgb]{0.000,0.000,0.000}{d'\'{e}valuer et de mettre en valeur}}\emph{\textcolor[rgb]{0.000,0.000,0.000}{ leur historicit\'{e}, leur port\'{e}e critique et leur capacit\'{e} d'invention \`{a} l'\'{e}gard du monde social. }}\emph{\textcolor[rgb]{0.000,0.000,0.000}{Analyser, comprendre, expliquer, \'{e}valuer}}\emph{\textcolor[rgb]{0.000,0.000,0.000}{, ce sont l\`{a} les quatre temps d'une herm\'{e}neutique. C'est pourquoi la sociocritique -- qui s'appellerait tout aussi bien « socios\'{e}miotique » -peut se d\'{e}finir de mani\`{e}re concise comme une }}\emph{\textcolor[rgb]{0.000,0.000,0.000}{herm\'{e}neutique sociale des textes.''}}





\textcolor[rgb]{0.000,0.000,0.000}{	Cette d\'{e}finition pos\'{e}e, on comprend que la sociocritique ait apport\'{e} beaucoup \`{a} l'\'{e}tude des liens entre litt\'{e}rature et discours social, \`{a} bien des niveaux diff\'{e}rents. Ce qui nous int\'{e}resse ici, c'est pr\'{e}cis\'{e}ment l'\'{e}tude des liens entre le texte litt\'{e}raire et le discours social, mais sur le plan des repr\'{e}sentations, des topiques, des motifs r\'{e}currents.}


\textcolor[rgb]{0.000,0.000,0.000}{Mais une telle \'{e}tude implique un important effort de d\'{e}finition du concept d'imaginaire social, et des modes d'interaction que la litt\'{e}rature entretient avec celui-ci. L\`{a} encore, Pierre Popovic en propose un expos\'{e} clair et succinct, que nous aimerions ici mobiliser :}





\textcolor[rgb]{0.000,0.000,0.000}{	}\emph{\textcolor[rgb]{0.000,0.000,0.000}{{``}}}\emph{\textcolor[rgb]{0.000,0.000,0.000}{L'imaginaire social est ce r\^{e}ve \'{e}veill\'{e} que les membres d'une soci\'{e}t\'{e} font, voient, lisent et entendent et qui leur sert d'horizon de r\'{e}f\'{e}rence pour tenter d'appr\'{e}hender, d'\'{e}valuer et de comprendre la r\'{e}alit\'{e} dans laquelle ils vivent. Il peut \^{e}tre pens\'{e} comme suit : }}





	L'imaginaire social est compos\'{e} d'ensembles interactifs de repr\'{e}sentations corr\'{e}l\'{e}es, organis\'{e}es en fictions latentes\hypertarget{bodyftn92}{}, sans cesse recompos\'{e}es par des propos, des textes, des chromos et des images, des discours ou des \oe{}uvres d'art. [...] Dans toute soci\'{e}t\'{e}, quatre de ces ensembles de repr\'{e}sentations sont essentiels : le premier concerne l'histoire et la structure de la soci\'{e}t\'{e} (repr\'{e}sentations du pass\'{e}, du pr\'{e}sent et de l'avenir ; repr\'{e}sentations des institutions, des hi\'{e}rarchies, des collectivit\'{e}s) ; le deuxi\`{e}me la relation entre l'individu et le collectif global (repr\'{e}sentation de l'individu, de sa vie, des rapports du priv\'{e} et du public) ; le troisi\`{e}me la vie erotique (repr\'{e}sentations des corps, des affects, des sentiments, du sexe) ; le quatri\`{e}me le rapport avec la nature (repr\'{e}sentations m\'{e}taphysiques, religieuses ou non religieuses, etc.).\\


\emph{\textcolor[rgb]{0.000,0.000,0.000}{	[\ldots{}] L'imaginaire social peut \^{e}tre vu comme le r\'{e}sultat de l'action de cinq modes majeurs de s\'{e}miotisation de la r\'{e}alit\'{e} [\ldots{}]  -- narrativit\'{e}, po\'{e}ticit\'{e}, cognitivit\'{e}, iconicit\'{e}, th\'{e}\^{a}tralit\'{e} -- signifie que l'imaginaire social est empreint de }}\emph{\textcolor[rgb]{0.000,0.000,0.000}{litt\'{e}rarit\'{e}. }}\emph{\textcolor[rgb]{0.000,0.000,0.000}{De l\`{a} d\'{e}coule une fa\c{c}on particuli\`{e}re de reposer la question nodale de la sociocritique. Ce qui est appel\'{e} « litt\'{e}rature » est ce qui r\'{e}sulte d'une formalisation probl\'{e}matique de l'imaginaire social, aussi bien au niveau du syst\`{e}me g\'{e}n\'{e}rique qu'au niveau des textes : un roman est en interaction dynamique majeure avec la narrativit\'{e} ambiante}}\hypertarget{bodyftn93}{}\href{https://journals.openedition.org/pratiques/1762#ftn93}{\textbf{\emph{\textcolor[rgb]{0.000,0.000,0.000}{93}}}}\emph{\textcolor[rgb]{0.000,0.000,0.000}{, c'est-\`{a}-dire avec les fa\c{c}ons de raconter l'histoire de la soci\'{e}t\'{e} et les \'{e}v\'{e}nements qu'elle vit. La mise en texte peut sans doute simplement calquer ces fa\c{c}ons de raconter, mais elle vise plus souvent -- dans les soci\'{e}t\'{e}s qui le permettent -- \`{a} les d\'{e}tourner, \`{a} les critiquer, \`{a} les transformer. Toute mise en forme vivace est alors dite probl\'{e}matique parce que les textes litt\'{e}raires sont susceptibles d'installer une distance s\'{e}miotique \`{a} l'int\'{e}rieur et \`{a} l'\'{e}gard de cet imaginaire social justement pour la raison qu'ils activent, individualisent les cinq modes de s\'{e}miotisation pr\'{e}d\'{e}crits}}\hypertarget{bodyftn94}{}\emph{\textcolor[rgb]{0.000,0.000,0.000}{.''}}





	Arm\'{e}s de ces pr\'{e}cieux concepts, on comprend l'int\'{e}r\^{e}t de la litt\'{e}rature comme support pour \'{e}tudier l'histoire des repr\'{e}sentations et l'\'{e}volution du discours social. Or, ce qui nous int\'{e}resse, c'est d'arriver \`{a} explorer les repr\'{e}sentations associ\'{e}es aux projections futuristes, la partie du discours social qui porte sur l'avenir. Pourquoi, alors, ne pas fonder notre analyse sur une litt\'{e}rature qui s'emploie \`{a} imaginer des fictions dans des mondes futurs ?








	Utiliser la SF pour faire de l'histoire des repr\'{e}sentations futuristes





	L'id\'{e}e d'utiliser la litt\'{e}rature de science-fiction pour sonder les repr\'{e}sentations n'est pas neuve. D\'{e}j\`{a} en 2011, dans l'avant-propos d'un num\'{e}ro de la revue \emph{Soci\'{e}t\'{e}s} consacr\'{e} \`{a} la science-fiction, les sociologues Wilfried Coussieu et Fr\'{e}d\'{e}ric Lebas d\'{e}montraient que la SF pouvait \^{e}tre appr\'{e}hend\'{e}e comme « une sociologie de l'imaginaire »\footnote{\textsuperscript{\newpage
}\textsuperscript{	LEBAS,\ Fr\'{e}d\'{e}ric,\ COUSSIEU,\ Wilfried.\ op.\ cit.}} .





	\emph{{``}La science-fiction exerce actuellement un champ d'influence que l'on pourrait qualifier de syst\'{e}mique. L'\'{e}ventail des hallucinations et des connaissances qu'elle met en jeu se situe \`{a} l'intersection complexe du savoir scientifique et des repr\'{e}sentations sociales, nourrissant l'un et l'autre de rapports vertigineux et contrast\'{e}s. Sa capacit\'{e} hybride \`{a} m\^{e}ler le vrai et le vraisemblable, sa mise en sc\`{e}ne des concepts dialectaux de la science rendus localement au service de la fiction, ainsi que son recours constant \`{a} des th\`{e}mes angoiss\'{e}s, anticipateurs et assoiff\'{e}s d'alternatives (utopiques ou dystopiques), semble \'{e}galement affirmer la porosit\'{e} de sens caract\'{e}ristique de la soci\'{e}t\'{e} postmoderne. Par ailleurs, l'\'{e}volution et l'impact actuel des franchises m\'{e}diatiques comme acc\'{e}l\'{e}ratrices de modes de consommation \'{e}labor\'{e}s sur des univers science-fictionnels, d\'{e}montrent l'int\'{e}r\^{e}t d'une v\'{e}ritable analyse \`{a} inscrire dans le champ des sciences humaines et sociales.'' }





	Ce court extrait synth\'{e}tise bien la pens\'{e}e des deux auteurs, et d\'{e}montre tout l'int\'{e}r\^{e}t de l'objet pour sonder les imaginaires d'une \'{e}poque. Mais c'est l\`{a} le propre de toutes les « litt\'{e}ratures de l'imaginaire », expression par laquelle on d\'{e}signe souvent le fantastique, la fantasy, et la science-fiction. Or, ce qui fait la sp\'{e}cificit\'{e} de la science-fiction, c'est qu'elle « s'attache \`{a} ce qui pourra/pourrait se produire, un jour, ailleurs »\footnote{\textsuperscript{\newpage
}\textsuperscript{	BESSON,\ Anne.\ {``}Aux\ fronti\`{e}res\ du\ r\'{e}el\ :\ les\ genres\ de\ l'imaginaire\ »,\ in\ La\ Revue\ des\ livres\ pour\ enfants\ (CNLJ-La\ Joie\ par\ les\ livres,\ BNF),\ n°\ 274,\ d\'{e}cembre\ 2013,\ dossier\ «\ Litt\'{e}ratures\ de\ l'imaginaire\ »,\ p.\ 86-93.}} . Par cons\'{e}quent, elle permet de cerner plus pr\'{e}cis\'{e}ment, au sein des imaginaires qu'elle explore, des projections d'avenir, des « mondes possibles »\footnote{\textsuperscript{\newpage
}\textsuperscript{	LAVOCAT,\ Fran\c{c}oise.\ op.\ cit.}}  : des futurs envisag\'{e}s.





{\footnotesize 	}On voit donc que la science-fiction est d\'{e}j\`{a} exploit\'{e}e comme un support d'investigation pour la recherche en sciences humaines et sociales. De mani\`{e}re plus r\'{e}cente, l'objet semble int\'{e}resser le prisme des humanit\'{e}s num\'{e}riques. Le 22 novembre 2018, une journ\'{e}e d'\'{e}tude consacr\'{e}e \`{a} « l'anticipation au prisme des humanit\'{e}s num\'{e}riques » se tenait \`{a} la maison de la recherche de l'universit\'{e} Paris-Sorbonne, montrant l'int\'{e}r\^{e}t des approches computationnelles et de la culture num\'{e}rique pour appr\'{e}hender un objet dont l'\'{e}tude appelle la transversalit\'{e}. C'est dans cette perspective que nous souhaitons inscrire nos travaux sur les futurs envisag\'{e}s.





	Il faut ici soulever un probl\`{e}me majeur pour notre d\'{e}marche. Lorsqu'on \'{e}tudie les futurs envisag\'{e}s, lorsqu'on sonde les imaginaires associ\'{e}s \`{a} une \oe{}uvre de science-fiction, sommes-nous r\'{e}ellement en train d'appr\'{e}cier les imaginaires collectifs d'une \'{e}poque ? Dans la mesure où l'\oe{}uvre renvoie surtout \`{a} l'imaginaire de l'auteur qui l'a compos\'{e}e, peut-on monter en g\'{e}n\'{e}ralit\'{e}, et mettre en lien cet imaginaire avec celui d'une \'{e}poque enti\`{e}re ? La science-fiction est souvent d\'{e}crite comme une litt\'{e}rature tr\`{e}s sp\'{e}cifique : on pourrait avancer que les futurs envisag\'{e}s que nous observerons \`{a} travers cette litt\'{e}rature ne proc\'{e}deront que des visions \'{e}sot\'{e}riques de l'auteur, ou au mieux d'un cercle d'initi\'{e}s, les\emph{ Fandoms}\footnote{\emph{\textsuperscript{\newpage
}}\emph{\textsuperscript{	Ce\ terme\ d\'{e}signe\ les\ lecteurs\ assidus\ de\ SF\ qui\ se\ r\'{e}unissent\ autour\ de\ magasins,\ newsgroups,\ forums,\ sites\ Internet\ sp\'{e}cialis\'{e}s\ et\ conventions.\ Il\ serait\ apparu\ lorsque\ les\ premiers\ lecteurs\ d'Amazing\ Stories\ se\ regroupaient\ pour\ partager\ leur\ passion.}}} \emph{, }qui envisageraient le futur de mani\`{e}re tr\`{e}s sp\'{e}cifique. C'est l\`{a} un biais \'{e}ventuel, dont on doit \^{e}tre conscient pour la suite de notre \'{e}tude. En effet, le raisonnement voulant que des th\'{e}matiques pr\'{e}sentes dans les romans de SF publi\'{e}s \`{a} une \'{e}poque donneraient acc\`{e}s aux th\'{e}matiques pr\'{e}sentes dans les repr\'{e}sentations collectives reste discutable. On pourrait facilement avancer que les th\'{e}matiques pr\'{e}sentes dans une litt\'{e}rature de grande diffusion comme la SF correspondent avant tout \`{a} ce que les acteurs de la production litt\'{e}raire de cette \'{e}poque -- auteurs en premier lieu, mais \'{e}galement \'{e}diteurs, critiques, journalistes -- estimaient \^{e}tre des th\'{e}matiques importantes, en r\'{e}sonnance avec leur temps. Les lecteurs ne sont pas oblig\'{e}s d'y adh\'{e}rer, quand bien m\^{e}me ils ach\`{e}tent, aiment, ou utilisent ces livres dans leurs pratiques discursives.





	Mais l'article de Wilfried Coussieu et Fr\'{e}deric Lebas montre justement bien que la science-fiction est d\'{e}sormais partie int\'{e}grante de la culture contemporaine, affirmant m\^{e}me que « personne n'est d\'{e}sormais \'{e}pargn\'{e} par la vague science-fictionnelle ». Surtout, les deux auteurs s'appuyant sur les concepts de H.R. Jauss sur l'esth\'{e}tique de la r\'{e}ception\footnote{\textsuperscript{\newpage
}\textsuperscript{	JAUSS,\ Hans\ Robert.\ Pour\ une\ esth\'{e}tique\ de\ la\ r\'{e}ception,\ Paris,\ Gallimard,\ coll.\ Tel,\ 1978,\ p.\ 55.\ }} , mettent en lumi\`{e}re l'importance des mondes possibles que la science-fiction imagine, des repr\'{e}sentations qu'elle forge, qui participent de la construction de consciences collectives et influencent m\^{e}me les dispositifs et conduites humaines. D'autant plus lorsque l'on consid\`{e}re uniquement la litt\'{e}rature de science fiction qui proc\`{e}de de l'anticipation r\'{e}aliste. En outre, il ne faut pas oublier de mentionner le caract\`{e}re collectif de cette litt\'{e}rature : les inventions ou concepts de certains auteurs sont r\'{e}employ\'{e}s par d'autres auteurs, constituant une sorte de r\'{e}pertoire commun. Ainsi, on observe une certaine circularit\'{e} entre repr\'{e}sentations collectives associ\'{e}es au futur et litt\'{e}rature de science-fiction : en m\^{e}me temps qu'elle t\'{e}moigne de ces imaginaires, elle les nourrit et contribue \`{a} les faire \'{e}voluer.





	Ainsi, la science-fiction appara\^{i}t aujourd'hui comme un genre riche et diversifi\'{e} qui, malgr\'{e} son caract\`{e}re fondamentalement populaire, a acquis ses lettres de noblesse sous la plume de grands auteurs, faisant aujourd'hui autorit\'{e} au-del\`{a} des cercles initi\'{e}s au genre. En outre, le genre s'est export\'{e} \`{a} l'international, et on compte de nombreuses sc\`{e}nes de la litt\'{e}rature de science-fiction \`{a} travers le monde, et dans diverses langues. On pourrait notamment citer les fr\`{e}res Strougatski en Russie, ou l'Espagnole Rosa Montero, dont le succ\`{e}s s'est \'{e}tendu \`{a} l'ensemble du monde hispanophone et au-del\`{a}. En dehors du monde occidental, \'{e}galement, la science-fiction se d\'{e}veloppe, notamment en Asie, par exemple au Japon et plus r\'{e}cemment, en Chine\footnote{\textsuperscript{\newpage
}\textsuperscript{	Pour\ approfondir\ sur\ la\ science-fiction\ chinoise,\ voir\ ALOÏSO,\ Loïc.\ «\ La\ science-fiction\ chinoise\ »,\ Impressions\ d'Extr\^{e}me-Orient\ 8\ |\ 2018,\ http://journals.openedition.org/ideo/772\ }} . Cette internationalisation du genre, amenant souvent des sp\'{e}cificit\'{e}s th\'{e}matiques r\'{e}gionales, contribue \`{a} dynamiser et \`{a} complexifier la science-fiction. Surtout, la science-fiction et les imaginaires auxquels elle renvoie semble omnipr\'{e}sente dans la culture contemporaine. A ce propos, le journaliste Ariel Kyrou a consacr\'{e} un article paru en 2016 dans la revue \emph{Multitudes}, dont le titre r\'{e}sumait bien la pens\'{e}e : « Nos subjectivit\'{e}s baignent dans un imaginaire de science-fiction »\footnote{\textsuperscript{\newpage
}\textsuperscript{	KYROU,\ Ariel,\ «\ Nos\ subjectivit\'{e}s\ baignent\ dans\ un\ imaginaire\ de\ science-fiction\ »,\ in\ Multitudes,\ 2016/1\ (n°\ 62),\ pp.\ 126-132.\ https://www.cairn.info/revue-multitudes-2016-1-page-126.htm\ }} . N\'{e}anmoins, l'article expose surtout, \`{a} raison, le r\^{o}le du cin\'{e}ma ou de l'industrie des jeux vid\'{e}o, voire d'institutions plus formelles comme la Nasa ou l'entreprise Google dans la profusion des imaginaires de science-fiction\footnote{\textsuperscript{\newpage
}\textsuperscript{	KYROU\ Ariel,\ op.cit.}} . Il est vrai que le r\^{o}le de la litt\'{e}rature a sans doute \'{e}t\'{e} moindre dans cette profusion. Cependant, elle a tiss\'{e} une toile de fond dans lequel le cin\'{e}ma et les jeux-vid\'{e}os sont venus puiser beaucoup de concepts, d'inventions, voire des univers tout entier. 








\textbf{	La place de la SF dans le monde acad\'{e}mique}





\emph{	}Il nous reste d\'{e}sormais \`{a} traiter la question de la place de la science-fiction dans le monde acad\'{e}mique, en dressant un bref aper\c{c}u de l'\'{e}tat de la recherche sur cet objet. En France, il faut souligner le peu d'int\'{e}r\^{e}t qu'a suscit\'{e} le genre jusque r\'{e}cemment. En 2012, dans un article intitul\'{e} « \'{E}tudier la science-fiction en France aujourd'hui », la professeure de litt\'{e}rature contemporaine Ir\`{e}ne Langlet \'{e}crivait : « La science-fiction est une pratique interm\'{e}diatique large et diversifi\'{e}e, aussi massivement pr\'{e}sente et reconnue dans la culture contemporaine que discr\`{e}te et m\'{e}connue dans le monde universitaire fran\c{c}ais »\footnote{\textsuperscript{\newpage
}\textsuperscript{	LANGLET\ Ir\`{e}ne,\ «\ \'{E}tudier\ la\ science-fiction\ en\ France\ aujourd'hui\ »,\ in\ ReS\ Futurae\ [En\ ligne],\ 1\ |\ 2012\ http://journals.openedition.org/resf/181}} . A une p\'{e}riode où la science-fiction faisait une entr\'{e}e discr\`{e}te dans le champ universitaire fran\c{c}ais, la professeure consacrait cet article \`{a} exposer les lacunes en la mati\`{e}re. Le genre semble n\'{e}anmoins conna\^{i}tre une vitalit\'{e} nouvelle depuis une dizaine d'ann\'{e}es dans le champ universitaire. En t\'{e}moigne l'ouvrage majeur publi\'{e} en 2012 par Simon Br\'{e}an\footnote{	BREAN Simon, \emph{La science-fiction en France: Th\'{e}orie et Histoire d'une litt\'{e}rature.} 2012, Presses Universitaires Paris-Sorbonne} sur l'histoire de la litt\'{e}rature de science-fiction francophone.





En outre, s'il est vrai que l'objet reste assez marginal pour ce qui est des \'{e}tudes litt\'{e}raires et stylistiques, une part croissante d'\'{e}tudes s'int\'{e}ressent \`{a} la science-fiction, comme comme support d'investigation dans des domaines vari\'{e}s. \'{E}tudi\'{e}e en sciences politiques\footnote{\textsuperscript{\newpage
}\textsuperscript{	RUMPALA\ Yannick,\ «\ Ce\ que\ la\ science-fiction\ pourrait\ apporter\ \`{a}\ la\ pens\'{e}e\ politique\ »,\ in\ Raisons\ politiques,\ 2010/4\ (n°\ 40),\ p.\ 97-113.\ URL:\ https://www.cairn.info/revue-raisons-politiques-2010-4-page-97.htm\ }} , en sociologie\footnote{\textsuperscript{\newpage
}\textsuperscript{	LEBAS,\ Fr\'{e}d\'{e}ric,\ COUSSIEU\ Wilfried,\ «\ Avant-propos.\ La\ science-fiction,\ litt\'{e}rature\ ou\ sociologie\ de\ l'imaginaire\ ?\ »,\ in\ Soci\'{e}t\'{e}s,\ 2011/3\ (n°113),\ p.\ 5-13.\ URL:					\ \ \ \ \ \ \ \ \ \ \ https://www.cairn.info/revue-societes-2011-3-page-5.htm\ }} , ou en litt\'{e}rature compar\'{e}e, avec des travaux remarqu\'{e}s sur la th\'{e}orie des mondes possibles\footnote{\textsuperscript{\newpage
}\textsuperscript{	LAVOCAT,\ Fran\c{c}oise,\ La\ th\'{e}orie\ litt\'{e}raire\ des\ mondes\ possibles,\ Paris,\ \'{E}ditions\ du\ CNRS.\ 2010}} , la science-fiction s'est attribu\'{e}e une place incontestable dans le paysage universitaire. Mais cette pr\'{e}sence est diss\'{e}min\'{e}e dans les disciplines les plus diverses et on note un vide dans le c\oe{}ur de discipline. Lou\'{e}e pour sa capacit\'{e} \`{a} saisir les imaginaires, et \`{a} les projeter dans l'avenir en inventant des mondes possibles, elle reste cependant un objet difficile \`{a} manipuler. D'abord, parce qu'elle s'est d\'{e}velopp\'{e}e en France comme une « subculture », la science-fiction se nourrit d'une pratique d'\'{e}rudition populaire qui peine \`{a} dialoguer avec le protocole acad\'{e}mique. Ensuite, son \'{e}tude n\'{e}cessite un travail d'articulation des diff\'{e}rents champs de la culture savante, des sciences exactes aux humanit\'{e}s, qui rend peu probante l'approche disciplinaire classique. Enfin, les institutions acad\'{e}miques peinent \`{a} concentrer les r\'{e}sultats de travaux divers et vari\'{e}s, proc\'{e}dant de disciplines diff\'{e}rentes. Si l'on sort du cadre fran\c{c}ais pour donner un aper\c{c}u de la recherche sur la science-fiction dans le monde, on ne sera pas surpris de trouver une litt\'{e}rature am\'{e}ricaine relativement abondante en la mati\`{e}re. Mais l\`{a} encore on note une disparit\'{e} dans les th\`{e}mes \'{e}tudi\'{e}s : tandis que la litt\'{e}rature tr\`{e}s contemporaine, du cyberpunk \`{a} nos jours, est tr\`{e}s largement sur-repr\'{e}sent\'{e}e, on note une absence de recherches sur la p\'{e}riode des \emph{pulps}, pourtant fondatrice\footnote{\textsuperscript{\newpage
}\textsuperscript{	LANGLET\ Ir\`{e}ne,\ op.\ cit.}} . D'une mani\`{e}re g\'{e}n\'{e}rale, si la science-fiction trouve d\'{e}sormais une place incontestable dans le monde universitaire, on voit donc que c'est un objet encore mal d\'{e}fini, aux enjeux mal cern\'{e}s, et qu'on peine encore \`{a} exploiter pleinement. 





On l'aura compris, la SF est un genre riche et foisonnant, qui rec\`{e}le d'images et d'id\'{e}es, et qui constitue un mat\'{e}riau de choix pour quiconque s'int\'{e}resse aux repr\'{e}sentations. Bien qu'encore relativement peu \'{e}tudi\'{e}e, le genre suscite un int\'{e}r\^{e}t croissant dans le monde acad\'{e}mique, et appelle de nombreux travaux. C'est aussi l'un des avatars de la production litt\'{e}raire {``}industrielle'' En cela, le recours aux Humanit\'{e}s Num\'{e}riques pour analyser cette litt\'{e}rature prend tout son sens, car elles permettent d'envisager de travailler plus facilement sur ces vastes corpus de donn\'{e}es. Par ailleurs, c'est un genre repr\'{e}sentatif d'une certaine viralit\'{e}, notamment dans le cadre des futurs imagin\'{e}s, puisqu'on y observe une circulation des th\`{e}mes et id\'{e}es entre les oeuvres. Il s'agit donc l\`{a} d'une ressource litt\'{e}raire id\'{e}ale pour faire de l'histoire des repr\'{e}sentations associ\'{e}es au futur. Il nous faut maintenant pr\'{e}senter plus longuement la litt\'{e}rature de science-fiction afin de bien d\'{e}finir le support d'investigation sur lequel fonder nos travaux.








\textbf{{\Large 2.}}{\Large  ...\`{a} la d\'{e}finition d'un support d'investigation adapt\'{e}}








	Apr\`{e}s avoir bien compris les enjeux et l'int\'{e}r\^{e}t d'une \'{e}tude plus approfondie des futurs envisag\'{e}s, il nous faut d\'{e}finir un support d'investigation adapt\'{e} \`{a} notre objectif. Au fur et \`{a} mesure de premiers travaux de recherche men\'{e}s dans le cadre du m\'{e}moire de M1, le choix de la litt\'{e}rature d'anticipation s'est progressivement pr\'{e}cis\'{e}, et il convient ici de r\'{e}sumer bri\`{e}vement ces travaux pour d\'{e}finir plus clairement cette litt\'{e}rature, et expliquer tout son potentiel en tant que mat\'{e}riau de recherche pour appr\'{e}hender les imaginaires projectionnistes.








	Science-fiction et litt\'{e}rature d'anticipation





\textbf{	}Revenons d'abord sur la notion de litt\'{e}rature d'anticipation, qui renvoie \`{a} une longue r\'{e}flexion men\'{e}e dans le cadre de mes travaux de recherche de M1. Nous souhaitions au d\'{e}part travailler sur la litt\'{e}rature de science-fiction mais, apr\`{e}s avoir analys\'{e} les sous-genres de la science-fiction, nous sommes parvenus \`{a} la conclusion qu'une part substantielle de la litt\'{e}rature de science-fiction ne permettait pas de sonder les {`}futurs envisag\'{e}s', en ce qu'elles d\'{e}crivaient des mondes tout \`{a} fait d\'{e}connect\'{e}s de la r\'{e}alit\'{e}. A l'inverse, l'anticipation ne constitue pas v\'{e}ritablement un genre de la litt\'{e}rature, mais plut\^{o}t un dispositif de narration. Ainsi, ce que j'appellerai ici litt\'{e}rature d'anticipation, et que j'ai choisi d'\'{e}tudier, renvoie en fait \`{a} une partie de la litt\'{e}rature de science-fiction qui proc\`{e}de de l'anticipation : revenons ici sur le cheminement de pens\'{e}e ayant men\'{e} au choix de la litt\'{e}rature d'anticipation.





	Initialement, nous avions l'intuition que la litt\'{e}rature de science-fiction pouvait \^{e}tre un support d'investigation int\'{e}ressant pour sonder les futurs envisag\'{e}s. La d\'{e}finition du dictionnaire Larousse explicite d'ailleurs tr\`{e}s clairement le lien qu'on peut faire entre projections futuristes et litt\'{e}rature SF, d\'{e}finissant cette derni\`{e}re comme « un genre litt\'{e}raire qui invente des mondes, des soci\'{e}t\'{e}s et des \^{e}tres situ\'{e}s dans des espaces-temps fictifs (souvent futurs) impliquant des sciences, des technologies et des situations radicalement diff\'{e}rentes ». rAinsi d\'{e}fini, l'objet de notre \'{e}tude soul\`{e}ve un probl\`{e}me majeur. Peut-on v\'{e}ritablement affirmer qu'un auteur de science-fiction « invente » des mondes « radicalement diff\'{e}rents » ? Dans la mesure où celui-ci d\'{e}crit un \'{e}tat futur du monde en sp\'{e}culant sur les avanc\'{e}es possibles de la science ou de la technologie, s'agit-il pour lui d'inventer un monde tout \`{a} fait nouveau, ou d'imaginer une suite possible \`{a} la r\'{e}alit\'{e} dans laquelle il \'{e}volue ? M\^{e}me si l'auteur projette son intrigue dans un monde nouveau, celle-ci reste fortement empreinte de consid\'{e}rations propres au r\'{e}el, au monde dans lequel l'auteur vit : pour le bien de la narration, ce dernier est tenu de pourvoir son r\'{e}cit de rep\`{e}res, d'\'{e}l\'{e}ments plus ou moins compr\'{e}hensibles auxquels le lecteur peut se r\'{e}f\'{e}rer. Quand bien m\^{e}me le monde d\'{e}crit dans certaines \oe{}uvres de science-fiction semble tr\`{e}s lointain du n\^{o}tre, la nature m\^{e}me du genre invite toujours le lecteur \`{a} imaginer, lorsqu'elle n'est pas sugg\'{e}r\'{e}e, une trame historique cr\'{e}dible, une suite d'\'{e}v\'{e}nements plausibles, faisant office de passerelle entre le monde de r\'{e}f\'{e}rence, dans lequel \'{e}volue l'auteur, et auquel peut se r\'{e}f\'{e}rer le lecteur,\textbf{ }et le monde fictif, dans lequel se d\'{e}roule l'histoire.





Ici, il faut r\'{e}-insister sur le lien qui existe entre la r\'{e}alit\'{e} de l'auteur qui compose ces \oe{}uvres, et le monde fictif qu'il imagine. C'est pr\'{e}cis\'{e}ment le caract\`{e}re envisageable du r\'{e}cit qui fait l'int\'{e}r\^{e}t de la SF pour notre \'{e}tude. \'{E}crire un ouvrage de science-fiction, c'est se projeter dans un ailleurs, souvent futur, \`{a} partir de son propre monde : c'est donc d'abord porter un regard sur son propre monde. En fin de compte, les \'{e}l\'{e}ments auxquels se r\'{e}f\`{e}rent ces r\'{e}cits renvoient \`{a} des probl\'{e}matiques propres au contexte dans lequel ces \oe{}uvres ont \'{e}t\'{e} produites. La science-fiction doit \`{a} ce titre \^{e}tre appr\'{e}hend\'{e}e comme une « litt\'{e}rature du r\'{e}el »\footnote{\textsuperscript{\newpage
}\textsuperscript{	COLSON,\ Rapha\"{e}l,\ RUAUD,\ Andr\'{e}-Fran\c{c}ois.\ La\ science-fiction,\ une\ litt\'{e}rature\ du\ r\'{e}el,\ Klincksieck}} , d\'{e}montrant l'int\'{e}r\^{e}t de cet objet d'\'{e}tude pour \'{e}tudier les imaginaires associ\'{e}s au futur. Or, et c'est l\`{a} tout le probl\`{e}me, toute science fiction ne se d\'{e}roule pas n\'{e}cessairement dans le futur, et m\^{e}me le cas \'{e}ch\'{e}ant, elle ne proc\`{e}de pas toujours du r\'{e}alisme, ou du vraisemblable. Si l'on cherche \`{a} appr\'{e}hender les futurs envisag\'{e}s \`{a} travers cette litt\'{e}rature, il nous faut donc restreindre notre champ d'\'{e}tude \`{a} la seule litt\'{e}rature qui s'attache \`{a} d\'{e}crire un futur envisageable pour l'homme, qu'on pourrait qualifier de litt\'{e}rature d'anticipation. Mais la litt\'{e}rature d'anticipation n'est pas un genre : elle est un dispositif, qui est, il faut le souligner, bien souvent superpos\'{e} au genre de la science-fiction, d'où une confusion notoire. 





C'est bien la litt\'{e}rature de science-fiction qui proc\`{e}de de l'anticipation qui int\'{e}resse nos recherches. Or, on l'a dit, toute science fiction ne proc\`{e}de pas de l'anticipation, et inversement, certaines des \oe{}uvres d'anticipation majeures n'appartiennent techniquement pas au genre de la SF : c'est par exemple le cas de \emph{1984} (1949) de George Orwell, ou du \emph{Meilleur des Mondes }(1932), d'Aldous Huxley, qui n'ont pas \'{e}t\'{e} publi\'{e}es dans des collections SF, alors m\^{e}me que les historiens du genre s'accordent \`{a} les qualifier de litt\'{e}rature de science-fiction, en ce qu'elles projettent toutes deux leur intrigue dans un monde futuriste aux technologies \'{e}volu\'{e}es\footnote{\textsuperscript{\newpage
}\textsuperscript{	}\textsuperscript{\ }\textsuperscript{{\tiny Ces\ oeuvres\ sont\ toutes\ deux\ mentionn\'{e}es\ comme\ des\ oeuvres\ de\ science-fiction\ dans\ les\ ouvrages\ suivants\ :}}\textsuperscript{.\ GOIMARD\ Jacques,\ AZIZA\ Claude.\ Encyclop\'{e}die\ de\ poche\ de\ la\ science\ fiction.\ Guide\ de\ lecture,\ Presses\ Pocket,\ coll.\ {``}Science-fiction'',\ 1986}\textsuperscript{{\tiny .\ GUIOT\ Denis.\ La\ Science-fiction,\ Massin,\ coll.\ {``}Le\ monde\ de\ldots{}'',\ 1987\ }}\textsuperscript{\ \ \ \ \ \ \ \ \ \ \ }} . Ce qu'on voit \'{e}merger, \`{a} travers cette tentative de d\'{e}finition probl\'{e}matique, c'est l'impossibilit\'{e} \`{a} d\'{e}finir clairement un objet complexe, qui s'est construit empiriquement, et qui appelle en fait une d\'{e}finition par l'histoire. Dressons donc un bref aper\c{c}u historique et th\'{e}matique de la science-fiction pour mieux cerner la part de cette litt\'{e}rature qui rel\`{e}ve de l'anticipation, et qui int\'{e}resse notre \'{e}tude.








Un objet historiquement construit





	Sans rentrer dans un d\'{e}bat fastidieux sur l'utilisation m\^{e}me de la notion de genre, il faut ici souligner qu'il n'existe pas v\'{e}ritablement de d\'{e}finition du genre de la science-fiction : davantage qu'un ensemble de textes coh\'{e}rent par leur nature, il se caract\'{e}rise par un usage fait de ces textes, une mani\`{e}re de les relier entre eux. Comme le montre tr\`{e}s bien l'historien de la litt\'{e}rature John Rieder dans un article remarqu\'{e} paru en novembre 2010 dans la revue \emph{science fiction Studies}\footnote{\textsuperscript{\newpage
}\textsuperscript{	RIEDER,\ John.\ « On\ defining\ SF,\ or\ Not :\ Genre\ Theory,\ SF\ and\ History »\ in\ science\ fiction\ Studies,\ vol.\ 37,\ n°116,\ novembre\ 2010,\ pp.\ 191-209}} , ce sont les acteurs de la production, de la distribution et de la consommation de cette litt\'{e}rature qui construisent le genre « par des actes de d\'{e}finition, cat\'{e}gorisation, inclusion et exclusion [\ldots{}] mais aussi par leur usage des protocoles et leurs strat\'{e}gies rh\'{e}toriques qui distinguent le genre d'autres formes d'\'{e}criture et de lecture ». En d'autres termes, en s'appropriant cette litt\'{e}rature, ses usages et ses codes, les auteurs, \'{e}diteurs, lecteurs contribuent collectivement \`{a} la d\'{e}finir. De fait, dresser un aper\c{c}u historique du genre, discuter de ses « origines », pour autant qu'elles soient clairement identifiables, constitue en soi une mani\`{e}re de d\'{e}finir le genre. Il faut bien comprendre que la science-fiction est un objet historiquement construit plus qu'une cat\'{e}gorie arr\^{e}t\'{e}e\footnote{\textsuperscript{\newpage
}\textsuperscript{	RIEDER,\ John.\ op.\ cit.}} . 





	Les historiens de la science-fiction ont longtemps cherch\'{e} des origines lointaines au genre ; certaines \oe{}uvres antiques, \`{a} l'instar de l'\emph{Histoire v\'{e}ritable }de Lucien de Samosate, compos\'{e}e au IIe si\`{e}cle, ont beaucoup int\'{e}ress\'{e} les sp\'{e}cialistes du genre\footnote{\textsuperscript{\newpage
}\textsuperscript{	FREDERICKS,\ S.C.\ {``}Lucian's\ True\ History\ as\ SF''\ in\ science\ fiction\ Studies,\ mars\ 1976,\ p.\ 49-60\ }} \footnote{\textsuperscript{\newpage
}\textsuperscript{	SWANSON,\ Roy\ Arthur\ Swanson.\ «\ The\ True,\ the\ False,\ and\ the\ Truly\ False:\ Lucian's\ Philosophical\ science\ fiction\ »\ ,\ science\ fiction\ Studies,\ Vol.\ 3,\ No.\ 3,\ novembre\ 1976,\ p.\ 227--239}} , qui y ont vu une forme de proto-science-fiction. A bien des \'{e}gards, il est vrai que la modernit\'{e} du r\'{e}cit, qui relate une guerre spatiale entre des peuples extraterrestres, le rapproche du genre. Plus tardivement, on peut consid\'{e}rer \emph{Utopia }(1516), de Thomas More, comme un r\'{e}cit proche de la SF, en ce qu'elle projette son intrigue dans un ailleurs, un \emph{non lieu}, un monde diff\'{e}rent. Cependant, nous prendrons le parti d'affirmer que la modernit\'{e} scientifique et technique du XIXe si\`{e}cle, ainsi que le passage \`{a} un r\'{e}gime d'historicit\'{e} dans lequel « le futur pr\'{e}vaut sur le pr\'{e}sent »\footnote{\textsuperscript{\newpage
}\textsuperscript{	Pour\ une\ bonne\ synth\`{e}se\ du\ concept\ de\ r\'{e}gime\ d'historicit\'{e}\ chez\ Fran\c{c}ois\ Hartog,\ voir\ Pascal\ Payen,\ «\ Fran\c{c}ois\ Hartog,\ R\'{e}gimes\ d'historicit\'{e}.\ Pr\'{e}sentisme\ et\ exp\'{e}riences\ du\ temps\ »,\ Anabases,\ 1\ |\ 2005,\ 295-298.\ }} , induit une v\'{e}ritable rupture, pr\'{e}alable \`{a} la constitution d'une litt\'{e}rature propre. Comme le souligne admirablement Jean Gatt\'{e}gno\footnote{\textsuperscript{\newpage
}\textsuperscript{	GATTEGNO,\ Jean.\ «\ La\ science-fiction\ »,\ col.\ «\ Que\ sais-je\ »,\ n°1426,\ 1971,\ p.9\ \ }} , « l'erreur de tout historien de la science-fiction est de n\'{e}gliger qu'il ne peut y avoir de science-fiction tant qu'il n'y a pas de sciences, et m\^{e}me de sciences appliqu\'{e}es ». Certes, certaines \oe{}uvres ant\'{e}rieures ont pu pr\'{e}senter des caract\'{e}ristiques tr\`{e}s semblables \`{a} celles des ouvrages que nous \'{e}tudions, mais ils rel\`{e}vent davantage de la fantaisie, du fantastique, voire du surnaturel. Or, tandis que ces derniers font intervenir des \'{e}l\'{e}ments surnaturels, parfois d'origine magique ou divine, qui ne n\'{e}cessitent pas de justification rationnelle, la science-fiction, elle, fournit une explication logique, scientifique, aux mondes qu'elle imagine\footnote{\textsuperscript{\newpage
}\textsuperscript{	BERTHELOT,\ Francis.\ Genres\ et\ sous-genres\ dans\ les\ litt\'{e}ratures\ de\ l'imaginaire,\ S\'{e}minaire\ de\ Narratologie,\ EHESS,\ 2004/2005}} .





	Au XIXe si\`{e}cle, \`{a} l'aune d'une modernit\'{e} nouvelle, la science-fiction se constitue notamment en litt\'{e}rature d'anticipation. C'est pr\'{e}cis\'{e}ment cet aspect qui int\'{e}resse nos travaux : l'ancrage du r\'{e}cit dans une r\'{e}alit\'{e} tangible, dans un monde qui appara\^{i}t comme un futur possible, ou du moins, envisageable, puisqu'il est parfois \`{a} la limite du vraisemblable. Les premiers r\'{e}cits d'anticipation datent du XVIIIe si\`{e}cle, mais ne sont pas encore empreints du r\'{e}alisme scientifique qui appara\^{i}t au XIXe si\`{e}cle\footnote{\textsuperscript{\newpage
}\textsuperscript{	ALKON,\ Paul.\ Origins\ of\ Futuristic\ Fiction,\ Georgia\ UP,\ 1987}} . Aussi, beaucoup consid\`{e}rent le roman \emph{Frankenstein} de Mary Shelley comme le premier v\'{e}ritable ouvrage de science-fiction, au sens où nous l'entendons : c'est en effet, le premier \`{a} faire le r\'{e}cit d'une fiction futuriste qui soit relativement plausible, acceptable, du moins envisageable pour les lecteurs contemporains de l'\'{e}poque. C'est d'ailleurs la position d\'{e}fendue par Brian Aldiss dans son essai \emph{Billion Year Spree.} Et l'auteur de poursuivre, lorsqu'il \'{e}voque\emph{ Frankenstein }:\emph{ }« \emph{The event on which this fiction is founded has been supposed, by Dr. Darwin, and some of the psysiological writers of Germany, as not of impossible occurrence »}\footnote{\textsuperscript{\newpage
}\textsuperscript{	}\textsuperscript{{\tiny ALDISS,\ Brian\ W.\ }}\emph{\textsuperscript{{\tiny Billion\ Year\ Spree}}}\textsuperscript{{\tiny ,\ Doubleday,\ 1973}}} . Remarquons cependant ici qu'il ne s'agit pas encore tout \`{a} fait d'un roman d'anticipation : l'auteur n'indique \`{a} aucun moment que l'histoire se d\'{e}roule dans le futur. Mais l'ouvrage marque une v\'{e}ritable rupture en ce qu'il est le premier \`{a} appara\^{i}tre, aux yeux de ses contemporains, comme scientifiquement probable.





	On observe \`{a} partir de la parution du roman de Mary Shelley et tout au long du XIXe si\`{e}cle, non seulement une profusion de la litt\'{e}rature de science-fiction, mais aussi une sp\'{e}cification du genre et de ses th\'{e}matiques. La projection dans un futur proche, aux sciences et technologies avanc\'{e}es, est un de ces motifs r\'{e}currents, qui s'illustre dans des \oe{}uvres comme \emph{The Mummy ! A Tale of the Twenty-Second Century} (1827) de Jane C. Loudon, \emph{Le Roman de l'Avenir} (1834) de F\'{e}lix Bodin, ou \emph{Le Monde tel qu'il sera} (1846) d'Emile Souvestre, qui s'attachent \`{a} imaginer l'avenir de l'humanit\'{e} dans un horizon relativement proche, du si\`{e}cle suivant pour les deux premiers, au XXIIe si\`{e}cle pour le roman de Jane Loudon. L'apparition plus tardive de la figure du robot, dans un roman de Samuel Butler intitul\'{e}\emph{ De l'autre c\^{o}t\'{e} des montagnes} (1872),\emph{ }est d'ailleurs assez r\'{e}v\'{e}latrice de la mani\`{e}re dont l'usage d'une science et d'une technologie de pointe structure le genre -- on parle bien de \emph{science}-fiction, apr\`{e}s tout. On retrouve aussi, souvent, la question de l'alt\'{e}rit\'{e}, qui s'illustre par la rencontre avec une civilisation \'{e}trang\`{e}re, dans bien des cas de nature extraterrestre. \emph{Star ou $\Psi$ de Cassiop\'{e}e} (1854) de Charlemagne Ischir Defontenay, ou \emph{La pluralit\'{e} des mondes habit\'{e}s} (1862) de Camille Flammarion, en sont de bons exemples. Enfin, il faut relever l'apparition d'un sous-genre plus rare, et plus difficile \`{a} traiter dans le cadre de notre \'{e}tude, mais sur lequel nous reviendrons : celui de l'uchronie. L'ouvrage \emph{Napol\'{e}on et la conqu\^{e}te du Monde} (1836) dans lequel Louis Geoffroy imagine le monde tel qu'il serait si Napol\'{e}on avait conquis Moscou, et \'{e}tablit une monarchie universelle. L'auteur ne se projette pas v\'{e}ritablement dans l'avenir : il imagine un futur non advenu au pass\'{e}\footnote{\textsuperscript{\newpage
}\textsuperscript{	Sur\ l'uchronie,\ voir\ BAUDOU,\ Jacques.\ op.\ cit.\ p.\ 91}} . Toujours est-il que la tentative d'imaginer un futur possible \`{a} partir d'un \'{e}tat donn\'{e} du monde reste au c\oe{}ur de l'id\'{e}e de science-fiction. 





	Cette litt\'{e}rature trouve, dans la deuxi\`{e}me moiti\'{e} du XIXe si\`{e}cle, de puissants \'{e}chos, sous les plumes de Jules Verne, et Herbert G. Wells, qui restent aujourd'hui consid\'{e}r\'{e}s comme les {``}p\`{e}res fondateurs du genre''\footnote{\textsuperscript{\newpage
}\textsuperscript{	}\textsuperscript{{\tiny L'expression\ {``}p\`{e}res\ fondateurs''\ est\ ici\ emprunt\'{e}e\ \`{a}\ Jacques\ Baudou,\ dans\ un\ chapitre\ qu'il\ consacre\ \`{a}\ Jules\ Vernes\ et\ Herbert\ G.\ Wells.\ Voir\ BAUDOU,\ Jacques.\ }}\emph{\textsuperscript{{\tiny La\ science-fiction}}}\textsuperscript{{\tiny ,\ Que\ sais-je?,\ 2003}}} . Ces auteurs sont, en effet, les premiers \`{a} avoir recours au proc\'{e}d\'{e} narratif de l'anticipation, tout en ancrant leur r\'{e}cit dans des consid\'{e}rations scientifiques. Des \oe{}uvres comme \emph{De la Terre \`{a} la Lune} (1865) et \emph{Vingt mille lieues sous les mers} (1869) pour le premier, ou \emph{La Machine \`{a} explorer le temps }(1895) et \emph{La guerre des mondes }(1898) pour le second, illustrent bien l'omnipr\'{e}sence des th\'{e}matiques \'{e}voqu\'{e}es pr\'{e}c\'{e}demment. C'est notamment gr\^{a}ce \`{a} ces auteurs que le genre s'exporte aux \'{E}tats-Unis, où il trouvera un nouvel essor au si\`{e}cle suivant. Cette \'{e}poque est aussi marqu\'{e}e par l'alphab\'{e}tisation croissante des soci\'{e}t\'{e}s occidentales, et le d\'{e}veloppement d'une litt\'{e}rature populaire grandement diffus\'{e}e par les revues, qui permettent l'\'{e}mergence d'un lectorat populaire encore relativement limit\'{e}, mais toujours plus nombreux.





	C'est v\'{e}ritablement \`{a} partir du XXe si\`{e}cle que le genre se popularise pour conna\^{i}tre, entre les ann\'{e}es 1930 et 1950, son « \^{a}ge d'or ». On l'a dit, le genre a \'{e}t\'{e} tr\`{e}s bien re\c{c}u aux Etats-Unis, où il inspire de nombreux auteurs. Or, les d\'{e}buts du XXe si\`{e}cle voient na\^{i}tre, dans ce m\^{e}me pays, les \emph{pulp magazines}. H\'{e}ritage direct des \emph{dime novels }(« romans \`{a} trois sous »), ces magazines imprim\'{e}s sur des mat\'{e}riaux de pi\`{e}tre qualit\'{e} se caract\'{e}risent par leur faible co\^{u}t et leur grande accessibilit\'{e}, et proposent des fictions de toutes sortes, destin\'{e}es \`{a} divertir un public tr\`{e}s large. Ils publient, entre autres, de nombreuses histoires de SF, qui connaissent un v\'{e}ritable succ\`{e}s aupr\`{e}s du grand public. Gr\^{a}ce \`{a} ce moyen de diffusion, la science-fiction conquiert un lectorat de plus en plus large et devient, d\`{e}s lors, une litt\'{e}rature fondamentalement populaire\footnote{\textsuperscript{\newpage
}\textsuperscript{	Pour\ une\ bonne\ synth\`{e}se\ en\ fran\c{c}ais\ sur\ l'histoire\ des\ pulp\ magazines,\ voir\ SAINT-MARTIN,\ Francis.\ Les\ pulps\ -\ l'\^{a}ge\ d'or\ de\ la\ litt\'{e}rature\ populaire\ am\'{e}ricaine,\ Encrage,\ 2000}} . En 1926, Hugo Gernsback cr\'{e}\'{e}e \emph{Amazing Stories}, le premier magazine consacr\'{e} exclusivement aux histoires de science-fiction. C'est dans les colonnes de ce magazine que le terme de science-fiction entre en usage. Bien qu'on le retrouve de mani\`{e}re isol\'{e}e, en 1851, sous la plume de l'auteur britannique William Wilson, c'est l'\'{e}diteur H. Gernsback qui r\'{e}pand son usage aux \'{E}tats-Unis et ailleurs\footnote{\textsuperscript{\newpage
}\textsuperscript{	}\textsuperscript{{\tiny Informations\ tir\'{e}es\ du\ centre\ national\ de\ ressources\ textuelles\ et\ lexicales.\ URL:}}\href{https://www.cnrtl.fr/etymologie/science-fiction}{\textsuperscript{{\tiny \ }}}\href{https://www.cnrtl.fr/etymologie/science-fiction}{\textsuperscript{\textcolor[rgb]{0.000,0.082,0.361}{{\tiny \uline{https://www.cnrtl.fr/etymologie/science-fiction}}}}}} . 





	A partir des ann\'{e}es 1930, on assiste \`{a} une multiplication des magazines sp\'{e}cialis\'{e}s dans la science-fiction. Au total, plus d'une trentaine de revues du genre\emph{ }sont cr\'{e}\'{e}es aux \'{E}tats-Unis pendant l'entre deux-guerres, faisant du pays le v\'{e}ritable \'{e}picentre de la production de science-fiction \`{a} cette \'{e}poque.\textbf{ }On l'a dit, les\emph{ pulp} sont des magazines \`{a} la production peu co\^{u}teuse, qui permettent un  tr\`{e}s grand nombre de tirages. Parfois, on compte jusqu'\`{a} 500 000 exemplaires pour une m\^{e}me revue\footnote{\textsuperscript{\newpage
}\textsuperscript{	SAINT-MARTIN,\ Francis.\ op.\ cit.}} . Parmi les plus embl\'{e}matiques, on compte notamment\emph{ Weird Tales},\emph{ Wonder Stories} et\emph{ Astounding Stories}. Bien que les\emph{ pulp magazin}es\emph{ }peinent \`{a} s'exporter hors du territoire am\'{e}ricain, la science-fiction se d\'{e}veloppe \'{e}galement au Royaume-Uni, en Allemagne, et en France, gr\^{a}ce \`{a} l'industrialisation de la presse et \`{a} la baisse du co\^{u}t des tirages. Le caract\`{e}re populaire de la science-fiction se traduit aussi par l'apparition du genre au cin\'{e}ma, d\`{e}s le d\'{e}but du XXe si\`{e}cle avec \emph{Le Voyage dans la lune} (1902) de Georges M\'{e}li\`{e}s. Sans trop s'\'{e}tendre sur le sujet, on retiendra de ce passage \`{a} l'\'{e}cran qu'il ouvre le genre \`{a} un public toujours plus large.





	Entre 1930 et 1950, la science-fiction conna\^{i}t donc un « \^{a}ge d'or »\footnote{\textsuperscript{\newpage
}\textsuperscript{	BAUDOU,\ Jacques.\ op.\ cit.\ p.\ 32}} : l'actualit\'{e} br\^{u}lante de la p\'{e}riode avec la mont\'{e}e des totalitarismes, la Seconde Guerre mondiale et les d\'{e}buts de la Guerre Froide, ainsi que l'\'{e}mergence de nouvelles technologies destructrices, \`{a} l'instar de la bombe nucl\'{e}aire, inspirent une g\'{e}n\'{e}ration d'auteurs. En 1937, John W. Campbell, un \'{e}crivain et \'{e}diteur de SF, prend la direction de la r\'{e}daction d'\emph{Astounding SF}. Il met en place une nouvelle politique \'{e}ditoriale, qui s'articule autour d'id\'{e}es et de sp\'{e}culations sur les sciences exactes. Il veut une SF d'ing\'{e}nieur : de mani\`{e}re tr\`{e}s imag\'{e}e, on parle souvent de « hard science-fiction » pour qualifier le style des auteurs de cette p\'{e}riode, en r\'{e}f\'{e}rence aux sciences dures (« hard sciences »). On compte par exemple parmi eux Isaac Asimov, inventeur du terme de « robotique » et concepteur des fameuses « trois lois de la robotique », ou encore le britannique Arthur C. Clarke. Les bouleversements politiques de la p\'{e}riode sont aussi un cadre propice \`{a} l'\'{e}mergence d'une litt\'{e}rature qui embrasse les th\`{e}mes politiques, et que certains qualifient de \textbf{\emph{politic fiction}}\footnote{\textsuperscript{\newpage
}\textsuperscript{	BERTHELOT,\ Francis.\ op.\ cit.}} . En 1932 para\^{i}t\emph{ Le Meilleur des mondes }d'Aldous Huxley, titre embl\'{e}matique qui pose assez t\^{o}t la question des risques de la mont\'{e}e des totalitarismes, de l'uniformisation et de l'eug\'{e}nisme. L'imm\'{e}diat apr\`{e}s-guerre est tout aussi marqu\'{e} par cette r\'{e}flexion, avec la parution en 1949 d'un titre embl\'{e}matique : \emph{1984}, de George Orwell, aura une tr\`{e}s grande port\'{e}e, attestant l'influence croissante de la science-fiction au sein des soci\'{e}t\'{e}s occidentales. 





	Dans les ann\'{e}es 1960, en effet, la science-fiction mute encore profond\'{e}ment, notamment sur le plan \'{e}ditorial. Depuis la cr\'{e}ation d'\emph{Amazing Stories} en 1926, les revues sp\'{e}cialis\'{e}es ont \'{e}t\'{e} le support quasi unique de la litt\'{e}rature de SF. Au d\'{e}but des ann\'{e}es 60, le march\'{e} des revues reste domin\'{e} par celle de J.W. Campbell, rebaptis\'{e}e \emph{Analog}, et d'autres revues comme \emph{Galaxy SF} et \emph{The Magazine of Fantasy and science-fiction} continuent d'\^{e}tre publi\'{e}es avec un certain succ\`{e}s\footnote{\textsuperscript{\newpage
}\textsuperscript{	BERTHELOT,\ Francis.\ op.\ cit.}} . Mais la science-fiction d\'{e}laisse progressivement les magazines sp\'{e}cialis\'{e}s pour investir le monde de l'\'{e}dition en livres. Jusqu'au d\'{e}but des ann\'{e}es 1960, les \oe{}uvres de SF ne paraissaient que tr\`{e}s rarement en librairie, m\^{e}me si quelques passionn\'{e}s du genre avaient d\'{e}j\`{a} cr\'{e}\'{e} des structures \'{e}ditoriales sp\'{e}cialis\'{e}es pour conserver les meilleures \oe{}uvres de SF, sous la forme d'ouvrages reli\'{e}s sous jaquette. Dans les ann\'{e}es 1960, justement, les \'{e}diteurs g\'{e}n\'{e}ralistes se mettent, eux aussi, \`{a} publier de la SF en \'{e}ditions reli\'{e}es, \`{a} un moment où le livre de poche fait une petite r\'{e}volution dans le monde de l'\'{e}dition. De nombreuses \oe{}uvres existantes sont d'abord r\'{e}\'{e}dit\'{e}es par des collections de livre de poche, puis progressivement, ces collections publient de l'in\'{e}dit. La SF rejoint alors le circuit de la grande distribution : b\'{e}n\'{e}ficiant d'une meilleure exposition, elle gagne un lectorat encore plus large. Cette mutation induit \'{e}galement une mutation de la forme du r\'{e}cit : jusqu'alors, la science-fiction se composait essentiellement de r\'{e}cits courts, les \oe{}uvres longues paraissant sous la forme de feuilleton dans les magazines ; \`{a} partir des ann\'{e}es 1960, cependant, le roman devient le mode d'expression dominant du genre.





	La mutation est aussi d'ordre th\'{e}matique : la science-fiction s'enrichit ainsi de nombreux sous-genres au cours des ann\'{e}es 1960. S'inspirant de plus en plus des probl\`{e}mes contemporains, elle invite le lecteur \`{a} s'interroger sur la soci\'{e}t\'{e} dans laquelle il \'{e}volue. En projetant les probl\'{e}matiques de leur temps dans un futur proche et plus que jamais envisageable, ses auteurs portent un regard social, parfois militant, sur leur \'{e}poque. La multiplication des th\'{e}matiques abord\'{e}es, des probl\`{e}mes envisag\'{e}s, et de mani\`{e}re plus g\'{e}n\'{e}rale, de la production de science-fiction, font de cette litt\'{e}rature un genre dynamique, riche et diversifi\'{e}, ramifi\'{e} en de nombreux sous-genres\footnote{\textsuperscript{\newpage
}\textsuperscript{	BERTHELOT,\ Francis.\ op.\ cit.}} . Loin de pr\'{e}tendre \`{a} en dresser un tableau exhaustif, nous souhaitons ici \'{e}voquer les principaux genres existants et leur histoire, afin de donner un aper\c{c}u de cette profusion et de mieux d\'{e}finir, par la suite, la litt\'{e}rature que nous ciblerons pour notre \'{e}tude. Ce qu'on voit se dessiner, \`{a} travers cette description th\'{e}matique des genres, c'est une litt\'{e}rature de science-fiction diff\'{e}renci\'{e}e en fonction des aspects du futur qu'elle choisit d'interroger, mais aussi en fonction du niveau de vraisemblance, de ces futurs envisag\'{e}s.








	\textbf{Un aper\c{c}u th\'{e}matique des genres de la SF : entre anticipation et fiction}


	


\textbf{	}Apr\`{e}s ce d\'{e}tour n\'{e}cessaire sur l'histoire de la litt\'{e}rature de science-fiction, nous voil\`{a} arm\'{e}s pour donner un aper\c{c}u th\'{e}matique des sous-genres de la SF, et bien cerner leur niveau de vraisemblance respectif. En d\'{e}terminant \`{a} chaque fois s'il rel\`{e}ve davantage de l'anticipation ou de la fiction, nous t\^{a}cherons d'en revenir \`{a} notre questionnement initial, et de pr\'{e}ciser ensemble la part de la litt\'{e}rature SF qui rel\`{e}ve de l'anticipation.





	On a d\'{e}j\`{a} \'{e}voqu\'{e} bri\`{e}vement l'apparition de la \textbf{\emph{hard science}}\emph{ }\textbf{\emph{fiction}}\textbf{ }dans les ann\'{e}es 1930, sous l'influence de Campbell et d'une vague d'auteurs qu'incarnent Isaac Asimov et Arthur C. Clarke. Ces \oe{}uvres se caract\'{e}risent par le fait qu'elles ne d\'{e}veloppent que des th\'{e}ories plausibles s'appuyant sur les d\'{e}couvertes les plus r\'{e}centes dans les domaines scientifiques de la physique ou de l'astronomie. Ainsi, le genre impose \`{a} l'auteur une rigueur m\'{e}thodique lorsque ce dernier imagine des mondes. L'usage d'un discours scientifique en est d'ailleurs caract\'{e}ristique. Au d\'{e}but des ann\'{e}es 1960, le march\'{e} des revues reste domin\'{e} par celle de J.W. Campbell, rebaptis\'{e}e \emph{Analog}. \emph{Galaxy SF} et \emph{The Magazine of Fantasy and science-fiction}. Il s'agit pour l'auteur d'explorer le domaine des sciences exactes \`{a} travers la fiction, et on en retrouve de bons exemples, dans la revue\emph{ Amazing Stories}, notamment avec les auteurs susmentionn\'{e}s, mais aussi plus tardivement, sous forme de livres de poche. On pense notamment au \emph{Nuage Noir} (1957) de Fred Hoyle, ou \`{a} \emph{L'Anneau-monde} (1970) de Larry Niven. Ce sous-genre proc\`{e}de de l'anticipation, et s'attache particuli\`{e}rement au r\'{e}alisme scientifique des mondes qu'il d\'{e}crit : il  est donc caract\'{e}ris\'{e} par un haut niveau de vraisemblance.





	Il existe \'{e}galement une litt\'{e}rature d'anticipation qui s'attache davantage \`{a} la dimension politique des mondes qu'elle anticipe, et que certains qualifient de \textbf{\emph{politic fiction}}. Elle s'applique \`{a} imaginer, souvent d'un point de vue critique, les cons\'{e}quences \`{a} court ou long terme de choix politiques et sociaux contemporains. On pense notamment au \emph{Meilleur des mondes} (1932) d'Aldous Huxley, ou \`{a} \emph{1984} (1949) de George Orwell, qui projettent tous deux les dangers du totalitarisme et de l'uniformisation dans les mondes qu'ils imaginent. Comme on l'a expliqu\'{e}, ces \oe{}uvres appartiennent bien au genre de la science-fiction, m\^{e}me si elles n'ont pas \'{e}t\'{e} publi\'{e}es dans des collections SF. Notons toutefois que nous souhaitons \^{e}tre assez prudents avec le terme de \textbf{\emph{politic fiction}}, qui ne fait pas consensus, et auquel nous privil\'{e}gierons ici le concept de \textbf{\emph{litt\'{e}rature d'anticipation politique}}. On dira d'elle qu'elle conna\^{i}t une vitalit\'{e} nouvelle dans les ann\'{e}es 1960-70, notamment en Europe, sous les plumes de l'Anglais Anthony Burgess, avec \emph{L'Orange m\'{e}canique} (1962), et du Fran\c{c}ais Philippe Curval dans \emph{Le Cycle de l'Europe apr\`{e}s la pluie} (1976-1983). En outre, remarquons que l'\textbf{\emph{anticipation politique }}pr\'{e}sente souvent une vision dystopique d'un monde en proie aux cons\'{e}quences dramatiques des choix politiques et sociaux contemporains, contrairement \`{a} la \textbf{\emph{hard science fiction}} qui loue les progr\`{e}s scientifiques et technologiques des futurs utopiques qu'elle imagine. On note ici l'existence d'une polarit\'{e} r\'{e}currente entre des conceptions oppos\'{e}es du futur, qui int\'{e}ressera nos travaux\footnote{\textsuperscript{\newpage
}\textsuperscript{	Peter\ Fitting,\ «\ Utopies/Dystopie/Science-fiction\ :\ l'interaction\ de\ la\ fiction\ et\ du\ r\'{e}el\ »,\ paru\ dans\ Alliage,\ n°60\ -\ Juin\ 2007,\ II.\ Science\ fiction\ et\ politique\ :\ le\ prisme\ des\ utopies,\ Utopies/Dystopie/Science-fiction\ :\ l'interaction\ de\ la\ fiction\ et\ du\ r\'{e}el,\ mis\ en\ ligne\ le\ 01\ ao\^{u}t\ 2012,\ URL\ :\ http://revel.unice.fr/alliage/index.html?id=3494.}} .





\emph{	}L'introduction des sciences humaines dans la litt\'{e}rature de science-fiction donne naissance \`{a} un nouveau genre dans les ann\'{e}es 1960 : la \textbf{\emph{speculative fiction.}} Par opposition \`{a} la \textbf{\emph{hard science fiction}}, la fiction sp\'{e}culative met l'accent sur les id\'{e}es, sur l'\'{e}volution psychosociologique des soci\'{e}t\'{e}s qu'elle d\'{e}crit : les questions qu'elle pose appellent en ce sens des r\'{e}ponses moins tranch\'{e}es, moins scientifiques et rationnelles, d'où son caract\`{e}re « sp\'{e}culatif ». Parmi ce genre, on compte notamment la \emph{S\'{e}rie des quatre apocalypses} de J.G. Ballard qui, entre 1962 et 1966, explore la question de la destruction du monde en quatre volets consacr\'{e}s aux quatre \'{e}l\'{e}ments. La question de la r\'{e}alit\'{e} truqu\'{e}e s'illustre \'{e}galement \`{a} travers ce genre : on pense notamment aux ouvrages de Philip K. Dick, \emph{Le Temps d\'{e}sarticul\'{e}} (1959), et \emph{Ubik} (1969). Le genre inspire beaucoup des grandes \oe{}uvres de science-fiction dans les ann\'{e}es 1960-70, et explore davantage les aspects sociaux des futurs qu'elle envisage. Ainsi, par exemple, des questions du langage et de la communication, avec \emph{L'Ench\^{a}ssement }(1973) de Ian Watson, ou de l'ambivalence sexuelle, avec \emph{La Main gauche de la nuit }(1969) d'Ursula le Guin.





	Le sous-genre du \textbf{\emph{Space Opera}} conna\^{i}t \'{e}galement un essor important dans les ann\'{e}es 1960. Le terme avait \'{e}t\'{e} forg\'{e} en 1941 par l'\'{e}crivain Wilson Tucker qui, ironisant sur certaines \oe{}uvres de SF aux intrigues naïves, faisait l'analogie avec les « soap opera »\footnote{\textsuperscript{\newpage
}\textsuperscript{	BERTHELOT,\ Francis.\ op.\ cit.}} . Ces drames aux intrigues longues et complexes se caract\'{e}risent par un cadre g\'{e}opolitique futuriste, souvent intergalactique. Aussi, ils prennent souvent la forme de cycles litt\'{e}raires, en plusieurs opus, \`{a} l'instar du c\'{e}l\`{e}bre \emph{Cycle de Dune} (1965) de Frank Herbert ou des \emph{Cycle de Mars} (1917-1922) et \emph{Cycle de V\'{e}nus} (1934-1946) d'Edgar R. Burroughs. A partir des ann\'{e}es 1960, le courant devient tr\`{e}s populaire, et on voit appara\^{i}tre de nombreuses \oe{}uvres \textbf{\emph{space opera}} : marqu\'{e} par une grande vitalit\'{e}, le sous-genre s'enrichit des apports d'autres courants de la SF, et sera \`{a} l'origine de beaucoup des grands classiques de la science-fiction. N\'{e}anmoins on voit que le genre ne proc\`{e}de pas de l'anticipation, en ce qu'il n'attache que peu d'importance au niveau de vraisemblance du r\'{e}cit : la litt\'{e}rature \textbf{\emph{space opera}} ne se veut pas n\'{e}cessairement r\'{e}aliste. En outre, il d\'{e}crit souvent un ailleurs lointain, quitte \`{a} priver parfois le r\'{e}cit d'un lien direct avec l'Histoire humaine. Signalons ici qu'on retrouvera parfois une variante plus r\'{e}cente du \textbf{\emph{space opera}} sous le nom de \textbf{\emph{planet opera}}, dont le r\'{e}cit est centr\'{e} non sur l'espace tout entier mais sur une plan\`{e}te unique. On pense notamment \`{a} \emph{Lum'en} (2015), de Laurent Genefort.





	Il faut ici mentionner un autre sous-genre majeur : le \textbf{\emph{cyberpunk}} (en r\'{e}f\'{e}rence au mouvement culturel \emph{punk}), apparu dans les ann\'{e}es 1980. Le terme est employ\'{e} pour la premi\`{e}re fois en d\'{e}cembre 1984 dans un article du \emph{Washington Post}, intitul\'{e} « SF in the Eighties », pour qualifier l'\oe{}uvre remarqu\'{e}e de William Gibson,\emph{ Neuromancien}, parue la m\^{e}me ann\'{e}e\footnote{\textsuperscript{\newpage
}\textsuperscript{	BERTHELOT,\ Francis.\ op.\ cit.}} . Ce sous-genre qualifie des \oe{}uvres d'anticipation dystopiques, aliment\'{e}es par les grands progr\`{e}s scientifiques et techniques des ann\'{e}es 1980, notamment concernant l'informatique, Internet, la robotique et l'a\'{e}ronautique. Le genre d\'{e}veloppe, entre autres, les th\`{e}mes du cyberespace, et de l'homme augment\'{e}. Outre le roman fondateur \'{e}voqu\'{e}, on compte notamment \emph{Les androïdes r\^{e}vent-ils de moutons \'{e}lectriques }(1966) de Philippe K. Dick, plus connu sous le nom de son adaptation au cin\'{e}ma, \emph{Blade Runner}, ou encore, tr\`{e}s r\'{e}cemment, \emph{Des larmes sous la pluie} (2011), de Rosa Montero, qui se situe dans un prolongement du monde imagin\'{e} dans \emph{Blade Runner}. Le \textbf{\emph{cyberpunk}} inspire \'{e}galement un autre courant plus marginal dans les ann\'{e}es 1980 : celui du\textbf{\emph{ steampunk}}, en r\'{e}f\'{e}rence \`{a} la machine \`{a} vapeur, puisqu'il revisite les progr\`{e}s technologiques du XIXe si\`{e}cle. L\`{a}, on ne peut \'{e}videmment pas parler de litt\'{e}rature d'anticipation, puisque l'intrigue se situe dans le pass\'{e}. Le monde imagin\'{e} ne rel\`{e}ve donc pas d'un futur envisag\'{e}, mais plut\^{o}t d'un paradoxe temporel. Plus d\'{e}licat \`{a} manipuler, il n'int\'{e}ressera pas nos travaux pour le moment.





	Enfin, un sous-genre majeur, encore peu \'{e}voqu\'{e} jusqu'ici, doit \^{e}tre mentionn\'{e} : celui de l'\textbf{\emph{uchronie}}. De mani\`{e}re r\'{e}trospective, l'\textbf{\emph{uchronie}} imagine des pass\'{e}s alternatifs, des futurs non advenus, en revisitant notre histoire. On peut citer \emph{Autant en emporte le temps} (1953) de Ward Moore, qui explore les cons\'{e}quences qu'aurait eu une victoire des conf\'{e}d\'{e}r\'{e}s lors de la guerre de S\'{e}cession, ou \emph{Le Ma\^{i}tre du haut ch\^{a}teau} (1962), dans lequel Philippe K. Dick imagine une Am\'{e}rique partag\'{e}e en zones d'occupations allemande et japonaise apr\`{e}s la victoire de l'Axe. L\`{a} encore, cette litt\'{e}rature rel\`{e}ve du paradoxe temporel, et non de l'anticipation. Elle sera donc \'{e}cart\'{e}e \`{a} ce stade de nos travaux. On doit \'{e}galement mentionner l'existence de sous-genres caract\'{e}ris\'{e}es par la recherche stylistique, davantage que par le souci de r\'{e}alisme scientifique, comme la \textbf{\emph{New Wave}}. N\'{e} dans les ann\'{e}es 1960 au Royaume-Uni, le courant est popularis\'{e} par l'\'{e}crivain Michael Moorcock \`{a} travers le magazine \emph{New Worlds}, dont il a pris la direction \'{e}ditoriale en 1964. Cette litt\'{e}rature ne relevant pas non plus n\'{e}cessairement de l'anticipation, nous ne nous \'{e}tendrons pas plus sur le sujet.





	 Encore une fois, il convient d'insister sur la complexit\'{e} \`{a} d\'{e}terminer nettement des cat\'{e}gories de sous-genres qui sont en mutation permanente, et s'influencent mutuellement. Parfois, ils se recoupent m\^{e}me, comme c'est le cas avec la\textbf{\emph{ New Wave}} et la \textbf{\emph{Speculative fiction}}\footnote{\textsuperscript{\newpage
}\textsuperscript{	C'est\ notamment\ le\ cas\ de\ l'oeuvre\ d'Ursula\ K.\ Le\ Guin,\ orient\'{e}e\ de\ consid\'{e}rations\ sociales\ propres\ \`{a}\ la\ speculative\ fiction,\ mais\ marqu\'{e}e\ par\ une\ recherche\ stylistique\ caract\'{e}ristique\ de\ la\ New\ Wave.\ On\ peut\ notamment\ citer\ La\ Main\ gauche\ de\ la\ nuit\ (1969)}} .\textbf{\emph{ }}A cet \'{e}gard, les travaux de Francis Berthelot montrent bien la porosit\'{e} des genres de la science-fiction, dont les limites sont souvent floues\footnote{\textsuperscript{\newpage
}\textsuperscript{	BERTHELOT,\ Francis.\ op.\ cit.}} . Loin d'\^{e}tre arr\^{e}t\'{e}e, cette cat\'{e}gorisation des sous-genres est perp\'{e}tuellement remise en cause par l'apparition de nouveaux courants, de nouvelles tendances, et par l'internationalisation de la science-fiction. Toujours est-il qu'ils donnent un aper\c{c}u de la grande diversit\'{e} de sous-genres qui se d\'{e}veloppent dans la seconde moiti\'{e} du XXe si\`{e}cle, et permettent d'expliquer la richesse actuelle de la SF. Tr\`{e}s dynamique, le genre est marqu\'{e} par une recherche permanente de renouveau, et puise son inspiration dans les probl\'{e}matiques socio-politiques contemporaines \'{e}mergentes, ce qui complique encore notre tentative de cat\'{e}gorisation. N\'{e}anmoins, ce travail pr\'{e}alable nous donne un aper\c{c}u th\'{e}matique des genres, et permet de bien cerner la part de cette litt\'{e}rature qui proc\`{e}de de l'anticipation dans son volet le plus r\'{e}aliste. 





	
























































\textbf{{\Large 3.}}{\Large  La constitution d'un corpus de litt\'{e}rature d'anticipation}








{\Large 	}Au terme de ces premi\`{e}res recherches, nous avons donc pos\'{e} un cadre pr\'{e}alable \`{a} la d\'{e}finition d'un support d'investigation adapt\'{e} \`{a} l'\'{e}tude des futurs envisag\'{e}s, la litt\'{e}rature d'anticipation, en d\'{e}montrant tout son potentiel heuristique. Il nous reste alors \`{a} constituer un v\'{e}ritable corpus d'analyse. Nous reviendrons ici sur les diff\'{e}rents probl\`{e}mes \'{e}pist\'{e}mologiques soulev\'{e}s par la constitution de notre corpus et sur les \'{e}l\'{e}ments de r\'{e}flexion que nous avons apport\'{e} \`{a} ces questions, avant de pr\'{e}senter plus en d\'{e}tail le corpus ainsi compos\'{e}.








	D\'{e}finir clairement la litt\'{e}rature d'anticipation





	Il doit d\'{e}sormais nous permettre de d\'{e}finir clairement ce que nous entendrons par litt\'{e}rature d'anticipation, qui se restreindra, \`{a} une partie seulement de la litt\'{e}rature de science-fiction. D'abord, \'{e}cartons les sous-genres de l'\textbf{\emph{uchronie}}\textbf{ }et du \textbf{\emph{steampunk}},\textbf{\emph{ }}puisqu'ils pr\'{e}sentent des paradoxes temporels. En effet, les mondes d\'{e}crits dans ces \oe{}uvres ne renvoient pas \`{a} des futurs envisag\'{e}s, mais plut\^{o}t \`{a} des futurs non advenus pour l'un, et \`{a} des pass\'{e}s alternatifs, revisit\'{e}s, pour l'autre. On pourrait avancer qu'ils d\'{e}crivent des futurs envisag\'{e}s du point de vue d'un moment historique pass\'{e}. Mais le regard ult\'{e}rieur port\'{e} par l'auteur sur ce moment historique introduit n\'{e}cessairement un biais, qui rend tr\`{e}s d\'{e}licate la manipulation de ces \oe{}uvres. Aussi, nous pr\'{e}f\'{e}rons les \'{e}carter de notre champ d'\'{e}tude, pour le moment. 





	De m\^{e}me, la question du \textbf{\emph{space opera}}, qui constitue une part cons\'{e}quente de la litt\'{e}rature de science-fiction, m\'{e}rite d'\^{e}tre pos\'{e}e. Il constitue un sous-genre complexe, parfois \`{a} la limite des proc\'{e}d\'{e}s de l'anticipation, mais d'une anticipation qui n'a pas vraiment de vis\'{e}e r\'{e}aliste. De plus, s'il d\'{e}crit ce qui pourrait se produire un jour, si les mondes qu'il invente proc\`{e}dent toujours d'un futur acceptable, il met bien souvent en sc\`{e}ne un ailleurs lointain, qui prive parfois le monde fictif d'un lien direct avec le monde du lecteur. Aussi, la question des futurs envisag\'{e}s invite un traitement plus rigoureux \`{a} travers le genre du \textbf{\emph{space opera}}, prenant en compte cette dimension sp\'{e}cifique. On choisira donc ici d'exclure temporairement le genre de notre corpus d'\'{e}tude. Soulignons n\'{e}anmoins que nous comptons, \`{a} terme, inclure cette litt\'{e}rature abondante \`{a} nos travaux. 





	Reste alors la litt\'{e}rature qui int\'{e}resse v\'{e}ritablement notre objet de recherche, \`{a} savoir celle qui proc\`{e}de v\'{e}ritablement d'une litt\'{e}rature d'anticipation \`{a} vis\'{e}e r\'{e}aliste. Que cette derni\`{e}re soit utopique ou dystopique, elle se caract\'{e}rise par son haut niveau de vraisemblance. On a d\'{e}j\`{a} \'{e}voqu\'{e} les principaux genres auxquels on pourra se r\'{e}f\'{e}rer en la mati\`{e}re : la \textbf{\emph{hard science-fiction}}, l'\textbf{\emph{anticipation politique}}, la\textbf{\emph{ speculative fiction}} et le\textbf{\emph{ cyberpunk}}. Comme on l'a d\'{e}j\`{a} expliqu\'{e}, ces genres ont en commun un caract\`{e}re anticipatoire, \`{a} la limite du prospectif. Comme l'exprime on ne peut mieux Ariel Kyrou {``}La science-fiction, lorsqu'elle ne se contente pas de divertir, fabrique des souvenirs du futur. Et ces souvenirs, \`{a} l'instar de dispositifs d'art contemporain, deviennent autant de tableaux d'hypoth\`{e}ses sur l'\`{a} venir''\footnote{\textsuperscript{\newpage
}\textsuperscript{	KYROU\ Ariel.\ op.cit.}} . La part de la litt\'{e}rature de science-fiction que nous retenons ici, celle qui {``}ne se contente pas de divertir'', constitue donc pr\'{e}cis\'{e}ment l'objet de nos recherches.








	\textbf{Le choix de la francophonie}





	Il faut ici \'{e}voquer le choix d\'{e}cisif de constituer un corpus compos\'{e} exclusivement de litt\'{e}rature francophone. De m\^{e}me que toute la discussion historique et th\'{e}matique que nous avons pr\'{e}c\'{e}demment men\'{e}e, ce choix intervient apr\`{e}s une premi\`{e}re ann\'{e}e de travaux de recherche et d'exp\'{e}rimentations. Dans le cadre des travaux de M1, en effet, nous avions constitu\'{e} un prototype exp\'{e}rimental compos\'{e} de deux oeuvres seulement - \emph{1984}, de G. Orwell, et \emph{Neuromancer}, de W. Gibson -- afin de tester nos m\'{e}thodes. Nous avions alors choisi de travailler sur les traductions fran\c{c}aises de ces deux ouvrages \'{e}crits en anglais. Or, les conclusions de ces premi\`{e}res phases exp\'{e}rimentales avaient soulev\'{e} un probl\`{e}me \'{e}pist\'{e}mologique central autour de la question de la traduction. Dans le cadre d'une analyse s\'{e}mantique, en effet, il est pr\'{e}f\'{e}rable de travailler sur des textes en version originale, d'autant plus que les traductions sont parfois bien ult\'{e}rieures \`{a} la composition des oeuvres, induisant un biais dans l'analyse historique de notre corpus. 





	Nous avons donc choisi de travailler sur des versions orginales. Mais il restait \`{a} choisir entre litt\'{e}rature anglophone et francophone. Nous avons pris la d\'{e}cision de travailler exclusivement sur de la litt\'{e}rature d'anticipation francophone, pour deux raisons principales. D'abord parce qu'il nous a sembl\'{e} que, dans le cadre d'une analyse s\'{e}mantique, il \'{e}tait important d'avoir une sensibilit\'{e} particuli\`{e}re aux mots qui constitueraient la base de nos r\'{e}sultats. De fait, nous nous sentions plus \`{a} l'aise pour appr\'{e}cier justement leurs diff\'{e}rentes connotations, leur texture, leur port\'{e}e dans notre langue maternelle, le fran\c{c}ais. Ensuite, car la litt\'{e}rature d'anticipation, et d'une mani\`{e}re plus g\'{e}n\'{e}rale, la science-fiction francophone, est un objet nettement moins \'{e}tudi\'{e} dans le monde universitaire, et qu'il me paraissait plus int\'{e}ressant de travailler sur des mat\'{e}riaux encore peu abord\'{e}s. 








	\textbf{Un vaste corpus de litt\'{e}rature contemporaine}





	Les retours constructifs des premiers travaux de recherches effectu\'{e}s dans le cadre du M1 avaient \'{e}galement attir\'{e} notre attention sur deux points essentiels pour la constitution de notre corpus. Tout d'abord, les r\'{e}sultats que nous avions obtenu sur le corpus exp\'{e}rimental -- compos\'{e}, rappelons-le, de seulement 2 oeuvres -- ne permettaient ni de monter en g\'{e}n\'{e}ralit\'{e}, ni d'appr\'{e}cier aussi finement les textes qu'avec une m\'{e}thode d'analyse litt\'{e}raire plus classique. Aussi, ils indiquaient que la m\'{e}thode n'aurait de l'int\'{e}r\^{e}t qu'avec la mont\'{e}e en \'{e}chelle de notre corpus, puisqu'elle permettrait \`{a} la fois de monter en g\'{e}n\'{e}ralit\'{e} dans l'exploration th\'{e}matique de nos oeuvres, et de surpasser la capacit\'{e} de traitement de m\'{e}thodes d'analyse traditionnelles. Il nous paraissait donc indispensable, pour poursuivre nos recherches, de constituer un corpus de grande taille, de l'ordre d'une centaine d'oeuvres au moins.





	Ensuite, ces retours d'exp\'{e}riences sugg\'{e}raient de traiter, au moins dans un premier temps, des oeuvres concentr\'{e}es sur une courte p\'{e}riode historique, plut\^{o}t qu'\'{e}tal\'{e}es sur plusieurs si\`{e}cles, comme nous l'avions envisag\'{e} initialement. Il nous fallait donc \'{e}galement choisir la p\'{e}riode sur laquelle nous souhaitons porter notre \'{e}tude des futurs envisag\'{e}s. D\`{e}s les pr\'{e}mices de ces travaux de recherche, nous souhaitions travailler sur de la litt\'{e}rature tr\`{e}s contemporaine : le courant \textbf{cyberpunk}, qu'on a d\'{e}j\`{a} pr\'{e}sent\'{e} auparavant, int\'{e}ressait particuli\`{e}rement notre \'{e}tude, en ce qu'il traite de th\'{e}matiques et de technologies tr\`{e}s actuelles -- le cyberespace, notamment. Cependant, il ne s'agit pas d'un courant tr\`{e}s repr\'{e}sent\'{e} dans la litt\'{e}rature de science-fiction francophone, et le seul choix de ce sous-genre n'aurait pas suffi \`{a} constituer un corpus de taille suffisante. En outre, en France comme dans le monde anglophone, il s'agit encore d'une litt\'{e}rature tr\`{e}s sp\'{e}cialis\'{e} et assez peu grand public, et la question de l'accessibilit\'{e} des oeuvres aurait sans doute pos\'{e} quelques probl\`{e}mes lors de la r\'{e}cup\'{e}ration des textes.





	Ainsi, nous avons choisi d'\'{e}largir notre s\'{e}lection en choisissant des oeuvres issues d'autres sous-genres, tant que celles-ci proc\'{e}daient de l'anticipation. Nous avons \'{e}galement allong\'{e} la p\'{e}riode \'{e}tudi\'{e}e : alors que la litt\'{e}rature cyberpunk appara\^{i}t \`{a} partir du milieu des ann\'{e}es 80, nous sommes all\'{e}s puiser des oeuvres du d\'{e}but des ann\'{e}es 50 jusqu' \`{a} la fin des ann\'{e}es 80 pour constituer ce corpus. On notera tout de m\^{e}me que la litt\'{e}rature \'{e}tudi\'{e}e reste tr\`{e}s contemporaine, et respecte relativement notre volont\'{e} de se concentrer sur une p\'{e}riode assez limit\'{e}e. En outre, cette p\'{e}riode se caract\'{e}rise en France par de nombreuses mutations technologiques (\'{e}l\'{e}ctrom\'{e}nager, conqu\^{e}te spatiale, t\'{e}l\'{e}phonie mobile, d\'{e}buts d'internet\ldots{}) et socio-\'{e}conomiques (p\'{e}riodes des trentes glorieuses puis chocs p\'{e}troliers et tournant n\'{e}o-lib\'{e}ral, \'{e}mancipation des femmes, questionnements environementaux), nourissant les diff\'{e}rents imaginaires associ\'{e}s aux projections futuristes.








	\textbf{Constituer un corpus d'analyse}





\textbf{	}A partir des \'{e}l\'{e}ments de r\'{e}flexions pr\'{e}sent\'{e}s, j'ai finalement choisi deux collections embl\'{e}matiques de la litt\'{e}rature de science fiction fran\c{c}aise : {`}Anticipations', de la maison d'\'{e}dition Fleuve noir, cr\'{e}\'{e}e en 1951, et {`}Pr\'{e}sence du futur', des \'{e}ditions Deno\"{e}l, lanc\'{e}e en 1954. Revenons d'abord sur les deux collections et leur politique \'{e}ditoriale, afin de comprendre l'int\'{e}r\^{e}t d'y puiser des ouvrages pour constituer notre corpus. Pour ce qui est de la collection Deno\"{e}l, il faut savoir qu'elle a d'abord \'{e}t\'{e} dirig\'{e}e par Robert Kanters, traduisant et popularisant de nombreux auteurs anglo-saxons de science-fiction, comme Brian Aldiss ou Philipp K. Dick, donc certaines oeuvres majeures, comme le \emph{cycle de Fondation}. Mais la collection publie aussi de jeunes auteurs fran\c{c}ais comme Jean-Pierre Andrevon, Serge Brussolo, Emmanuel Jouanne, Pierre Pelot ou Antoine Volodine, notamment sous l'impulsion d'Elisabeth Gille qui en prend la direction en 1976. L'accent est port\'{e}e sur la science-fiction litt\'{e}raire, mais la collection \'{e}dite aussi de la \textbf{hard science-fiction} et du \textbf{cyberpunk}. De mani\`{e}re analogue, la collection anticipations se fait surtout, \`{a} ses d\'{e}buts, la traductrice et la promotrice de la SF anglo-saxonne. Mais elle publie \'{e}galement de jeunes auteurs auteurs francophones, comme Jean-Louis et Doris Le May, ou Louis Thirion. Remarquons que la maison d'\'{e}dition Fleuve Noir adopte \`{a} cette \'{e}poque une politique encourageant la fid\'{e}lit\'{e} de ses auteurs -- ces derniers n'\'{e}taient publi\'{e}s qu'\`{a} partir du deuxi\`{e}me manuscrit s\'{e}lectionn\'{e} et devaient s'engager \`{a} ne pas publier ailleurs sous le m\^{e}me nom pour 5 ans, d'où les nombreuses oeuvres publi\'{e}s sous des pseudonymes, comme {``}Gilles d'Argyre'' pour G\'{e}rard Klein.





{\small 	}J'ai donc proc\'{e}d\'{e} en listant toutes les oeuvres qui avaient \'{e}t\'{e} \'{e}dit\'{e}es dans ces deux collections entre leur cr\'{e}ation et la fin des ann\'{e}es 80, en \'{e}cartant les oeuvres qui n'\'{e}taient pas des versions originales, puis en s\'{e}lectionnant celles qui, apr\`{e}s lecture de leur r\'{e}sum\'{e}, proc\'{e}daient de la litt\'{e}rature d'anticipation. Evidemment, cette s\'{e}lection introduit un biais, puisqu'elle semble quelque peu arbitraire : les oeuvres ne sont pas toujours pr\'{e}sent\'{e}es comme des oeuvres d'anticipation \`{a} proprement parler, et m\^{e}me si on a d\'{e}fini plus clairement les sous-genres qui relevaient de l'anticipation, ces sous-genres ne sont pas explicitement mentionn\'{e}s. Cependant, je me suis astreint au suivi de quelques crit\`{e}res pour tenter d'objectiver un tant soit peu cette s\'{e}lection : de fait, j'ai \'{e}cart\'{e} toutes les oeuvres qui s'apparentaient au genre du space opera, et qui faisaient le r\'{e}cit d'aventures intergalactiques et de batailles interstellaires : elles n'int\'{e}ressent pas cette analyse des imaginaires projectionnistes ; j'ai \'{e}galement \'{e}cart\'{e} les oeuvres qui relevaient du paradoxe temporel, c'est-\`{a}-dire qui racontaient des voyages dans le temps, et qui l\`{a} encore n'ont pas vraiment \`{a} voir avec les futurs envisag\'{e}s ; enfin, j'ai \'{e}cart\'{e} les oeuvres dans lesquels l'intrigue se d\'{e}roulait dans des mondes extraterrestres tr\`{e}s diff\'{e}rents de la Terre, et qui n'avaient aucun lien avec l'Homme. 





	Nous souhaiterions ici faire remarquer que sur une liste d'oeuvres de l'ordre de 2000 oeuvres -- tous les ouvrages publi\'{e}s par ces collections, seule une petite partie des oeuvres -- environ 5\% - remplissaient les crit\`{e}res mentionn\'{e}s. Nous avons donc tout bonnement s\'{e}lectionn\'{e} toutes ces oeuvres pour constituer notre corpus final. Ce corpus comptait \'{e}galement des recueils de nouvelles, et nous avons choisi d'en extraire les nouvelles qui nous paraissaient ad\'{e}quates. Finalement, on comptait alors 112 oeuvres et nouvelles dans notre corpus. Mais certains d\'{e}tails techniques nous ont conduit \`{a} affiner encore l\'{e}g\`{e}rement notre corpus. En effet, nous avons mentionn\'{e} notre volont\'{e} d'avoir recours \`{a} une m\'{e}thode informatique pour appr\'{e}hender notre corpus, le\emph{ topic modeling},\emph{ }sur laquelle nous reviendrons. Or, il faut un minimum de l'ordre de 1000 mots par document pour que cette m\'{e}thode soit efficace, et nous avons d\^{u} retirer une vingtaine de nouvelles qui n'\'{e}taient pas assez longues et qui auraient perturb\'{e} le fonctionnement de notre mod\`{e}le.








	Pr\'{e}sentation du corpus final





	Apr\`{e}s ce premier traitement, je disposais finalement d'un corpus constitu\'{e} de 94 oeuvres litt\'{e}raires en versions originales fran\c{c}aises, toutes issues des deux collections pr\'{e}sent\'{e}es pr\'{e}c\'{e}demment. Pour la liste exhaustive des diff\'{e}rents ouvrages qui composent notre corpus, nous invitons nos lecteurs \`{a} consulter rapidement la bibliographie du corpus\footnote{	Voir Bibliographie du Corpus} . Il convient ici de donner un aper\c{c}u des diff\'{e}rents ouvrages qui composent notre corpus.





	Il faut d'abord souligner la grande vari\'{e}t\'{e} d'auteurs dont est issue ce corpus : on en compte 39, dont on ne fera pas le r\'{e}pertoire exhaustif ici. N\'{e}anmoins, il est int\'{e}ressant de mentionner cette diversit\'{e}, qui n'\'{e}tait pas forc\'{e}ment un objectif dans la constitution du corpus, mais qui se r\'{e}v\`{e}le une force lorsqu'on souhaite explorer les futurs envisag\'{e}s. De fait, en s'int\'{e}ressant \`{a} une plus grande vari\'{e}t\'{e} de plumes, on acc\`{e}de \`{a} davantage de points de vue diff\'{e}rents, qui sont autant d'images et de questionnements d'int\'{e}r\^{e}t pour notre analyse. Remarquons \'{e}galement que pour la collection {`}Anticipations', de nombreux ouvrages sont publi\'{e}s sous des alias pour les raisons qu'on a \'{e}voqu\'{e}es plus haut. Ces alias correspondent au pseudonyme d'un auteur ou, parfois, d'un collectif d'auteurs : on citera notamment {``}Christopher Stork'', l'alias choisi par un collectif d'auteurs belges pour publier certaines oeuvres, \`{a} l'instar de \emph{L'ordre \'{e}tabli }(1979), ou encore Fran\c{c}ois Richard Bessi\`{e}re, l'alias de Fran\c{c}ois Richard et Henri Bessi\`{e}re. Aussi, il faut r\'{e}insister sur la diversit\'{e} autorale de notre corpus, non seulement sur le plan du nombre d'auteurs, mais \'{e}galement sur celui des modalit\'{e}s de composition des romans, parfois \'{e}crits collectivement. 





	On compte aussi une grande diversit\'{e} dans la forme et la longueur des textes. On dispose ainsi de 38 nouvelles, dont les plus courtes font \`{a} peu pr\`{e}s un millier de mots, et de 56 romans, dont le plus long fait environ 79 000 mots. Sous cette forme, on a donc un \'{e}cart 1 \`{a} 80 entre le texte le plus court le plus long, ce qui est encore une fois int\'{e}ressant pour la diversit\'{e} de notre corpus, mais soul\`{e}ve, il faut le dire, certains probl\`{e}mes m\'{e}thodologiques pour l'impl\'{e}mentation du \emph{topic modeling}. Nous reviendrons sur les probl\`{e}mes d'harmonisation de la longueur des textes au moment d'expliquer plus en d\'{e}tail notre m\'{e}thode. Notons toutefois qu'une fois le pr\'{e}traitement de nos donn\'{e}es textuelles r\'{e}alis\'{e}, cet \'{e}cart de longueur est moins significatif, et qu'en ayant recours \`{a} des techniques d'\'{e}chantillonnage, nous pouvons r\'{e}soudre la question de la longueur des textes de mani\`{e}re relativement efficace.





\textbf{	}Donnons enfin un bref aper\c{c}u de la composition litt\'{e}raire et th\'{e}matique de ce corpus, en revenant plus en d\'{e}tail sur certaines des oeuvres majeures. L\`{a} encore, on remarque qu'il est compos\'{e} aussi bien d'oeuvres majeures, qui a posteriori, apparaissent comme de v\'{e}ritables monuments de la SF francophone, que d'oeuvres nettement moins connues du grand public. Se c\^{o}toient ainsi de petites nouvelles assez peu remarqu\'{e}es, enfouies au sein de larges recueil de nouvelles, comme \emph{Univers-Code }de Jean-Claude Dunyach ou \emph{Soleil de soufre} de Serge Brussolo, aussi bien que des oeuvres majeures, parmi lesquelles on pourra notamment citer \emph{Niourk}, de St\'{e}phane Wul. Du point de vue th\'{e}matique, on a aussi une grande profusion de sujets abord\'{e}s et de questionnements. Une partie importante de notre corpus interroge des ph\'{e}nom\`{e}nes apocalyptiques, li\'{e}s \`{a} l'effondrement des soci\'{e}t\'{e}s humaines. On peut \'{e}voquer, de mani\`{e}re non exhaustive, certains th\`{e}mes r\'{e}currents qui concernent comme les pand\'{e}mies, avec par exemple \emph{L'\`{e}re des biocybs} de Jimmy Guieu (1960) et \emph{Pand\'{e}moniopolis} de Gabriel Jan (1975), le climat, avec \emph{La sortie est au fond de l'espace} de Jacques Sternberg (1956) ou \emph{La Mort de la vie }de Jimmy Guieu (1957) ou encore les guerres robotiques, avec \emph{Territoire robot }de Jean-Gaston Vandel (1954). D'autres ouvrages abordent des th\'{e}matiques relatives \`{a} l'avanc\'{e}e scientifique et l'expansion de la civilisation humaine : reviennent notamment les th\`{e}mes de la conqu\^{e}te spatiale, de la ville du futur ou des avanc\'{e}es scientifiques, au sujet desquels on trouvera diverses oeuvres dans les recueils de nouvelles\emph{ Vue en coupe d'une ville malade} de Serge Brussolo (1980) et \emph{Superfuturs}, de Philippe Curval (1986). Si d'autres th\`{e}mes sont \'{e}galement abord\'{e}s, l'ensemble ainsi d\'{e}crit r\'{e}sume assez bien les motifs r\'{e}currents et questionnements abord\'{e}s au sein des oeuvres qui composent notre corpus.





	En fin de compte, on dispose d'un corpus qui rend assez bien compte de la diversit\'{e} de la sc\`{e}ne SF fran\c{c}aise de notre p\'{e}riode, avec diff\'{e}rents types d'ouvrages, sur le plan de la forme comme sur celui du fond. Le corpus prend en compte les \'{e}l\'{e}ments de r\'{e}flexion qu'on a pu formuler en vue de sa constitution, et il est en ce sens un support d'investigation adapt\'{e} \`{a} l'\'{e}tude des futurs envisag\'{e}s. Notons toutefois qu'on aurait aim\'{e} pouvoir agrandir encore la taille de notre corpus, mais que seule part assez restreinte de la litt\'{e}rature d'anticipation francophone, du moins pour les collections qu'on a utilis\'{e}es, semble correspondre aux crit\`{e}res de s\'{e}lection \'{e}tablis. N\'{e}anmoins, le corpus ainsi compos\'{e} constitue d\'{e}j\`{a} un mat\'{e}riau solide pour explorer les imaginaires projectionnistes. Il nous faut donc maintenant expliquer la m\'{e}thode informatique que nous avons choisie pour explorer s\'{e}mantiquement ce corpus, et sonder les futurs envisag\'{e}s.





\newpage






\newpage



\textbf{II) }\textbf{\uline{La construction d'une m\'{e}thode computationnelle pour explorer notre corpus}}








	En 2017, un article de Guillaume Carbou intitul\'{e} {``}Analyser les textes \`{a} l'\`{e}re des humanit\'{e}s num\'{e}riques. Quelques questions pour l'analyse statistique des donn\'{e}es textuelles'' paraissait dans \emph{Les Cahiers du num\'{e}rique}\footnote{\textsuperscript{\newpage
}\textsuperscript{	}\textsuperscript{{\tiny CARBOU\ Guillaume.\ «\ Analyser\ les\ textes\ \`{a}\ l'\`{e}re\ des\ humanit\'{e}s\ num\'{e}riques.\ Quelques\ questions\ pour\ l'analyse\ statistique\ des\ donn\'{e}es\ textuelles\ »,\ Les\ Cahiers\ du\ num\'{e}rique,\ 2017/3\ (Vol.\ 13),\ p.\ 91-114.\ URL:\ }}\href{https://www.cairn.info/revue-les-cahiers-du-numerique-2017-3-page-91.htm}{\textsuperscript{\textcolor[rgb]{0.000,0.082,0.361}{{\tiny \uline{https://www.cairn.info/revue-les-cahiers-du-numerique-2017-3-page-91.htm}}}}}} \emph{. }Faisant \'{e}tat de l'int\'{e}r\^{e}t des outils statistiques et des m\'{e}thodes d'analyse informatiques, ce dernier pr\^{o}nait {``}la reconnaissance de la dimension herm\'{e}neutique des humanit\'{e}s num\'{e}riques''. Mais l'article de Guillaume Carbou s'attachait surtout \`{a} poser les questions m\'{e}thodologiques que soul\`{e}vent une telle d\'{e}marche, montrant que le traitement automatique de grands volumes de donn\'{e}es n\'{e}cessitait un effort \'{e}pist\'{e}mologique de r\'{e}flexion sur l'objet \'{e}tudi\'{e}, et la mani\`{e}re de l'aborder sur le plan de l'analyse. Actant l'apport majeur des humanit\'{e}s num\'{e}riques, nous souhaitons employer ces m\'{e}thodes afin d'appr\'{e}hender les futurs envisag\'{e}s \`{a} travers la litt\'{e}rature d'anticipation. Nous allons donc par aborder la m\'{e}thode que nous proposons d'utiliser, le \emph{topic modeling}, d'un point de vue scientifique et technique. Puis, nous pr\'{e}senterons le module que nous avons d\'{e}velopp\'{e}, avant de faire le compte rendu de son utilisation sur nos donn\'{e}es, en \'{e}voquant les diff\'{e}rentes consid\'{e}rations \'{e}pist\'{e}mologiques auxquelles nos donn\'{e}es ont pu nous mener.








1. Le \emph{topic modeling}








\textbf{	Qu'est ce que le }\textbf{\emph{topic modeling}}\textbf{ ?}





\textbf{	}L'appellation \emph{topic modeling }regroupe un ensemble de mod\`{e}les probabilistes g\'{e}n\'{e}ratifs, qui fournissent des m\'{e}thodes pour analyser la fr\'{e}quence d'occurrence de mots et de th\`{e}mes dans un corpus de documents. On note ici que ces m\'{e}thodes se r\'{e}f\`{e}rent uniquement \`{a} la fr\'{e}quence d'occurrence des mots, leur position dans le document \'{e}tant parfaitement n\'{e}glig\'{e}e. On parlera de mod\`{e}les \emph{bag of words} : on consid\`{e}re chaque document comme un {``}sac de mots'', c'est-\`{a}-dire un ensemble faisant abstraction de l'ordre d'apparition des mots. Le \emph{topic modeling} se perfectionne et s'affirme depuis une vingtaine d'ann\'{e}es comme l'approche la plus pertinente pour le traitement th\'{e}matique de donn\'{e}es textuelles. Pour mieux cerner ce concept complexe, et pour avoir un meilleur aper\c{c}u des m\'{e}thodes qui sont \`{a} notre disposition, revenons bri\`{e}vement sur l'histoire des principaux mod\`{e}les et de leur apport.





Dans les ann\'{e}es 1990, avec l'arriv\'{e}e d'internet, la question de la classification de grandes masses de donn\'{e}es s'impose avec force. De premiers mod\`{e}les de traitement statistique du lexique voient le jour : on pense notamment \`{a} la Latent Semantic Analysis (LSA) en 1990\footnote{\textsuperscript{\newpage
}\textsuperscript{	DEERWESTER,\ Scott,\ DUMAIS,\ Susan,\ FURNAS,\ George,\ LANDAUER,\ Thomas,\ HARSHMAN,\ Richard,\ {``}Indexing\ by\ latent\ semantic\ analysis'',\ Journal\ of\ the\ american\ society\ for\ information\ science,\ 1990}} , et aux m\'{e}thodes qui se fondent sur le mod\`{e}le unigramme\footnote{\textsuperscript{\newpage
}\textsuperscript{	NIGAM,\ K.,\ MCCALUM,\ A.K.,\ THRUN,\ S.\ et\ al.\ {``}Text\ Classification\ from\ Labeled\ and\ Unlabeled\ Documents\ using\ EM'',\ Machine\ Learning\ 39,\ 103--134\ (2000).\ https://doi.org/10.1023/A:1007692713085}} . Ces derniers mod\`{e}les sont, en effet, limit\'{e}s en ce qu'ils attribuent \`{a} chaque document un th\`{e}me unique, et se r\'{e}v\`{e}lent incapables de traiter le probl\`{e}me de l'appartenance simultan\'{e}e de certains mots \`{a} plusieurs th\`{e}mes, et celui des documents comportant plusieurs th\`{e}mes. C'est pour tenter d'apporter une r\'{e}ponse \`{a} ces probl\`{e}mes que de v\'{e}ritables mod\`{e}les de traitement th\'{e}matiques sont \'{e}labor\'{e}s. Un proto-mod\`{e}le de \emph{topic modeling} est 


propos\'{e} par Hofman en 1999, lorsqu'il d\'{e}veloppe la LSA probabiliste (PLSA)\footnote{\textsuperscript{\newpage
}\textsuperscript{	HOFFMANN,\ Thomas,\ {``}Probabilistic\ latent\ semantic\ analysis'',\ UAI'99\ Proceedings\ of\ the\ Fifteenth\ conference\ on\ Uncertainty\ in\ artificial\ intelligence,\ July\ 1999\ Pages\ 289--296}} . Mais la Latent Dirichlet Allocation (LDA) qui, en 2003\footnote{\textsuperscript{\newpage
}\textsuperscript{	BLEI,\ David\ M.,\ NG,\ Andrew\ Y.,\ JORDAN\ Michael\ I.,\ {``}Latent\ Dirichlet\ Allocation'',\ Journal\ of\ Machine\ Learning\ Research\ 3\ (2003)\ 993-1022}} , s'affirme v\'{e}ritablement dans le monde universitaire comme un outil de \emph{topic modeling }pertinent et fonctionnel. Cette m\'{e}thode permet \`{a} la fois de r\'{e}v\'{e}ler des structures th\'{e}matiques cach\'{e}s au sein de grands corpus, et d'analyser les fr\'{e}quences d'occurrence de ces th\`{e}mes au sein de chaque document. Diff\'{e}rentes extensions du mod\`{e}le LDA seront ensuite pr\'{e}sent\'{e}es, et on citera notamment le \emph{correlated topics model} (CTM) en 2007\footnote{\textsuperscript{\newpage
}\textsuperscript{	BLEI,\ David\ M.,\ LAFFERTY,\ John\ D.,\ {``}A\ CORRELATED\ }\emph{\textsuperscript{TOPIC}}\textsuperscript{\ MODEL\ OF\ SCIENCE'',\ The\ Annals\ of\ Applied\ Statistics\ 2007,\ Vol.\ 1,\ No.\ 1,\ 17--35}} qui autorise les corr\'{e}lations entre diff\'{e}rents th\`{e}mes. La LDA, et les diff\'{e}rentes m\'{e}thodes qui se fondent sur ses principes, sont aujourd'hui le meilleur moyen d'analyser th\'{e}matiquement de vastes ensembles litt\'{e}raires; en ce sens, c'est celles que nous retiendrons dans le cadre de nos travaux, et il convient ici de rentrer davantage dans le d\'{e}tail de leur fonctionnement.








\textbf{	Utiliser la LDA pour \'{e}tudier les futurs envisag\'{e}s}





\textbf{	}La LDA, on l'a dit, est une des m\'{e}thodes les plus utilis\'{e}es lorsqu'on cherche \`{a} faire du \emph{topic modeling}. Elle offre des solutions int\'{e}ressantes en mati\`{e}re d'organisation de documents par th\`{e}mes et d'analyse du texte. Sans rentrer dans les math\'{e}matiques complexes qui se cachent derri\`{e}re ces m\'{e}thodes, t\^{a}chons ici de comprendre son fonctionnement. La LDA est fond\'{e}e sur deux principes majeurs. D'abord, chaque document est un m\'{e}lange de th\`{e}mes (\emph{topics}), pr\'{e}sents en diff\'{e}rentes proportions. Ensuite, chaque \emph{topic} est un m\'{e}lange de mots, eux-aussi, pr\'{e}sents en diff\'{e}rentes proportions dans chaque document. De mani\`{e}re int\'{e}ressante, on remarque que les mots peuvent appartenir simultan\'{e}ment \`{a} plusieurs th\`{e}mes. La LDA est une m\'{e}thode math\'{e}matique qui permet de trouver le m\'{e}lange de mots qui correspond \`{a} chaque th\`{e}me, et le m\'{e}lange de th\`{e}mes qui d\'{e}crit chaque document.





	Sans rentrer dans des d\'{e}tails trop techniques, approfondissons tout de m\^{e}me l'explication math\'{e}matique de ce mod\`{e}le, avant de donner un exemple plus parlant. Nous disposons d'un ensemble de documents textuels, notre corpus d'oeuvres d'anticipation, chaque document \'{e}tant d\'{e}fini par l'ensemble de mots qui le composent. La LDA suppose que chaque document correspond au m\'{e}lange d'un petit nombre de \emph{topics}, et que la g\'{e}n\'{e}ration de chaque occurrence d'un mot est attribuable \`{a} l'un des th\`{e}mes du document. Lorsqu'on entra\^{i}ne un mod\`{e}le de LDA, on commence par une phase d'initialisation, en attribuant un \emph{topic} \`{a} chaque mot de chaque document, selon une distribution de Dirichlet sur un ensemble de N \emph{topics}. Notons donc qu'il faut choisir un nombre de th\`{e}mes lors de l'impl\'{e}mentation de la LDA. Ce premier \emph{topic model} est assez peu coh\'{e}rent, puisqu'il est g\'{e}n\'{e}r\'{e} al\'{e}atoirement. Intervient alors la phase d'apprentissage au cours de laquelle on cherche \`{a} affiner notre mod\`{e}le, Pour cela, dans chaque document, on prend chaque mot et on met \`{a} jour le th\`{e}me auquel il est li\'{e}, en lui attribuant le \emph{topic }qui a la plus forte probabilit\'{e} de g\'{e}n\'{e}rer le mot donn\'{e} dans ce document. On r\'{e}p\`{e}te ces \'{e}tapes un grand nombre de fois : c'est le nombre d'it\'{e}rations, qu'il faut \'{e}galement d\'{e}finir lors de l'impl\'{e}mentation de la LDA. A l'issue de cette phase d'apprentissage, un obtient un mod\`{e}le de sujets affin\'{e}, qui comporte la probabilit\'{e} d'occurrence des \emph{topics }pr\'{e}sents dans chaque document -- on compte le nombre de repr\'{e}sentation de chaque th\`{e}me dans chaque document, et les probabilit\'{e} d'occurrence des mots associ\'{e}s \`{a} chaque th\`{e}me -- on compte les mots associ\'{e}s \`{a} chaque th\`{e}me dans l'ensemble du corpus.





Pour mieux comprendre ce proc\'{e}d\'{e}, prenons l'exemple d'un corpus d'articles de journaux, dans lequel on aurait un premier article relatif \`{a} la politique et un second article relatif aux sciences dures. Le premier article pourrait ainsi contenir des mots tels que {``}r\'{e}f\'{e}rendum'', {``}gouvernement'', {``}parlement'', {``}d\'{e}bat''. Le second article, pourrait quant \`{a} lui contenir des mots tels que {``}\'{e}quations'', {``}physique'', et \'{e}ventuellement {``}d\'{e}bat''. En appliquant nos m\'{e}thodes \`{a} ces deux documents, on pourrait donc voir \'{e}merger deux \emph{topics}, l'un relatif aux sciences, l'autre \`{a} la politique et tous deux contenant notamment les mots leur \'{e}tant respectivement associ\'{e}s. Maintenant, imaginons qu'un troisi\`{e}me article relate un d\'{e}bat parlementaire, au cours duquel auraient \'{e}t\'{e} abord\'{e}es, entre autres, des probl\'{e}matiques scientifiques. Notre m\'{e}thode permettrait ici d'attribuer une proportion d'utilisation de chaque \emph{topic} \`{a} ce troisi\`{e}me article : ainsi, si l'article invoque principalement un lexique politique, et de mani\`{e}re plus br\`{e}ve, un lexique scientifique, on pourrait grossi\`{e}rement obtenir quelque chose comme {``}l'article 3 rel\`{e}ve \`{a} 85\% du \emph{topic} 1, et \`{a} 15\% du \emph{topic} 2'', le \emph{topic} 1 et le \emph{topic} 2 pouvant, par le travail d'interpr\'{e}tation du chercheur, \^{e}tre caract\'{e}ris\'{e}s : le \emph{topic} 1 renvoie \`{a} la politique, le \emph{topic} 2 renvoie aux sciences.





Au regard de cet exemple, on imagine ais\'{e}ment l'int\'{e}r\^{e}t de la m\'{e}thode lorsqu'on souhaite \'{e}tudier les futurs envisag\'{e}s \`{a} travers la litt\'{e}rature SF d'anticipation. La double approche nous permet d'envisager deux niveaux d'analyse compl\'{e}mentaires. D'abord au niveau du corpus, en rep\'{e}rant des \emph{topics} r\'{e}currents au sein d'un corpus constitu\'{e} de science fiction, nous pouvons faire \'{e}merger des th\`{e}mes r\'{e}currents, correspondant \`{a} autant de points de crispation, de consid\'{e}rations qui reviennent souvent lorsque les auteurs du corpus se projettent dans le futur. Cette analyse, men\'{e}e sur diff\'{e}rents corpus, permettra aussi d'appr\'{e}cier la mani\`{e}re dont ces consid\'{e}rations \'{e}volue en fonction des \'{e}poques, des r\'{e}gions, ou encore des sous-genre de la SF. Ensuite au niveau des textes, elle permet de faire ressortir les mots qui sont utilis\'{e}s, dans chaque texte, pour \'{e}voquer chaque th\`{e}me. Cette analyse nous permettra, ici, d'analyser les mots choisis par les auteurs pour parler de chacune de ces consid\'{e}rations, l\`{a} encore, en consid\'{e}rant les diff\'{e}rents param\`{e}tres que sont le temps, l'espace, et le sous-genre.








	Le \emph{topic modeling} appliqu\'{e} \`{a} des corpus litt\'{e}raires





\textbf{\textcolor[rgb]{0.000,0.000,0.000}{	}}\textcolor[rgb]{0.000,0.000,0.000}{Qu'en est-il de l'utilisation du }\emph{\textcolor[rgb]{0.000,0.000,0.000}{topic modeling }}\textcolor[rgb]{0.000,0.000,0.000}{sur des corpus litt\'{e}raires dans le monde acad\'{e}mique? Existe-t-il des cas pr\'{e}c\'{e}dents d'utilisation sur de la litt\'{e}rature de Science-fiction, ou de mani\`{e}re plus g\'{e}n\'{e}rale, sur des corpus d'oeuvres litt\'{e}raires ? Le cas \'{e}ch\'{e}ant, ces recherches avaient-elles \'{e}galement pour but de sonder les repr\'{e}sentations ?}








	Apr\`{e}s avoir parcouru longuement parcouru les principales revues et actes de congr\`{e}s en Humanit\'{e}s Num\'{e}riques (DSH, DHQ, actes du congr\`{e}s DH), nous devons souligner l'absence d'exemples d'application de \emph{topic modeling}, ou m\^{e}me plus g\'{e}n\'{e}ralement de m\'{e}thodes de Distant Reading, sur des corpus de litt\'{e}rature d'anticipation. A d\'{e}faut de trouver des exemples sur des corpus de Science Fiction, nous pouvons ici \'{e}voquer quelques travaux majeurs dont les perspectives se rapprochent des n\^{o}tres, sur d'autres genres litt\'{e}raires.





	Pour ce qui est de l'\'{e}tude des repr\'{e}sentations \`{a} proprement parler, il n'existe que peu d'exemples d'utilisation des m\'{e}thodes de Distant Reading \`{a} ces fins. On \'{e}voquera ici des travaux assez m\'{e}connus, quoique particuli\`{e}rement int\'{e}ressants, sur la l'imaginaire litt\'{e}raire et les repr\'{e}sentations li\'{e}es \`{a} l'Ancien Monde dans la fiction du XIX\`{e}me si\`{e}cle\footnote{	http://digitalhumanities.org:8081/dhq/vol/10/2/000250/000250.html} . Dans ce compte-rendu all\'{e}g\'{e} d'un cours donn\'{e} \`{a} la premi\`{e}re r\'{e}union de la \emph{Digital Classicists Association}, Matthew L. Jokers fait l'expos\'{e} de recherches m\^{e}lant exploration th\'{e}matique, g\'{e}ographique et sentimentale des repr\'{e}sentations li\'{e}es \`{a} l'Ancien Monde, usant notamment du \emph{topic modeling} comme d'une m\'{e}thode d'exploration th\'{e}matique. On a ici un exemple int\'{e}ressant, bien qu'esseul\'{e}, d'utilisation du \emph{topic modeling} pour sonder les repr\'{e}sentations dans des corpus de litt\'{e}rature. En fait, pour trouver des travaux dont les perspectives se rapprochent des n\^{o}tres, il faut bien souvent d\'{e}caler un peu le regard, en s'\'{e}cartant de l'exploration th\'{e}matique des repr\'{e}sentations. Lorsqu'on s'int\'{e}resse aux exemples d'application du \emph{topic modeling} sur des corpus litt\'{e}raires, en effet, on trouve nettement plus de recherches li\'{e}es \`{a} des probl\`{e}mes de distinction et de classifcation th\'{e}matiques.





	En particulier, on pourra citer l'article remarqu\'{e} de Christop Sch\"{o}ch sur les genres du th\'{e}\^{a}tre classique\footnote{	http://www.digitalhumanities.org/dhq/vol/11/2/000291/000291.html\#politz2015} . Dans cette contribution le \emph{topic modeling} est utilis\'{e} pour analyser un corpus de pi\`{e}ces de th\'{e}\^{a}tre fran\c{c}aises de l'\^{a}ge classique et de la Renaissance. De mani\`{e}re remarquable, Sch\"{o}ch y m\`{e}ne une d\'{e}monstration de l'int\'{e}r\^{e}t du \emph{topic modeling} comme m\'{e}thode d'analyse quantitative des genres pour les textes de cette p\'{e}riode. L'\'{e}tude se fonde sur un corpus de 391 pi\`{e}ces publi\'{e}es entre 1610 et 1810, repr\'{e}sentatives des diff\'{e}rents genres traditionnellement attribu\'{e}s aux pi\`{e}ces de l'\'{e}poque : la com\'{e}die, la trag\'{e}die et la tragi-com\'{e}die. Les recherches de l'auteur parviennent justement \`{a} mettre en \'{e}vidence l'utilit\'{e} du \emph{topic modeling}, non seulement pour explorer s\'{e}mantiquement les th\'{e}matiques abord\'{e}es dans ces oeuvres, mais aussi pour distinguer th\'{e}matiquement les sous-genres de l'\'{e}poque \`{a} l'aide de crit\`{e}res quantitatifs. Fort de r\'{e}sultats probants, Christop Sch\"{o}ch parvient \`{a} op\'{e}rer une classification de ces sous-genres en distinguant diff\'{e}rents sch\'{e}mas th\'{e}matiques dominants pour chacun d'entre-eux. En comparant ses r\'{e}sultats \`{a} une classification conventionnelle des sous-genres de la p\'{e}riode, l'auteur confirme quantiativement et nuance des r\'{e}sultats qui avaient \'{e}t\'{e} \'{e}tablis par des m\'{e}thodes d'analyse plus classiques. 





	Plus r\'{e}cemment, on pourra citer un article paru en 2019 dans \emph{digital humanities quartertly} sur une \'{e}tude computationnelle du \emph{Quan Tang Shi}\footnote{	http://digitalhumanities.org:8081/dhq/vol/13/4/000434/000434.html} , anthologie monumental de la po\'{e}sie tang, produite tout au long du r\`{e}gne de la dynastie Qing (1644-1912) sur la Chine. Le corpus analys\'{e} est constitu\'{e} de plus de 50 000 po\`{e}mes et fragments de po\`{e}me. De fait, ce volume rend difficile une analyse des textes pour l'\'{e}chelle humaine, et l'article contribue \`{a} d\'{e}montrer l'int\'{e}r\^{e}t des m\'{e}thodes de distant reading, et en particulier du \emph{topic modeling}, pour cadrer et penser les questions traditionnelles pos\'{e}es par les \'{e}tudes d'histoire litt\'{e}raire sur le \emph{Quan Tang Shi}, et apporte une nouvelle perspective de r\'{e}ponse sur ce que signifie v\'{e}ritablement lire la po\'{e}sie tang.





	De cette br\`{e}ve revue de l'\'{e}tat de l'art en mati\`{e}re de \emph{topic modeling} appliqu\'{e} \`{a} des corpus litt\'{e}raires, on retiendra donc qu'il existe assez peu d'exemples qui se servent de ces m\'{e}thodes pour sonder les repr\'{e}sentations. Cependant, de nombreuses \'{e}tudes, dont on aura cit\'{e} ici qu'une infime part, montrent l'int\'{e}r\^{e}t bien fond\'{e} du \emph{topic modeling} pour des probl\'{e}matiques li\'{e}es \`{a} la classification des textes, et pour changer changer d'\'{e}chelle dans l'analyse de tr\`{e}s vastes corpus litt\'{e}raires.. D'ailleurs, puisqu'on a longuement \'{e}voqu\'{e} la diversit\'{e} de la SF et la difficult\'{e} \`{a} la classifier en sous-genres, il faut souligner qu'utiliser le \emph{topic modeling} \`{a} des fins de classification th\'{e}matique du genre serait une piste int\'{e}ressante \`{a} suivre. Mais il faut bien comprendre que nos travaux se distinguent clairement de cette perspective. Nous cherchons ici explorer th\'{e}matiquement notre corpus \`{a} des fins interpr\'{e}tatives, et non typologiques.




















































































































\textbf{2.} Construction d'un module





	Nous avions d\'{e}j\`{a} pu utiliser le \emph{topic modeling} sur un prototype exp\'{e}rimental dans le cadre des recherches men\'{e}es en M1. Les retours plut\^{o}t positifs sugg\'{e}raient de poursuivre dans cette voie, en exploitant n\'{e}anmoins davantage les possibilit\'{e}s de visualisation des r\'{e}sultats. Nous avons donc d\'{e}cid\'{e} de mettre au point un module pour faciliter la mise en oeuvre du \emph{topic modeling} et d\'{e}velopper des solutions de visualisation plus adapt\'{e}es \`{a} l'int\'{e}rpr\'{e}tation de nos r\'{e}sultats.  Aussi, la cr\'{e}ation de notre module devait ob\'{e}ir \`{a} un cahier des charges d\'{e}fini de mani\`{e}re claire et exigeante, que nous allons maintenant pr\'{e}senter.








	\textbf{Preprocessing des donn\'{e}es}


	Avant toute chose, l'impl\'{e}mentation de la LDA n\'{e}cessite donc un travail de nettoyage de notre corpus et de lemmatisation des documents. On a besoin d'une liste de tous les mots pour chaque texte, sous une forme lemmatis\'{e}e. Notre module comporte donc une composante d\'{e}di\'{e}e au preprocessing des donn\'{e}es, permettant d'abord de \emph{tokeniser }les donn\'{e}es, puis de les lemmatiser, et enfin, d'op\'{e}rer des op\'{e}rations de filtrage des mots pour retirer les \'{e}l\'{e}ments qui n'int\'{e}ressent pas notre analyse. 


	Pour traiter nos donn\'{e}es textuelles, il faut d'abord tokeniser notre corpus, c'est-\`{a}-dire convertir l'ensemble de nos documents textuelles en des listes de mots, en parsant les diff\'{e}rents \'{e}l\'{e}ments du texte. Cette op\'{e}ration est assez facile, mais certains probl\`{e}mes apparaissent rapidement lorsqu'on traite de gros volumes de textes \'{e}dit\'{e}s pour l'impression, notamment au niveau des tirets en fin de ligne et des mots compos\'{e}s. Mon module utilise le mod\`{e}le de Spacy, auquel j'ai adjoint une fonction utilisant les expressions reguli\`{e}res pour am\'{e}liorer le parsage des mots compos\'{e}s et traiter les probl\`{e}mes de tirets en fin de ligne. Apr\`{e}s avoir obtenu une liste de \emph{tokens, }il faut lemmatiser notre corpus.\emph{ }La lemmatisation est une op\'{e}ration qui consiste \`{a} rapporter chacun des mots du texte \`{a} une forme neutre, canonique. On rapporte les formes conjugu\'{e}es des verbes \`{a} leur radicaux où \`{a} une forme infinitive, les mots au pluriel sont mis au singulier, etc. Les lettres capitales sont \'{e}galement mises en lettres minuscules. De nombreuses biblioth\`{e}ques de fonctions proposent des m\'{e}thodes efficaces pour lemmatiser les textes en langue anglaise; en revanche, l'op\'{e}ration est plus d\'{e}licate et moins bien renseign\'{e}e pour les textes en fran\c{c}ais. L\`{a} encore, nous utilisons les mod\`{e}les propos\'{e}s par le module Spacy, et notamment le mod\`{e}le le plus lourd performant, {`}fr\_core\_news\_lg'. Nous avions d\'{e}j\`{a} utilis\'{e} un mod\`{e}le plus all\'{e}g\'{e} de spacy, {`}fr\_core\_news\_sm' dans le cadre de notre approche exp\'{e}rimentale en M1. Au cours des recherches de cette ann\'{e}e, nous avons pu comparer les performances des diff\'{e}rents mod\`{e}les de spacy, ainsi que d'un mod\`{e}le concoct\'{e} par des chercheurs de l'Ecole nationale des Chartes, pie-extended. Ce dernier semblait plus adapt\'{e} \`{a} la langue fran\c{c}aise mais, entra\^{i}n\'{e} sur un corpus francophone de l'\'{e}poque classique, il avait beaucoup de mal \`{a} traiter notre corpus de litt\'{e}rature d'anticipation tr\`{e}s contemporaine. En outre, les r\'{e}centes mises \`{a} jour du module Spacy ont permis des progr\`{e}s consid\'{e}rables des performances de lemmatisation de leurs mod\`{e}les, nous d\'{e}terminant finalement \`{a} utiliser ces derniers.





	A l'issue de la phase de lemmatisation, non obtenons donc une liste de lemmes pour chaque texte de notre corpus. Mais il subsiste de nombreux lemmes qui n'int\'{e}ressent pas notre analyse th\'{e}matique : aussi, il nous faut intervenir pour \'{e}carter de l'analyse ces \'{e}l\'{e}ments. Ils se divisent en deux cat\'{e}gories. D'une part, les Stopwords, ou mot-outils, c'est-\`{a}-dire les mots dont le r\^{o}le syntaxique est important, mais qui n'ont aucune importance s\'{e}mantique : pr\'{e}positions, conjonctions, pronoms, d\'{e}terminants, etc. On dispose d'une liste de \emph{stopwords} pour la langue fran\c{c}aise, dont on va se servir comme d'un filtre sur notre corpus lemmatis\'{e}. De m\^{e}me, on va exploiter les pr\'{e}dictions morphosyntaxiques (POS) du mod\`{e}le Spacy pour retirer tous les \'{e}l\'{e}ments qui ne nous int\'{e}ressent pas (pronoms, d\'{e}terminants, adverbes, etc). D'autre part, les mots de narration, qui l\`{a} encore ne sont pas porteur de beaucoup de sens d'un point de vue th\'{e}matique. Il s'agit ici de toutes les entit\'{e}s nomm\'{e}es propres \`{a} l'intrigue des ouvrages -- noms de personnages, de lieux, dates, etc -- ainsi que des verbes de narration, qui servent \`{a} d\'{e}crire l'action mais qui disent peu du fond th\'{e}matique de nos textes. De surcro\^{i}t, leur grande fr\'{e}quence d'occurrence peut biaiser notre mod\`{e}le de \emph{topic modeling}. Ici, on utilise les fonctions de reconnaissance d'entit\'{e}s nomm\'{e}es propos\'{e}es par le mod\`{e}le de Spacy pour retirer ces entit\'{e}s.





\textbf{	}Gr\^{a}ce \`{a} l'utilisation du mod\`{e}le de Spacy, j'obtiens de bonnes performances de tokenisation et de lemmatisation, qui permettent d'am\'{e}liorer consid\'{e}rablement les r\'{e}sultats des analyses de \emph{topic modeling}. Empiriquement, apr\`{e}s avoir essay\'{e} diff\'{e}rents lemmatiseurs, celui de Spacy est un de ceux qui fonctionne le mieux pour la langue fran\c{c}aise, ou du moins pour notre corpus. A cet \'{e}gard, il est n\'{e}cessaire de souligner que c'est un corpus de litt\'{e}rature d'anticipation tr\`{e}s contemporaine, et que certains auteurs \'{e}crivent parfois de  sultats des analyses de \emph{topic modeling} sont dans un premier temps assez peu pertinents. Il est donc n\'{e}cessaire de refiltrer manuellement notre liste de lemmes, en utilisant une liste de mots de narration constitu\'{e}e ad hoc. Or, la constitution de cette liste de lemmes constitue, de fait, un probl\`{e}me pour la g\'{e}n\'{e}ralisation de notre module de \emph{topic modeling}. 


	


	Enfin, \'{e}voquons ici le probl\`{e}me de l'\'{e}chantillonnage de nos donn\'{e}es. Une fois nos donn\'{e}es pr\'{e}-trait\'{e}es, et pr\^{e}tes \`{a} l'emploi pour l'impl\'{e}mentation de la LDA, il peut tout de m\^{e}me subsister un probl\`{e}me d'in\'{e}galit\'{e} de longueur des textes. Or, cela constitue un probl\`{e}me pour l'analyse de \emph{topic modeling}, tr\`{e}s sensible \`{a} la longueur des documents. Aussi, j'ai ajout\'{e} \`{a} mon module des fonctions d'\'{e}chantillonnage pour traiter ces cas-l\`{a}. Concr\`{e}tement, c'est fonctions permettent de tirer au sort des \'{e}chantillons de tailles \'{e}gales pour chaque document du corpus. Remarquons ici qu'il est important d'effectuer plusieurs analyses de validation sur diff\'{e}rents tirages au sort pour confirmer une tendance.  





	Impl\'{e}menter la LDA \`{a} l'aide de diff\'{e}rentes m\'{e}thodes





Apr\`{e}s avoir pr\'{e}par\'{e} et \'{e}chantillonn\'{e} nos donn\'{e}es, nous pouvons entrer dans la phase d'impl\'{e}mentation de la LDA. Nous cherchons d'abord \`{a} cr\'{e}er un bag of words \`{a} partir de l'ensemble des mots de notre corpus. Concr\`{e}tement, ce bag of words se pr\'{e}sente sous la forme d'une liste de tuples, contenant l'ID de chaque mot et son nombre d'occurrences : c'est-\`{a}-partir de ce format que nous pouvons impl\'{e}menter la LDA. Comme on l'a expliqu\'{e} pr\'{e}c\'{e}demment, de nombreuses m\'{e}thodes existent. L\`{a} encore, apr\`{e}s une phase d'essais, nous avons retenu les deux m\'{e}thodes dont les r\'{e}sultats semblaient les plus probants. 





\textcolor[rgb]{0.000,0.000,0.000}{Elles se fondent toutes deux sur les fonctions propos\'{e}es par la biblioth\`{e}que gensim. Int\'{e}ressons-nous de plus pr\`{e}s aux fonctions utilis\'{e}es : concr\`{e}tement, elles entra\^{i}nent la machine sur notre bag of words en lui apprenant \`{a} classer les mots en }\emph{\textcolor[rgb]{0.000,0.000,0.000}{topics}}\textcolor[rgb]{0.000,0.000,0.000}{. Initialement, nous avions appliqu\'{e} cette m\'{e}thode aux donn\'{e}es brutes de notre bag of words. Mais cette fonction retournait, lors de nos premiers essais, des r\'{e}sultats assez \'{e}tonnants, et peu coh\'{e}rents. Nous avons tout de m\^{e}me choisi de conserver la possibilit\'{e} d'utiliser cette fonction dans notre module puisqu'il semble qu'elle puisse mieux fonctionner avec certaines donn\'{e}es. Cependant, nous avons \'{e}galement mis au point une deuxi\`{e}me fonction d'impl\'{e}mentation de la LDA, qui se se fonde sur un bag of words dans lequel chaque mot est pond\'{e}r\'{e} gr\^{a}ce \`{a} la m\'{e}thode TF-IDF -- pour term frequency / inverted document frequency. Cette m\'{e}thode permet de mettre en relief les mots discriminants d'un texte au sein d'un corpus. Elle s'est r\'{e}v\'{e}l\'{e}e nettement plus probante dans le cas de notre prototype d'exp\'{e}rimentation de M1.}





On note ici avec int\'{e}r\^{e}t que notre m\'{e}thode permet de jouer sur plusieurs param\`{e}tres pour affiner notre mod\`{e}le. D'abord, le nombre de \emph{topics} que la machine est cens\'{e}e rep\'{e}rer. Il n'y a pas vraiment de r\`{e}gles en la mati\`{e}re; aussi, nous avons mis au point une fonction d'assistance au nombre de choix de \emph{topics}, fond\'{e}e sur la coh\'{e}rence du mod\`{e}le pour diff\'{e}rents nombres de \emph{topics}. Elle permet d'obtenir un graphique d'aide au choix du nombre de \emph{topics}, bien que cette d\'{e}cision ne puisse se passer du regard \'{e}clair\'{e} du chercheur. Ensuite, et c'est l\`{a} un facteur essentiel, le nombre de passes, c'est-\`{a}-dire le nombre d'it\'{e}rations que la machine va effectuer sur notre jeu de donn\'{e}es. Plus la machine s'entra\^{i}ne, plus les \emph{topics} qu'elle d\'{e}termine sont pertinents, et mieux elle pourra appr\'{e}cier chacun des textes au regard de ces \emph{topics}. Ici, on n'est limit\'{e} que par le temps de traitement de l'ordinateur. Il faut ici souligner qu'\`{a} partir d'un certain nombre d'it\'{e}rations, les r\'{e}sultats semblent se stabiliser. Un nombre trop important d'it\'{e}rations serait trop co\^{u}teux en temps, et n'apporterait finalement rien de plus \`{a} notre mod\`{e}le. On peut \'{e}galement jouer sur les donn\'{e}es d'entr\'{e}es, puisque notre module permet de convertir notre liste de lemmes en liste de n-grammes - c'est-\`{a}-dire, d'ensemble de lemmes de longueur n. Par exemple, si nous souhaitions extraire les trigrammes de la phrase {``}Nous sommes all\'{e}s \`{a} la plage hier soir'' qui, lemmatis\'{e}e donnera {``}Nous \^{e}tre aller plage hier soir'', nous obtiendrions {``}nous-\^{e}tre-aller, \^{e}tre-aller-plage, aller-plage-hier, plage-hier-soir''. Or, les n-grammes permettent parfois d'aboutir \`{a} des r\'{e}sultats particuli\`{e}rement probants dans le cadre d'analyses de LDA. Enfin, on dispose dans notre module d'une fonction d'\'{e}valuation de la coh\'{e}rence de notre mod\`{e}le apr\`{e}s entra\^{i}nenement de celui-ci. Ce score de coh\'{e}rence permet de mesurer la performance g\'{e}n\'{e}rale de notre mod\`{e}le et \'{e}ventuellement, de le comparer \`{a} d'autres mod\`{e}les, m\^{e}me si la mesure de la coh\'{e}rence doit \^{e}tre prise avec beaucoup de prudence.








\textbf{	Offrir des solutions de visualisation de donn\'{e}es adapt\'{e}es \`{a} nos r\'{e}sultats}





\textbf{{\Large 	}}Notre module doit finalement offrir de nombreuses solutions de visualisation de donn\'{e}es permettant d'interpr\'{e}ter nos r\'{e}sultats, et de les faire parler de mani\`{e}re plus aboutie. Nous souhaitions notamment impl\'{e}menter diff\'{e}rents types de visualisations permettant d'explorer autant les r\'{e}sultats \`{a} l'\'{e}chelle des \emph{topics} eux-m\^{e}me qu'\`{a} une \'{e}chelle inter-\emph{topic}, pour l'ensemble du corpus ou pour un \'{e}l\'{e}ment sp\'{e}cifique de ce dernier. Pour r\'{e}pondre \`{a} cette n\'{e}cessit\'{e}, nous avons choisi d'impl\'{e}menter diff\'{e}rentes fonctionnalit\'{e}s que nous allons ici pr\'{e}senter.





	On trouve en premier lieu des visualisations assez simples permettant d'afficher nos r\'{e}sultats de mani\`{e}re plus parlante que la sortie console du programme. On mentionnera ici la possibilit\'{e} de produire des dataframes, c'est-\`{a}-dire des tableaux de donn\'{e}es, permettant de synth\'{e}tiser d'une part les informations relatives aux \emph{topics} \'{e}tablis par notre mod\`{e}le sur l'ensemble du corpus, et d'autre part celles relatives \`{a} l'attribution de ces diff\'{e}rents \emph{topics} \`{a} chacun des \'{e}l\'{e}ments du corpus. Nous avons \'{e}galement ajout\'{e} une fonction permettant produire des graphiques \`{a} partir des r\'{e}sultats de notre mod\`{e}le, et notamment d'un graphique permettant de visualiser le nombre d'occurences de chaque mot et leu poids relatif au sein des diff\'{e}rents \emph{topics}. Enfin, le module permet de produire des \emph{wordclouds}, c'est-\`{a}-dire des nuages de mots de chaque \emph{topic}, dans lesquels la taille des mots est variable et correspond \`{a} l'importance relative du mot au sein du \emph{topic}.





	Dans un second temps, nous avons travaill\'{e} sur des fonctionnalit\'{e}s de visualisation nettement plus avanc\'{e}es. D'abord, avec le T-SNE Clustering, pour \emph{t-distributed stochastic neighbor embedding}. Derri\`{e}re ce nom assez complexe se cache une m\'{e}thode de classification assez originale, qui permet, \`{a} l'instar d'une Principal Component Analysis (PCA), de r\'{e}duire la dimensionnalit\'{e} d'un probl\`{e}me complexe et d'en offrir une visualisation en deux dimensions nettement plus manipulable. Dans notre cas, le T-SNE Clustering permet de visualiser en deux dimensions la distance th\'{e}matique entre les diff\'{e}rents \'{e}l\'{e}ments du corpus, \`{a} partir des r\'{e}sultats obtenus par notre mod\`{e}le. Cette solution de visualisation ouvre de nombreuses perspectives pour \'{e}valuer la qualit\'{e} de notre mod\`{e}le, en ce qu'elle permet de porter un regard sur l'homog\'{e}n\'{e}it\'{e} des \emph{topics} obtenus et sur la pr\'{e}sence d'\'{e}ventuels \emph{outliers}, c'est-\`{a}-dire d'\'{e}l\'{e}ments tr\`{e}s \'{e}loign\'{e}s du reste du corpus, qui pourrait attirer notre attention. Ici, il est d'ailleurs int\'{e}ressant de noter qu'un tel r\'{e}sultat doit permettre un aller retour plus fin entre notre m\'{e}thode de distant reading et le corps des textes \'{e}tudi\'{e}s, justement en focalisant notre regard sur les \'{e}l\'{e}ments situ\'{e}s \`{a} la marge du corpus.





	Enfin, la derni\`{e}re fonctionnalit\'{e} qu'il convient de pr\'{e}senter est LDAvis, un outil particuli\`{e}rement puissant et adapt\'{e} \`{a} la visualisation de r\'{e}sultats obtenus avec une analyse de LDA. LDAvis est un package permettant une visualisation interactive de notre \emph{topic} model en ligne. Elle permet notamment, de man\`{e}re tr\`{e}s int\'{e}ressante, de mod\'{e}liser la distance inter-\emph{topic} en deux dimensions ainsi que le poids respectif de chacun des \emph{topics} sur l'ensemble du \emph{topic}. LDAvis permet \'{e}galement de zoomer sur chacun des \emph{topics}, en observant les 30 mots les plus d\'{e}terminants du \emph{topic} ainsi que leur poids relatif. En outre, elle permet de faire varier certains param\`{e}tres en explorant nos \emph{topics}, afin d'aller plus en profondeur sur l'appr\'{e}ciation de la r\'{e}partition des diff\'{e}rents \emph{topics}. C'est sans doute l'outil le plus abouti dont nous disposons pour visualiser et interpr\'{e}ter nos r\'{e}sultats. 








	Conclusions sur notre module





\textbf{	}Pour conclure, nous disposons d\'{e}sormais d'un module pleinement op\'{e}rationnel et facile d'utilisation, offrant une base solide pour nos recherches dans le cadre de ce m\'{e}moire. En permettant de centraliser et de faciliter l'ensemble de la cha\^{i}ne du traitement des donn\'{e}es d'un corpus litt\'{e}raire, du preprocessing des donn\'{e}es jusqu'\`{a} la visualisation des r\'{e}sultats des analyses de \emph{topic modeling}, ce module est parfaitement adapt\'{e} aux analyses que nous souhaitons produire sur notre corpus. 





	En outre, il doit permettre de faciliter et de populariser l'utilisation du \emph{topic modeling} aupr\`{e}s d'autres chercheurs en Humanit\'{e}s Num\'{e}riques. C'est notamment le cas au sein du Master de l'Ecole des Chartes, où plusieurs \'{e}l\`{e}ves ont d\'{e}j\`{a} r\'{e}employ\'{e} ce module dans le cadre de leurs propres recherches. D'ailleurs, le script du module est disponible en open source en ligne, sur Github\footnote{	Voir https://github.com/leo8/ModuleTM} , et se veut accessible et consultable par tout le monde. Il a d'ailleurs vocation \`{a} \^{e}tre am\'{e}lior\'{e} collectivement, et nous invitons tout lecteur int\'{e}ress\'{e} \`{a} consulter le lien indiqu\'{e} et \`{a} ne pas h\'{e}siter \`{a} \'{e}mettre remarque et propositions de modification.
























































































































































\textbf{{\Large 3.}}{\Large  Elaboration d'un mod\`{e}le entra\^{i}n\'{e} sur les donn\'{e}es de notre corpus}








	R\'{e}cup\'{e}ration et normalisation des textes





	A l'issue de la phase de constitution de mon corpus, j'ai pu commencer \`{a} r\'{e}cup\'{e}rer les textes et \`{a} pr\'{e}-traiter les donn\'{e}es textuelles, l'objectif \'{e}tant de parvenir \`{a} des formes normalis\'{e}es de mes textes, au format .txt, pour toutes les oeuvres de mon corpus. Pour les recueils de nouvelles, j'ai choisi d'extraire chaque nouvelle pour en faire un nouveau document. Apr\`{e}s ce premier traitement, je disposais finalement d'un corpus constitu\'{e} de 94 documents, soit 94 oeuvres litt\'{e}raires toutes issues des deux collections \'{e}voqu\'{e}es. L'objectif \'{e}tait alors de parvenir A des formes normalis\'{e}es de mes textes, au format .txt, pour toutes les oeuvres de mon corpus. Pour ce qui est des ouvrages complets, j'ai simplement retir\'{e} manuellement les pr\'{e}faces, tables des mati\`{e}res, mention de {`}chapitres' et de {`}parties' -- en prenant soin de conserver leur titre. Pour les recueils de nouvelles, en revanche, j'ai choisi d'extraire chaque nouvelle pour en faire un nouveau document. J'ai ensuite appliquer le m\^{e}me traitement manuel que pour les ouvrages. Ici, il faut \'{e}galement mentionner le retrait d'une vingtaine de nouvelles, qui \'{e}taient trop courtes pour que le \emph{topic modeling} soit efficace -- il faut un minimum de lemmes par document, de l'ordre de 1000, que ces nouvelles ne permettaient pas d'atteindre.








	Il faut souligner la grande vari\'{e}t\'{e} d'auteurs dont est issue ce corpus -- on en compte 39 -- qui n'\'{e}tait pas forc\'{e}ment un objectif dans la constitution du corpus, mais qui se r\'{e}v\`{e}le une force lorsqu'on souhaite explorer les futurs envisag\'{e}s. Remarquons \'{e}galement que la longueur des diff\'{e}rents documents qui composent le corpus est tr\`{e}s variable : de l'ordre de 1 \`{a} 20, ce qui constitue un probl\`{e}me pour l'analyse de \emph{topic modeling}, tr\`{e}s sensible \`{a} la longueur des documents.  Aussi, j'ai souhait\'{e} enrichir la phase de lemmatisation de nos documents d'un processus d'\'{e}chantillonage. Concr\`{e}tement, c'est fonctions permettent de tirer au sort des \'{e}chantillons de tailles \'{e}gales pour chaque document du corpus. Mais sous cette forme, les donn\'{e}es textuelles doivent encore \^{e}tre pr\'{e}-trait\'{e}es avant de pouvoir utiliser l'algorithme de LDA.








	Echantillonnage et lemmatisation





	Avant d'impl\'{e}menter la LDA, il faut tokeniser et lemmatiser nos documents, c'est-\`{a}-dire transformer nos textes bruts en listes de mots -- tokens -- et rapporter tous ces mots \`{a} une forme racine et invariable -- lemmes - afin de compter le nombre total d'occurrences de ces mots sous leurs diff\'{e}rentes formes. Pour ce faire, notre module dispose de deux fonctions distinctes, faisant appel \`{a} deux modules diff\'{e}rents : spacy et pie-extended. Initialement, je comptais utiliser le lemmatiseur de pie-extended, qui obtient des r\'{e}sultats nettement meilleurs, m\^{e}me s'il est entra\"{e}®n\'{e} sur du fran\c{c}ais classique. Mais l'utilisation de pie-extended reposant largement sur la POS, il convient de tokeniser au niveau des phrases pour l'utiliser correctement. Ainsi, les premi\`{e}res phases d'essais de lemmatisation ont men\'{e} \`{a} des r\'{e}sultats assez \'{e}tranges -- tendance, entre autres, \`{a} rapporter les noms propres et les noms communs finissant par -a \`{a} une forme infinitive, si bien qu j'ai d\'{e}cid\'{e} de revenir au lemmatiseur de spacy qui obtenait des r\'{e}sultats convenables. Il faudra envisager, dans un second temps, de revoir la phase de tokenisation afin de permettre l'utilisation du lemmatiseur de pie-extended et de comparer les r\'{e}sultats obtenus A ceux de spacy. 





	Apr\`{e}s cette premi\`{e}re phase, il subsiste de nombreux lemmes qui n'int\'{e}ressent pas notre analyse th\'{e}matique : aussi, il nous faut intervenir pour \'{e}carter de l'analyse ces \'{e}l\'{e}ments. Ils sedivisent en deux cat\'{e}gories. D'une part, les Stopwords, ou mot-outils, c'est-\`{a}-dire les mots dont le r\"{e}´le syntaxique est important, mais qui n'ont aucune importance s\'{e}mantique : pr\'{e}positions, conjonctions, pronoms, d\'{e}terminants, etc. D'autre part, les mots de narration, qui l\`{a} encore ne sont pas porteur de beaucoup de sens d'un point de vue th\'{e}matique. Il s'agit ici de toutes les entit\'{e}s nomm\'{e}es propres \`{a} l'intrigue des ouvrages -- noms de personnages, de lieux, dates, etc -- ainsi que des verbes de narration, qui servent \`{a} d\'{e}crire l'action mais qui disent peu du fond th\'{e}matique de nos textes. Ici, nous avons donc \'{e}cart\'{e} manuellement ces \'{e}l\'{e}ments, en utilisant une liste de Stopwords \`{a} partir de ce qui existait d\'{e}j\`{a} en ligne et en la compl\'{e}tant au fur et \`{a} mesure de l'analyse, et une liste de mots de narration, constitu\'{e}e ad hoc en explorant la liste de lemmes obtenue avec Spacy. A l'issue de cette phase de lemmatisation, j'ai donc r\'{e}cup\'{e}r\'{e} un vaste ensemble de lemmes, qui conservait la structure de mes documents. Concr\`{e}tement, il s'agissait d'une liste de listes de lemmes, chaque sous liste repr\'{e}sentant les lemmes d'un document. Ces donn\'{e}es sont stock\'{e}es dans un fichier texte, et facilement r\'{e}cup\'{e}rables pour la phase analytique. Cependant, comme on l'a dit plus haut, le \emph{topic modeling} est tr\`{e}s sensible \`{a} la longueur des documents fournis en entr\'{e}e. Aussi, j'ai souhait\'{e} \'{e}chantillonner mes documents pour harmoniser leur longueur. J'ai donc enrichi mon module d'une fonction d'\'{e}chantillonnage permettant de d\'{e}couper mes diff\'{e}rentes listes de lemmes en \'{e}chantillons de taille \'{e}gale. L\`{a} encore, le \emph{topic modeling} est tr\`{e}s sensible \`{a} la question de la taille des documents fournis en entr\'{e}e. Apr\`{e}s diff\'{e}rents essais, il semble qu'un minimum de 1000 lemmes par document soit essentiel au bon fonctionnement de nos m\'{e}thodes. J'ai donc d\'{e}coup\'{e} mes documents en \'{e}chantillons de 1000 lemmes, pour finalement obtenir environ 700 \'{e}chantillons lemmatis\'{e}s, pr\^{e}ts \`{a} \^{e}tre fournis en entr\'{e}e pour la phase analytique.










































































	\textbf{Obtenir un mod\`{e}le pertinent}





	Pour obtenir un mod\`{e}le pertinent, nous avons commenc\'{e} par utiliser la fonction d'assistance au choix du bon nombre de \emph{topics}. Voici le graphique que nous avons obtenu \`{a} partir de cette fonction pour 300 passes et un nombre maximal de 20 \emph{topics}.











Figure 1


\footnote{} 














\textcolor[rgb]{0.000,0.000,0.000}{	On observe deux pics de coh\'{e}rence pour 2 }\emph{\textcolor[rgb]{0.000,0.000,0.000}{topics}}\textcolor[rgb]{0.000,0.000,0.000}{ et 6 }\emph{\textcolor[rgb]{0.000,0.000,0.000}{topics}}\textcolor[rgb]{0.000,0.000,0.000}{. Le mod\`{e}e obtenu avec deux }\emph{\textcolor[rgb]{0.000,0.000,0.000}{topics}}\textcolor[rgb]{0.000,0.000,0.000}{ semble d'ailleurs un peu plus coh\'{e}rent que pour 6 }\emph{\textcolor[rgb]{0.000,0.000,0.000}{topics}}\textcolor[rgb]{0.000,0.000,0.000}{. Cependant, on a d\'{e}j\`{a} expliqu\'{e} que cette fonction permettait une assistance au choix du bon nombre de }\emph{\textcolor[rgb]{0.000,0.000,0.000}{topics}}\textcolor[rgb]{0.000,0.000,0.000}{, mais ne pouvait en aucun cas se substituer au regard du chercheur. Aussi, il est \'{e}vident que nous pr\'{e}f\'{e}rerons travailler avec 6 }\emph{\textcolor[rgb]{0.000,0.000,0.000}{topics}}\textcolor[rgb]{0.000,0.000,0.000}{ plut\^{o}t qu'avec seulement 2, m\^{e}me au prix d'une petite perte de coh\'{e}rence de notre mod\`{e}le. C'est ce param\`{e}tre que nous retiendrons pour la suite de notre \'{e}tude}





\textcolor[rgb]{0.000,0.000,0.000}{	A partir de l\`{a}, nous avons travaill\'{e} \`{a} affiner notre mod\`{e}le, en l'entra\^{i}nant avec davantage d'it\'{e}rations sur nos donn\'{e}es. Nous avons, \`{a} ce stade de nos recherches, rencontr\'{e} un deuxi\`{e}me questionnement majeur, sur lequel il nous faut ici revenir. En effet, nous avons expliqu\'{e} plus haut que notre module poss\'{e}dait deux fonctions pour impl\'{e}menter la LDA, l'une ayant recours aux donn\'{e}es brutes de notre bag of words, et l'autre se fondant sur la m\'{e}thode TF-IDF, que nous tenions pour plus efficace compte tenu d'exp\'{e}rimentations men\'{e}es sur un prototype.}


\textcolor[rgb]{0.000,0.000,0.000}{	Or, en menant de brefs essais \`{a} l'aide des deux fonctions, et en confrantant les r\'{e}sultats obtenus avec et sans TF-IDF, il nous est apparu que c'\'{e}tait sans TF-IDF que nous obtenions les meilleurs r\'{e}sultats. Voici deux }\emph{\textcolor[rgb]{0.000,0.000,0.000}{wordclouds }}\textcolor[rgb]{0.000,0.000,0.000}{obtenus avec les premiers mod\`{e}les -- assez peu entra\^{i}n\'{e}s -- pour chacune des deux fonctions.}














Figure 2








\textcolor[rgb]{0.000,0.000,0.000}{Wordcloud Model 5 - 3000 passes - 3000 iter (sans TF-IDF)}





\footnote{} 























Figure 3








Wordcloud Model 4 - 5000 passes (avec TF-IDF)





\footnote{} 














	En comparant bri\`{e}vement ces deux nuages de mots, on se rend compte que le mod\`{e}le qui n'utilise pas la m\'{e}thode TF-IDF fait beaucoup plus de sens. Le deuxi\`{e}me est tr\`{e}s redondant, puisque diff\'{e}rents \emph{topics} partagent les m\^{e}mes mots (angoisser, chloroforme, d\'{e}sinvolture, ins\'{e}curit\'{e}). Si on r\'{e}fl\'{e}chit plus longuement au probl\`{e}me qui se pose \`{a} nous, on peut entrevoir une piste explicative pour comprendre les moins bonnes performances du mod\`{e}le ayant recours \`{a} TF-IDF. Comme on l'a dit, la m\'{e}thode TF-IDF s'applique \`{a} faire ressortir les mots discriminants au sein de chacun des textes. Or, dans le cadre d'une exploration th\'{e}matique de la s\'{e}mantique employ\'{e}e dans notre corpus, nous souhaitons au contraire faire ressortir les th\'{e}matiques communes qui transcendent notre corpus et se retrouvent de mani\`{e}re abondante dans diff\'{e}rents textes. Les mots tr\`{e}s sp\'{e}cifiques \`{a} chacune des oeuvres, au contraire, n'int\'{e}ressent que dans une moindre mesure notre analyse, puisqu'ils renvoient \`{a} des th\'{e}matiques trop proches des intrigues, et nous disent que tr\`{e}s peu des repr\'{e}sentations partag\'{e}es.





	A partir de ces diff\'{e}rents \'{e}l\'{e}ments, nous avons donc entra\^{i}n\'{e} diff\'{e}rents mod\`{e}les -- plus d'une dizaine en tout, que l'on pourra retrouver en annexe - en affinant tout particulirement un mod\`{e}le qui n'utilisait pas TF-IDF : le mod\`{e}le 12, entra\^{i}n\'{e} avec 5000 passes et 5000 it\'{e}rations. En parall\`{e}le, nous avons \'{e}galement entra\^{i}n\'{e} plusieurs mod\`{e}les sur les diff\'{e}rentes listes de n-grams dont nous disposions, quoiqu'avec un entra\^{i}nement plus all\'{e}g\'{e} - 500 passes et 500 it\'{e}rations. Pour comprendre comment, \`{a} partir des fonctions de notre module, nous avons proc\'{e}d\'{e} pour entra\^{i}ner les mod\`{e}les et obtenir des r\'{e}sultats, nous invitons le lecteur \`{a} consulter le script pr\'{e}sent en annexe\footnote{	Cf Script TMAnalysis} . Pr\'{e}sentons donc maintenant les r\'{e}sultats que nous obtenons avec ces diff\'{e}rents mod\`{e}les.






















































































III) \uline{R\'{e}sultats et retours m\'{e}thodologiques}











1. Pr\'{e}sentation des r\'{e}sultats








	\textbf{Les r\'{e}sultats de notre mod\`{e}le final}





\textbf{\textcolor[rgb]{0.000,0.000,0.000}{	}}\textcolor[rgb]{0.000,0.000,0.000}{Commen\c{c}ons ici par pr\'{e}senter les r\'{e}sultats obtenus sur notre corpus \'{e}chantillonn\'{e} avec le mod\`{e}le 12, c'est-\`{a}-dire le mod\`{e}le le plus affin\'{e}, entra\^{i}n\'{e} pour 6 }\emph{\textcolor[rgb]{0.000,0.000,0.000}{topics}}\textcolor[rgb]{0.000,0.000,0.000}{ avec 5000 passes et 5000 it\'{e}rations. Voici un }\emph{\textcolor[rgb]{0.000,0.000,0.000}{wordcloud}}\textcolor[rgb]{0.000,0.000,0.000}{ et un graphique repr\'{e}sentant les mots principaux des diff\'{e}rents }\emph{\textcolor[rgb]{0.000,0.000,0.000}{topics}}\textcolor[rgb]{0.000,0.000,0.000}{, tous deux obtenus \`{a} partir de notre mod\`{e}le.}





\textcolor[rgb]{0.000,0.000,0.000}{Figure 4}


\footnote{} 


\textcolor[rgb]{0.000,0.000,0.000}{Wordcloud mod\`{e}le 12 -- 5000 passes -- 5000 iter}








Figure 5





Graphique des mots cl\'{e}s mod\`{e}le 12 -- 5000 passes -- 5000 iter





\footnote{} 











\textcolor[rgb]{0.000,0.000,0.000}{	Ces deux r\'{e}sultats nous permettent ici de faire quelques remarques int\'{e}ressantes. D'abord, int\'{e}ressons nous aux 6 }\emph{\textcolor[rgb]{0.000,0.000,0.000}{topics}}\textcolor[rgb]{0.000,0.000,0.000}{ en eux-m\^{e}mes. }





	Le premier est relativement coh\'{e}rent. Il semble renvoyer \`{a} un imaginaire m\'{e}di\'{e}val, chevaleresque, avec des termes comme {``}princesse'', {``}mort'', {``}village'', {``}soldat'', {``}cheval'' et {``}fleur''. Le mot {``}bibendum'', qui d\'{e}signe la mascotte de la marque Renault, et dont la figure a \'{e}t\'{e} r\'{e}employ\'{e}e dans diff\'{e}rentes oeuvres de science-fiction, n'a pas \'{e}norm\'{e}ment de coh\'{e}rence avec le reste du \emph{topic}. En outre, on note la pr\'{e}sence de noms propres, qui auraient pourtant d\^{u} \^{e}tre \'{e}cart\'{e}s lors de la phase de preprocessing : presque toujours pr\'{e}sents, quoiqu'en petit nombre, dans les \emph{topics} que nous obtenons en r\'{e}sultats, nous reviendrons plus longuement sur cette question.





	Le deuxi\`{e}me \emph{topic}, quant \`{a} lui, ne peut que difficilement \^{e}tre associ\'{e} \`{a} une th\'{e}matique propre. On retrouve quelques mots qui peuvent \^{e}tre, de pr\`{e}s ou de loin, associ\'{e}s au th\`{e}me du voyage, avec par exemple {``}route'', {``}camion'', {``}radio'', {``}animal'', ou {``}glace''. Mais on retrouve \'{e}galement des termes assez \'{e}loign\'{e}s th\'{e}matiquement, comme {``}minuscule'', {``}sang'' et {``}bouton'', qu'il est difficile d'associer \`{a} de m\^{e}mes repr\'{e}sentations. Enfin, on trouve les noms propres {``}jarel'' et {``}kokoï''.





	Le troisi\`{e}me \emph{topic}, en revanche, est assez coh\'{e}rent et renvoie tr\`{e}s largement aux th\'{e}matiques de l'exploration et de la conqu\^{e}te spatiales. On retrouve ainsi les termes {``}plan\`{e}te'', {``}espace'', {``}humain'', {``}cr\'{e}ature'' et {``}astronef''. On retrouve \'{e}galement le terme {``}sconge'', qui renvoie \`{a} une cr\'{e}ature extraterrestre''. Enfin, on retrouve encore une fois quelques noms propres : {``}n\'{e}da'', {``}norv'' et {``}luc''.





	Le quatri\`{e}me \emph{topic} semble renvoyer aux th\`{e}mes de la famille, avec des termes comme {``}situation'', {``}enfant'', {``}marier'' et {``}maison''. On retrouve \'{e}galement les couleurs {``}blanc'' et {``}noir'' qui peuvent \^{e}tre associ\'{e}es s\'{e}mantiquement au th\`{e}me \'{e}voqu\'{e}. Enfin, on retrouve les termes {``}sable'', et {``}ombre'', ainsi que les verbes {``}descendre'' et {``}recouvrir'' qui, sans \^{e}tre en pleine ad\'{e}quation avec le th\`{e}me de la famille, n'en sont pas franchement \'{e}loign\'{e}s. Aussi, ce quatri\`{e}me \emph{topic} appara\^{i}t \'{e}galement comme relativement coh\'{e}rent.





	Le cinqui\`{e}me \emph{topic} ne rassemble que des verbes de narration, que nous avions pourtant \'{e}cart\'{e} de l'analyse. On trouve ainsi les termes {``}faire'', {``}dire'', {``}pouvoir'', {``}aller'', {``}savoir'', {``}venir'', {``}voir'', {``}devoir'', {``}prendre'', ainsi que le mot assez g\'{e}n\'{e}rique -- surtout en fran\c{c}ais -- {``}temps''. Ici, le \emph{topic} est parfaitement coh\'{e}rent en ce qu'il rassemble des termes d'une m\^{e}me nature morphosyntaxique; cependant, ces termes auraient d\^{u} \^{e}tre \'{e}cart\'{e}s lors de la phase de preprocessing des donn\'{e}es. En fait, la pr\'{e}sence de ce th\`{e}me r\'{e}v\`{e}le les limites des fonctions de notre module en mati\`{e}re de pr\'{e}traitement des donn\'{e}es : certaines formes des verbes qu'on a mentionn\'{e}s \'{e}chappent \`{a} notre algorithme, si bien qu'ils se retrouvent en grand nombre dans notre analyse. Nous pouvons donc ignorer ce \emph{topic} dans notre travail d'interpr\'{e}tation. 





	Enfin, le dernier \emph{topic} n'est pas tr\`{e}s coh\'{e}rent. On retrouve des mots associ\'{e}s au th\`{e}me de la famille, comme {``}p\`{e}re'' et {``}fr\`{e}re'' -- on peut d'ailleurs ici se demander pourquoi ils n'ont pas \'{e}t\'{e} rang\'{e}s dans le quatri\`{e}me \emph{topic}. On retrouve \'{e}galement, de mani\`{e}re esseul\'{e}e, les termes {``}cave'', d'une part, et {``}art'' et {``}imaginer'', d'autre part. Enfin, on retrouve de nombreux noms propres : {``}geroges'', {``}bor'', {``}serge'', {``}laetitia'' et {``}lovskovitch''.





\textcolor[rgb]{0.000,0.000,0.000}{	Revenons justement sur la pr\'{e}sence de noms propres dans les diff\'{e}rents }\emph{\textcolor[rgb]{0.000,0.000,0.000}{topics}}\textcolor[rgb]{0.000,0.000,0.000}{ que nous venons de pr\'{e}senter. De mani\`{e}re analogue aux verbes de narration, ils auraient d\^{u} \^{e}tre \'{e}cart\'{e}s lors de la phase de preprocessing des donn\'{e}es. Cependant, l\`{a} encore, on assiste aux limites de nos fonctions de pr\'{e}traitement, qui ne parviennent pas \`{a} distinguer tous les noms propres. Nous avons travaill\'{e} manuellement \`{a} \'{e}carter les noms propres en les ajoutant \`{a} la liste des mots de narration que nous supprimons lors du preprocessing. Mais leur omnipr\'{e}sence dans l'ensemble du corpus, ainsi que leur grande fr\'{e}quence d'occurrence -- puisqu'il s'agit de romans, les noms des personnages sont cit\'{e}s un grand nombre de fois - rend cette t\^{a}che difficile. Nous avons donc d\^{u} nous r\'{e}soudre \`{a} ignorer leur pr\'{e}sence dans notre intepr\'{e}tation th\'{e}matique des }\emph{\textcolor[rgb]{0.000,0.000,0.000}{topics}}\textcolor[rgb]{0.000,0.000,0.000}{.}





	Afin de se livrer \`{a} un examen plus approfondi des r\'{e}sultats du mod\`{e}le 12 et des \emph{topics} obtenus, nous allons d\'{e}sormais explorer les solutions de visualisation offertes par le module LDAvis. Nous pr\'{e}sentons ici les screenshots de ces visualisations, avec les diff\'{e}rents \emph{topics} s\'{e}lectionn\'{e}s. Pour consulter directement la visualisation en ligne, nous invitons le lecteur \`{a} consulter le lien qui m\`{e}ne directement \`{a} celle-ci\footnote{	file:///home/leo/Desktop/M\%C3\%A9moire\%20M2/Annexes/Results/LDAvis-Model12-6\emph{topics}-5000passes-5000iter.html} .




















Figure 6 





LDAvis model 12 -- 5000 passes 5000 iter -- no \emph{topic} selected


\footnote{} 


Figure 7





\footnote{} 


LDAvis model 12 -- 5000 passes 5000 iter -- \emph{topic} 1 selected


Figure 8





LDAvis model 12 -- 5000 passes 5000 iter -- \emph{topic} 2 selected


\footnote{} 


Figure 9





\footnote{} 


LDAvis model 12 -- 5000 passes 5000 iter -- \emph{topic} 3 selected


Figure 10





LDAvis model 12 -- 5000 passes 5000 iter -- \emph{topic} 4 selected


\footnote{} 


Figure 11





LDAvis model 12 -- 5000 passes 5000 iter -- \emph{topic} 5 selected


\footnote{} 


Figure 12





LDAvis model 12 -- 5000 passes 5000 iter -- \emph{topic} 6 selected





\footnote{} 








\textcolor[rgb]{0.000,0.000,0.000}{	Ces visualisations permettent, en premier lieu, d'explorer plus en profondeur chacun des }\emph{\textcolor[rgb]{0.000,0.000,0.000}{topics}}\textcolor[rgb]{0.000,0.000,0.000}{. En observant davantage de mots pour chaque }\emph{\textcolor[rgb]{0.000,0.000,0.000}{topic}}\textcolor[rgb]{0.000,0.000,0.000}{, on peut mettre en perspective les interpr\'{e}tations qu'on avait faites, et les affiner. Ainsi du }\emph{\textcolor[rgb]{0.000,0.000,0.000}{topic}}\textcolor[rgb]{0.000,0.000,0.000}{ 3, qu'on avait associ\'{e} \`{a} l'exploration et \`{a} la conqu\^{e}te spatiales. A ce dernier s'ajoutent notamment les termes {``}civilisation'', {``}intelligence'', {``}terrien'', {``}dimension'', {``}race'', {``}espace'', {``}survivant et {``}espoir'', qui appuient encore notre interpr\'{e}tation.}





\textcolor[rgb]{0.000,0.000,0.000}{	On mesure \'{e}galement, gr\^{a}ce aux visualisations offertes par LDAvis, l'importance relative des }\emph{\textcolor[rgb]{0.000,0.000,0.000}{topics}}\textcolor[rgb]{0.000,0.000,0.000}{ dans l'ensemble du corpus. Sans surprise, c'est le }\emph{\textcolor[rgb]{0.000,0.000,0.000}{topic}}\textcolor[rgb]{0.000,0.000,0.000}{ associ\'{e} aux verbes de narration (}\emph{\textcolor[rgb]{0.000,0.000,0.000}{topic}}\textcolor[rgb]{0.000,0.000,0.000}{ 5) qui repr\'{e}sente la part la plus cons\'{e}quente de tokens dans notre corpus. Viennent ensuite, respectivement, celui qu'on avait associ\'{e} \`{a} la famille (}\emph{\textcolor[rgb]{0.000,0.000,0.000}{topic}}\textcolor[rgb]{0.000,0.000,0.000}{ 4), celui de l'exploration et de la conqu\^{e}te spatiales (}\emph{\textcolor[rgb]{0.000,0.000,0.000}{topic}}\textcolor[rgb]{0.000,0.000,0.000}{ 3), celui qui nous semblait vaguement li\'{e} au voyage (}\emph{\textcolor[rgb]{0.000,0.000,0.000}{topic}}\textcolor[rgb]{0.000,0.000,0.000}{ 2), puis celui renvoyant \`{a} des repr\'{e}sentations chevaleresques et m\'{e}di\'{e}vales (}\emph{\textcolor[rgb]{0.000,0.000,0.000}{topic}}\textcolor[rgb]{0.000,0.000,0.000}{ 1) et enfin celui dont on n'avait pas pu tirer d'interpr\'{e}tation tr\`{e}s claire (}\emph{\textcolor[rgb]{0.000,0.000,0.000}{topic}}\textcolor[rgb]{0.000,0.000,0.000}{ 6).}





\textcolor[rgb]{0.000,0.000,0.000}{	Enfin, ces visualisations nous permettent de faire quelques commentaires sur la distance inter-}\emph{\textcolor[rgb]{0.000,0.000,0.000}{topic}}\textcolor[rgb]{0.000,0.000,0.000}{ et les axes structurants pour notre mod\`{e}le. On note que les }\emph{\textcolor[rgb]{0.000,0.000,0.000}{topics}}\textcolor[rgb]{0.000,0.000,0.000}{ 1 et 2 sont tous deux relativement singuliers, et se distinguent particuli\`{e}rement des deux autres. Les }\emph{\textcolor[rgb]{0.000,0.000,0.000}{topics}}\textcolor[rgb]{0.000,0.000,0.000}{ 3 \`{a} 6, en revanche, sont assez proches, et se superposent m\^{e}me par endroits. On peut \'{e}galement noter que le premier axe structurant (PC1) semble opposer th\'{e}matiques terrestres et spatiales, tandis que le deuxi\`{e}me (PC2) correspond \`{a} l'opposition entre termes de narration et termes propres aux intrigues.}


\textbf{\textcolor[rgb]{0.000,0.000,0.000}{	R\'{e}sultats compl\'{e}mentaires obtenus sur nos listes de n-grams}}





	Ici, nous allons pr\'{e}senter plus succintement les r\'{e}sultats que nous avons obtenus sur les diff\'{e}rentes listes de n-grams, et qui r\'{e}v\`{e}lent parfois certaines composantes th\'{e}matiques int\'{e}ressantes. Sans nous adonner \`{a} une analyse aussi aboutie que pour le mod\`{e}le pr\'{e}c\'{e}dent, nous les commenterons toutefois en relevant les \'{e}l\'{e}ments que nous jugerons particuli\`{e}rement pertinents. Nous proc\'{e}derons dans l'ordre du nombre de grams, c'est-\`{a}-dire en allant des analyses port\'{e}es sur les bi-grammes \`{a} celles qui concernent les heptagrammes. Nous pr\'{e}senterons d'abord l'ensemble des visualisations, avant de faire quelques commentaires.











Figure 13





Wordcloud mod\`{e}le 6 bigrammes -- 500 passes -- 500 iter





\footnote{} 














Figure 14





\footnote{} 


Graphique des mots cl\'{e}s mod\`{e}le 6 bigrammes -- 500 passes -- 500 iter
































	





























\footnote{} 


Figure 15





Wordcloud mod\`{e}le 7 trigrammes -- 500 passes -- 500 iter















































Figure 16





Graphique des mots cl\'{e}s mod\`{e}le 7 trigrammes -- 500 passes -- 500 iter





\footnote{} 
























































Figure 17





Wordcloud mod\`{e}le 8 t\'{e}tragrammes -- 500 passes -- 500 iter








\footnote{} 





























Figure 18





Graphique des mots cl\'{e}s mod\`{e}le 8 t\'{e}tragrammes -- 500 passes -- 500 iter


\footnote{} 






























































Figure 19





Wordcloud mod\`{e}le 9 pentagrammes -- 500 passes -- 500 iter











\footnote{} \footnote{} 














Figure 20





Graphique des mots cl\'{e}s mod\`{e}le 9 pentagrammes -- 500 passes -- 500 iter








\footnote{} 








Figure 21





Wordcloud mod\`{e}le 10 hexagrammes -- 500 passes -- 500 iter

















Figure 22





Graphique des mots cl\'{e}s mod\`{e}le 10 hexagrammes -- 500 passes -- 500 iter


\footnote{} 


















































Figure 23





Wordcloud mod\`{e}le 11 heptagrammes -- 500 passes -- 500 iter





\footnote{} 




















\footnote{} 


Figure 24





Graphique mots-cl\'{e}s mod\`{e}le 11 heptagrammes -- 500 passes -- 500 iter




















\textcolor[rgb]{0.000,0.000,0.000}{	Oublions dans un premier temps les questions li\'{e}es \`{a} la forme. Nous reviendrons plus tard sur les probl\`{e}mes de lisibilit\'{e} que posent ces visualisations \`{a} partir d'un certain nombre de }\emph{\textcolor[rgb]{0.000,0.000,0.000}{topics}}\textcolor[rgb]{0.000,0.000,0.000}{.}





\textcolor[rgb]{0.000,0.000,0.000}{	Mais faisons plut\^{o}t ici quelques observations sur les r\'{e}sultats des listes de n-grams. De mani\`{e}re int\'{e}ressante, trois mots sont souvent au centre des n-grams les plus fr\'{e}quents : {``}phrase'', {``}physique'' et {``}ph\'{e}nom\`{e}ne''. Ils sont ensuite associ\'{e}s \`{a} une multitude de combinaison de mots diff\'{e}rents. Les termes {``}physique'' et {``}ph\'{e}nom\`{e}ne'' montrent l'omni-pr\'{e}sence de th\'{e}matiques li\'{e}es \`{a} des questionnements scientifiques. Ils sont souvent associ\'{e}s \`{a} des concepts propres aux d\'{e}couvertes scientifiques qui marquent la p\'{e}riode de production de ces oeuvres : ainsi, on trouve par exemple les termes {``}\'{e}lectron'' et {``}atmosph\`{e}re''. }





	Le terme {``}physicien'' appara\^{i}t \'{e}galement de mani\`{e}re r\'{e}currente dans l'ensemble des visualisations que nous venons de pr\'{e}senter. Il montre que les th\'{e}matiques relatives \`{a} des questionnements scientifiques apparaissent de mani\`{e}re incarn\'{e}e, souvent sous la figure d'un personnage central, dans les oeuvres qui composent notre corpus.


	Toutefois, il faut peut-\^{e}tre ici nuancer les observations que nous menons sur nos r\'{e}sultats. Comme on peut le voir sur les graphiques, l'importance relative des groupes de n-grams au sein de chaque \emph{topic} est n\'{e}gligeable, et d\'{e}cro\^{i}t \`{a} mesure que l'on augmente le nombre de grammes pris en compte. Cela ne semble pas anormal : plus les n-grams auxquels on s'int\'{e}resse comptent de mots, moins ils ont de chance d'\^{e}tre retrouv\'{e}s en ces termes exacts ailleurs dans le texte. Cependant, cette observation invite \`{a} la prudence lorsqu'il s'agit de tirer des conclusions de ces r\'{e}sultats.








\textbf{\textcolor[rgb]{0.000,0.000,0.000}{{\Large 2. }}}\textcolor[rgb]{0.000,0.000,0.000}{{\Large Int\'{e}r\^{e}ts et limites des r\'{e}sultats obtenus}}








\textcolor[rgb]{0.000,0.000,0.000}{	Il nous reste d\'{e}sormais \`{a} conclure, et \`{a} mettre en perspective les r\'{e}sultats obtenus : quels enseignements peut-on tirer de ces travaux, pour l'histoire des repr\'{e}sentations associ\'{e}es au futur d'une part, et d'autre part, d'un point de vue purement m\'{e}thodologique ? Quels auront \'{e}t\'{e} les int\'{e}r\^{e}ts de ces recherches ? Quels en sont les limites, et quelles perspectives s'ouvrent \`{a} nous d\'{e}sormais ?}








\textcolor[rgb]{0.000,0.000,0.000}{{\Large 	}}\textbf{\textcolor[rgb]{0.000,0.000,0.000}{Un int\'{e}r\^{e}t m\'{e}thodologique avant tout}}





\textbf{\textcolor[rgb]{0.000,0.000,0.000}{	}}\textcolor[rgb]{0.000,0.000,0.000}{Tout d'abord, \'{e}voquons la question des r\'{e}sultats obtenus et de leur contribution \`{a} l'histoire des repr\'{e}sentations associ\'{e}es \`{a} l'avenir. A partir des mod\`{e}les obtenus et des diff\'{e}rentes visualisations qu'on a pr\'{e}sent\'{e}es, on obtient quelques \'{e}l\'{e}ments th\'{e}matiques int\'{e}ressants pour sonder l'imaginaire projectionniste de la p\'{e}riode \'{e}tudi\'{e}e. La pr\'{e}sence relativement abondante du th\`{e}me de la conqu\^{e}te et de l'exploration spatiales, en particulier, semble indiquer que la projection de la civilisation au-del\`{a} des fronti\`{e}res terrestres renvoie \`{a} un questionnement largement partag\'{e} dans la litt\'{e}rature d'anticipation contemporaine. De m\^{e}me pour les deux th\'{e}matiques du voyage et de la famille, qui renvoient \`{a} des mutations socio-\'{e}conomiques propres \`{a} la p\'{e}riode d'apr\`{e}s-guerre, et impr\`{e}gnent l'imaginaire social d'apr\`{e}s-guerre.}





\textcolor[rgb]{0.000,0.000,0.000}{	En outre, la grande fr\'{e}quence d'apparition des termes {``}physique'' et {``}ph\'{e}nom\`{e}ne'', ainsi que la figure omnipr\'{e}sente du physicien, t\'{e}moignent d'une \'{e}poque marqu\'{e}e par les transformations technologiques. Les repr\'{e}sentations associ\'{e}es au futur en sont d'autant plus empreintes de questionnements scientifiques, questionnements qui, semble-t-il trasncendent la litt\'{e}rature d'anticipation contemporaine.}





\textcolor[rgb]{0.000,0.000,0.000}{	On s'\'{e}tonnera, en revanche, de l'absence de certaines repr\'{e}sentations dont on aurait pens\'{e} qu'elles se retrouveraient tr\`{e}s fr\'{e}quemment dans l'imaginaire associ\'{e} au futur. On pense notamment \`{a} la th\'{e}matique de l'\'{e}nergie nucl\'{e}aire, et en particulier \`{a} l'id\'{e}e d'une apocalypse nucl\'{e}aire. M\^{e}me si on retrouve le mot {``}radiation'' dans le deuxi\`{e}me }\emph{\textcolor[rgb]{0.000,0.000,0.000}{topic}}\textcolor[rgb]{0.000,0.000,0.000}{, on en compte relativement peu d'occurrences, et peu ou pas d'autres termes li\'{e}s \`{a} cette th\'{e}matique. }





\textcolor[rgb]{0.000,0.000,0.000}{	Ces observations permettent donc de mettre en perspective, gr\^{a}ce \`{a} des r\'{e}sultats obtenus de mani\`{e}re quantitative, ce que l'on sait des repr\'{e}sentations futuristes de l'\'{e}poque. Elles montrent tout l'int\'{e}r\^{e}t de notre m\'{e}thode pour sonder et appr\'{e}hender l'imaginaire projectionniste d'une p\'{e}riode.}





\textcolor[rgb]{0.000,0.000,0.000}{	Cependant, il faut bien reconna\^{i}tre qu'il s'agit encore d'un mat\'{e}riau insuffisant pour b\^{a}tir v\'{e}ritablement une histoire des repr\'{e}sentations associ\'{e}es \`{a} l'avenir. A ce stade, ces analyses peuvent constituer, au mieux, un outil compl\'{e}mentaire pour les historiens des repr\'{e}sentations. L'int\'{e}r\^{e}t de cette \'{e}tude, en termes de r\'{e}sultats, semble donc limit\'{e}. Mais nous reviendrons plus tard sur les limites de cette \'{e}tude. Car si les r\'{e}sultats de ces recherches ne constituent pas un mat\'{e}riau suffisant pour l'histoire des repr\'{e}sentations, l'int\'{e}r\^{e}t m\'{e}thodologique, lui, est tout \`{a} fait remarquable.}





\textcolor[rgb]{0.000,0.000,0.000}{	D'abord, parcqu'il est certain que tout le questionnement pr\'{e}alable qui constitue le point de d\'{e}part de cette enqu\^{e}te, et qui concerne les repr\'{e}sentations associ\'{e}es au futur et les mani\`{e}res de les appr\'{e}hender, est d'int\'{e}r\^{e}t pour quiconque s'int\'{e}resse \`{a} l'histoire des repr\'{e}sentations. En particulier, toute la discussion sur l'existence d'un imaginaire social et de repr\'{e}sentations associ\'{e}es au futur, et sur le lien entre cet imaginaire et la litt\'{e}rature d'anticipation, d\'{e}friche bon nombre de questionnements auxquels les r\'{e}ponses sont loin d'\^{e}tre \'{e}videntes.}





\textcolor[rgb]{0.000,0.000,0.000}{	Ensuite, parce que l'utilisation du }\emph{\textcolor[rgb]{0.000,0.000,0.000}{topic modeling}}\textcolor[rgb]{0.000,0.000,0.000}{ ouvre de nombreuses perspectives d'application sur des corpus litt\'{e}raires, m\^{e}me si elle ne permet pas encore de mener \`{a} des r\'{e}sultats aboutis dans le cadre de notre \'{e}tude. Il s'agit, en effet, d'une m\'{e}thode particuli\`{e}rement int\'{e}ressante pour explorer de vastes corpus et syst\'{e}matiser leur analyse. Or, puisque la quantit\'{e} d'oeuvres litt\'{e}raires cro\^{i}t exponentiellement, c'est pr\'{e}cis\'{e}ment l'un des d\'{e}fis qui se posent aux chercheurs en histoire de la litt\'{e}rature. Nous proposons donc, avec cette exemple d'application, d'explorer une voie d'avenir en posant quelques bases essentielles. }





	Notre m\'{e}thode ainsi que notre d\'{e}marche, en ce qu'elles ouvrent de nombreuses perspectives pour la recherche en histoire de la litt\'{e}rature, poss\`{e}dent donc un int\'{e}r\^{e}t certain. Il faut cependant reconna\^{i}tre qu'elles sont limit\'{e}es dans les r\'{e}sultats qu'elles apportent sur l'objet m\^{e}me de notre \'{e}tude, c'est-\`{a}-dire l'\'{e}tude de l'histoire des repr\'{e}sentations associ\'{e}es au futur.








	Les limites de notre \'{e}tude





	Abordons donc ici la question des limites de nos recherches. D'abord, sur le plan de la forme, nous nous int\'{e}resserons \`{a} des aspects li\'{e}s \`{a} la pr\'{e}sentation et aux outils de visualisation de nos r\'{e}sultats. Ensuite, sur celui du fond, nous \'{e}voquerons les diff\'{e}rentes conclusions auxquelles nous sommes parvenus, en soulevant la question des limites de nos r\'{e}sultats en termes d'apport \`{a} la recherche.





\textcolor[rgb]{0.000,0.000,0.000}{	Sur le plan de la forme, donc, nous pouvons \'{e}voquer quelques probl\`{e}mes de lisibilit\'{e} sur certains de nos graphiques. En particulier, avec les analyses de n-grams, qui deviennent assez charg\'{e}es d\`{e}s lors qu'on observe des listes de plus de trois grammes. En outre, le format \`{a} respecter pour la bonne insertion de ces graphiques et nuages de mots au sein de notre m\'{e}moire nous oblige \`{a} limiter la taille de ces graphiques, n'arrangeant pas la question de leur lisibilit\'{e}. On peut \'{e}galement faire un commentaire sur les visualisations des r\'{e}sultats qu'on obtient avec LDAvis, pour lesquelles on a du se contenter d'int\'{e}grer une image et un lien. Pourtant, la page internet \`{a} laquelle on acc\`{e}de en obtenant ces r\'{e}sultats est nettement plus int\'{e}ressante, en ce qu'elle est con\c{c}ue de mani\`{e}re interactive et offre de belles perspectives d'exploration des }\emph{\textcolor[rgb]{0.000,0.000,0.000}{topics}}\textcolor[rgb]{0.000,0.000,0.000}{.}





\textcolor[rgb]{0.000,0.000,0.000}{	Ces consid\'{e}rations triviales peuvent sembler assez anecdotiques, mais il est vraiment important d'offrir des visualisations particuli\`{e}rement claires et parlantes dans le cadre de notre \'{e}tude : elle qui se fonde d\'{e}j\`{a} sur une m\'{e}thode compliqu\'{e}e, peu \'{e}vidente \`{a} comprendre pour les non-initi\'{e}s, doit pouvoir pr\'{e}tendre \`{a} une pr\'{e}sentation de r\'{e}sultats clairs. Aussi, il est essentiel de bien comprendre les limites des possibilit\'{e}s de visualisation des }\emph{\textcolor[rgb]{0.000,0.000,0.000}{topic models}}\textcolor[rgb]{0.000,0.000,0.000}{ obtenus en r\'{e}sultats. Il faut par exemple relever que, pour des raisons de clart\'{e}, toutes ces visualisations se contentent de pr\'{e}senter les dix -- pour nos wordclouds, graphiques - ou trente mots -- pour LDAvis - les plus pertinents, mais qu'ils ne pr\'{e}sentent pas l'ensemble des mots associ\'{e}s \`{a} chaque }\emph{\textcolor[rgb]{0.000,0.000,0.000}{topic}}\textcolor[rgb]{0.000,0.000,0.000}{. Pour conclure bri\`{e}vement sur cet aspect, il s'agit donc pour nous de rester conscients des limites de nos pr\'{e}sentations de r\'{e}sultats et de rester ouverts \`{a} toutes nouvelles perspectives de visualisations des }\emph{\textcolor[rgb]{0.000,0.000,0.000}{topic models}}\textcolor[rgb]{0.000,0.000,0.000}{.}





\textcolor[rgb]{0.000,0.000,0.000}{	De mani\`{e}re plus int\'{e}ressante, \'{e}voquons maintenant la question plus d\'{e}licate des limites de nos r\'{e}sultats sur le plan du fond. Il faut ici admettre que nos r\'{e}sultats sont assez limit\'{e}s dans l'apport qu'ils font \`{a} l'histoire des futurs envisag\'{e}s, alors m\^{e}me que c'\'{e}tait l\`{a} l'objectif premier de cette \'{e}tude. Si on parvient \`{a} identifier quelques \'{e}l\'{e}ments th\'{e}matiques structurants pour les repr\'{e}sentations futuristes, si quelques termes et sujets int\'{e}ressants se distinguent, il serait pr\'{e}tentieux d'en d\'{e}duire que notre m\'{e}thode permet d'explorer les imaginaires et les repr\'{e}sentations de mani\`{e}re aboutie. Il nous faut donc ici aborder r\'{e}trospectivement la question de l'int\'{e}r\^{e}t du choix du }\emph{\textcolor[rgb]{0.000,0.000,0.000}{topic modeling }}\textcolor[rgb]{0.000,0.000,0.000}{comme m\'{e}thode, qui bien que tr\`{e}s int\'{e}ressant dans le cadre de notre \'{e}tude, ne permet pas d'aboutir \`{a} un mat\'{e}riau suffisant pour faire de l'histoire des futurs envisag\'{e}s.}





\textcolor[rgb]{0.000,0.000,0.000}{	Peut-\^{e}tre qu'une des pistes pour comprendre l'\'{e}chec de notre m\'{e}thode r\'{e}side en notre volont\'{e} de sonder le contenu de notre corpus \`{a} des fins interpr\'{e}tatives, et non \`{a} des fins typologiques. La plupart du temps, en effet, le }\emph{\textcolor[rgb]{0.000,0.000,0.000}{topic modeling}}\textcolor[rgb]{0.000,0.000,0.000}{ permet d'explorer de tr\`{e}s vastes corpus sans intervention humaine \`{a} des fins de classification th\'{e}matique du contenu : ainsi, les diff\'{e}rents }\emph{\textcolor[rgb]{0.000,0.000,0.000}{topics}}\textcolor[rgb]{0.000,0.000,0.000}{ obtenus, ainsi que les termes qui les composent, n'ont que tr\`{e}s peu d'int\'{e}r\^{e}t }\emph{\textcolor[rgb]{0.000,0.000,0.000}{per se}}\textcolor[rgb]{0.000,0.000,0.000}{. Ce qui importe avant tout, c'est de parvenir \`{a} classer le contenu en diff\'{e}rentes cat\'{e}gories pertinentes, quand bien m\^{e}me la coh\'{e}rence interne de chacune de ces cat\'{e}gories n'est pas optimales. }





\textcolor[rgb]{0.000,0.000,0.000}{	Au contraire, dans le cadre de notre d\'{e}marche, c'est la coh\'{e}rence de ces }\emph{\textcolor[rgb]{0.000,0.000,0.000}{topics}}\textcolor[rgb]{0.000,0.000,0.000}{, la pertinence des termes qu'ils r\'{e}v\`{e}lent, qui nous int\'{e}ressent, puisque ceux sont ces \'{e}l\'{e}ments qui sous-tendent l'exploration th\'{e}matique de l'imaginaire social et des repr\'{e}sentations. Si nous avions utilis\'{e} le }\emph{\textcolor[rgb]{0.000,0.000,0.000}{topic modeling}}\textcolor[rgb]{0.000,0.000,0.000}{ \`{a} des fins de classification des diff\'{e}rents textes de notre corpus, et }\emph{\textcolor[rgb]{0.000,0.000,0.000}{a fortiori}}\textcolor[rgb]{0.000,0.000,0.000}{, afin de dresser une typologie des oeuvres de science-fiction fond\'{e}e sur leur contenu th\'{e}matique, aurait-on obtenu des r\'{e}sultats plus pertinents ? C'est l\`{a} une question qui m\'{e}rite d'\^{e}tre pos\'{e}e, et une perspective int\'{e}ressante pour la suite de nos recherches. Mais nous y reviendrons plus tard dans notre expos\'{e}. Retenons surtout ici que bien qu'elle offre des r\'{e}sultats pertinents dans le cadre de notre \'{e}tude, notre m\'{e}thode n'a pas les moyens de ses ambitions en termes d'exploration des repr\'{e}sentations.}





\textcolor[rgb]{0.000,0.000,0.000}{	En fait, il semblerait surtout que l'utilisation exclusive du }\emph{\textcolor[rgb]{0.000,0.000,0.000}{topic modeling}}\textcolor[rgb]{0.000,0.000,0.000}{ soit une limite pour notre \'{e}tude. Si cette m\'{e}thode permet d'obtenir des r\'{e}sultats int\'{e}ressants pour appr\'{e}hender et sonder un vaste corpus comme le n\^{o}tre, est-ce pour autant suffisant pour explorer les imaginaires, et plus pr\'{e}cis\'{e}ment les futurs envisag\'{e}s d'une \'{e}poque ? Ce qui est certain, c'est que l'analyse en }\emph{\textcolor[rgb]{0.000,0.000,0.000}{topic modeling}}\textcolor[rgb]{0.000,0.000,0.000}{ ne permet pas, seule, d'aboutir \`{a} des r\'{e}sultats suffisants pour servir de base \`{a} une histoire des repr\'{e}sentations li\'{e}es \`{a} l'avenir. C'est justement pour cette raison que nous terminerons notre \'{e}tude en \'{e}voquant les m\'{e}thodes compl\'{e}mentaires que nous pourrions envisager d'utiliser, ainsi que les diff\'{e}rentes perspectives de recherche qui s'offrent \`{a} nous pour la suite.}








	M\textbf{\'{e}thodes compl\'{e}mentaires et perspectives de recherche}





\textcolor[rgb]{0.000,0.000,0.000}{	Nous commencerons ici par \'{e}voquer les diff\'{e}rents moyens qui s'offrent \`{a} nous pour enrichir notre m\'{e}thode et poursuivre nos travaux sur l'histoire des futurs envisag\'{e}s, avant d'\'{e}voquer plus en profondeur les perspectives de recherches qui pourraient \^{e}tre explor\'{e}es pour la suite. Quelles analyses compl\'{e}mentaires, quels outils pouvons nous envisager d'exploiter pour affiner notre appr\'{e}hension des repr\'{e}sentations li\'{e}es \`{a} l'avenir?}\textbf{\textcolor[rgb]{0.000,0.000,0.000}{ }}\textcolor[rgb]{0.000,0.000,0.000}{Quelles perspectives s'offrent \`{a} nous pour la suite de nos recherches ?}





	Commen\c{c}ons par rappeler la possibilit\'{e} d'utiliser d'autres techniques d'impl\'{e}mentation du \emph{topic }modeling. On a expliqu\'{e} que la LDA n'\'{e}tais pas la seule m\'{e}thode existante pour la mod\'{e}lisation de sujets. On a \'{e}voqu\'{e} bri\`{e}vement les mod\`{e}les de la LSA et la PLSA. Mais l\`{a} encore, il nous appartient de suivre l'actualit\'{e} des m\'{e}thodes et usages en la mati\`{e}re, pour essayer de mettre en oeuvre d'autres m\'{e}thodes parall\`{e}lement \`{a} notre algorithme de LDA. On pourra ici se r\'{e}f\'{e}rer \`{a} un ouvrage plus r\'{e}cent (2012), et relativement complet, sur les m\'{e}thodes de l'analyse statistiques de donn\'{e}es textuelles\footnote{\textsuperscript{\newpage
}\textsuperscript{	AGGARWAL\ C.,\ ZHAI\ C.,\ «\ Mining\ Text\ Data\ »,\ Springer,\ 2012.}} . On consultera notamment les chapitres 4 et 5, respectivement consacr\'{e}s au \emph{clustering} de textes, et au \emph{topic modeling}, qui nous renseignent pr\'{e}cis\'{e}ment et clairement sur les diff\'{e}rentes alternatives qui s'offrent \`{a} nous. On pense notamment \`{a} un autre type de mod\'{e}lisation de sujet automatique, la Non-Negative Matrix Factorisation (NMF), qui pourrait se r\'{e}v\'{e}ler assez pertinent dans le cadre de notre analyse.





	Dans un second temps, on peut envisager d'enrichir notre analyse en couplant notre algorithme de LDA \`{a} d'autres m\'{e}thodes d'analyses textuelles compl\'{e}mentaires. Dans le cadre d'un d\'{e}fi propos\'{e} annuellement par l'association EGC (Extraction et Gestion des Connaissances), un article paru en 2016 a attir\'{e} notre attention\footnote{\textsuperscript{\newpage
}\textsuperscript{	}\textsuperscript{{\tiny ALIGON,\ Julien,\ GUILLET,\ Fabrice,\ BLANCHARD,\ Julien,\ PICAROUGNE,\ Fabien.\ }}\emph{\textsuperscript{{\tiny Analyse\ par\ Motifs\ Fr\'{e}quents\ et\ Topic\ Modeling}}}\textsuperscript{{\tiny ,\ paru\ dans\ le\ cadre\ du\ d\'{e}fi\ EGC,\ Laboratoire\ Informatique\ de\ Nantes\ Atlantique,\ 2016}}} : l'\'{e}quipe de recherche y montrait l'int\'{e}r\^{e}t d'utiliser conjointement les techniques de topic modeling et d'extraction de motifs r\'{e}currents, pour mettre en \'{e}vidence non seulement les mots partageant des relations th\'{e}matiques, mais aussi leurs relations fr\'{e}quentes, au niveau intra et inter th\'{e}matiques. En confrontant leurs m\'{e}thodes aux donn\'{e}es du D\'{e}fi, l'\'{e}quipe a ainsi pu aboutir \`{a} une utilisation probante de ces m\'{e}thodes coupl\'{e}es, et \`{a} une repr\'{e}sentation int\'{e}ressante, en bi-graphe, des relations entre les mots et les \emph{topics}. On a donc l\`{a} un bon moyen d'affiner notre analyse de topics, en s'int\'{e}ressant aux associations r\'{e}currentes des mots, et donc \`{a} l'usage qui est fait de ces mots. 





	Analoguement, on pourra \'{e}galement employer l'un des modules d\'{e}velopp\'{e}s par un autre \'{e}tudiant du master, Michel Capot, pour travailler sur les cooccurrences lexicales\footnote{	Pour consulter le github du module en question, voir https://github.com/johnloque/lexnet} . Le module porte le nom de \emph{lexnet} et se fonde, \`{a} l'instar de notre module, sur l'utilisation de biblioth\`{e}ques ext\'{e}rieures (NetworkX, pandas, scipy, matplolib et numpy). A partir d'un fichier .tsv contenant un texte annot\'{e}, il permet d'obtenir diff\'{e}rents graphes pond\'{e}r\'{e}s pour visualiser les cooccurrences lexicales au sein du texte. On note \'{e}galement la possibilit\'{e} de faire certains calculs fond\'{e}s sur les propri\'{e}t\'{e}s des graphes, notamment le taux d'intersection de diff\'{e}rents champs lexicaux qui pourrait s'av\'{e}rer int\'{e}ressant sur notre corpus. Ce module permettrait d'explorer d'une autre mani\`{e}re la s\'{e}mantique et le lexique de notre corpus de textes. 





	Il faut \'{e}galement \'{e}voquer la dimension temporelle, que nous avons ici volontairement  \'{e}cart\'{e}e en nous concentrant sur une p\'{e}riode relativement restreinte. Il s'agit pourtant l\`{a} d'une des perspectives de recherche majeures pour la suite de nos travaux. Il s'agirait donc d'employer des m\'{e}thodes qui permettent de sonder th\'{e}matiquement un corpus compos\'{e} d'oeuvres \'{e}tal\'{e}es sur une plus vaste p\'{e}riode, et d'appr\'{e}cier justement l'\'{e}volution s\'{e}mantique des repr\'{e}sentations et des images \`{a} travers le temps. Bien entendu, de premi\`{e}res analyses pourraient \^{e}tre r\'{e}alis\'{e}es tr\`{e}s simplement, en d\'{e}coupant notre corpus en plusieurs sous-corpus pour chaque p\'{e}riode et en comparant les r\'{e}sultats de chaque sous-corpus. Cette m\'{e}thode reposerait, certes, sur un d\'{e}couage un peu arbitraire des diff\'{e}rentes sous-p\'{e}riode; mais elle aurait le m\'{e}rite de la clart\'{e}, et permettrait sans doute de faire ressortir et d'appr\'{e}cier diff\'{e}rentes \'{e}volutions historiques. 





	Mais une m\'{e}thode plus \'{e}l\'{e}gante doit ici \^{e}tre \'{e}voqu\'{e}e, d'autant plus qu'elle semble de plus en plus prometteuse et que des chercheurs de nombreux champs la mettent \`{a} contribution, dans le cadre de travaux vari\'{e}s. Celle des \emph{word embeddings} -- plongement lexical en fran\c{c}ais, et en particulier de ce qu'on appelle les \emph{diachronic word embeddings}, qui s'int\'{e}ressent \`{a} l'\'{e}volution historique du lexique. On citera notamment l'article de r\'{e}f\'{e}rence de William Hamilton, intitul\'{e} \emph{Diachronic Word Embeddings Reveal Statistical Laws of Semantic Change}\footnote{\emph{\newpage
}\emph{	}\emph{{\scriptsize  HAMILTON William, LESKOVEC Jure, JURAFSKY Dan. Diachronic Word Embeddings Reveal Statistical Laws of Semantic Change}}{\scriptsize , arXiv e-prints, Computer Science Department, Stanford University.}} . Cet article paru en 2016 a permis de poser bon nombre de jalons, et notamment deux lois statistiques de l'\'{e}volution s\'{e}mantique, pour mieux comprendre et analyser l'\'{e}volution historique du lexique. Depuis, on assiste \`{a} une profusion de litt\'{e}rature sur le sujet, pr\'{e}sentant de nouvelles m\'{e}thodes et des travaux int\'{e}ressant. Une th\`{e}se soutenue en 2021 \`{a} l'universit\'{e} Paris-Saclay\footnote{	 MONTARIOL Syrielle, \emph{Models of diachronic semantic changes using word embeddings. Document and Text Processing.} Universit\'{e} Paris-Saclay, 2021}  rend bien compte de l'\'{e}tat de l'art en la mati\`{e}re, en m\^{e}me temps qu'elle propose une m\'{e}thode innovante et des exemples d'application pratiques. Elle utilise notamment des m\'{e}thodes d'apprentissage de plongements contextualis\'{e}s \`{a} l'aide de mod\`{e}les de langue pr\'{e}-entra\^{i}n\'{e}s, comme BERT. Particuli\`{e}rement probante dans le cadre des diff\'{e}rents exemples d'applications pr\'{e}sent\'{e}s, cette m\'{e}thode ouvre des perspectives prometteuses pour sonder l'\'{e}volution historique des repr\'{e}sentations s\'{e}mantiques li\'{e}es \`{a} l'avenir dans un corpus de litt\'{e}rature d'anticipation.





\textbf{	}Enfin, une derni\`{e}re perspective de recherche, plus ambitieuse et lointaine, peut \^{e}tre mentionn\'{e}e. On a vu que la science-fiction \'{e}tait un genre complexe, qui peinait \`{a} \^{e}tre d\'{e}fini, et que les topologies existantes des sous-genres de la SF n'\'{e}taient pas pleinement satisfaisantes. En ce sens, pourquoi ne pas \'{e}galement employer cette exploration th\'{e}matique \`{a} des fins de cat\'{e}gorisation de notre objet? On pourrait \`{a} terme, en constituant un corpus de textes suffisamment cons\'{e}quent, envisager de proposer une classification th\'{e}matique des sous-genres de la SF, en se fondant sur nos m\'{e}thodes de \emph{topic modeling}, ainsi que sur d'autres m\'{e}thodes de clustering. Il faudrait \'{e}videmment un corpus de textes suffisamment important pour qu'il soit repr\'{e}sentatif de la litt\'{e}rature SF dans son ensemble, et en ce sens, le choix m\^{e}me des textes \`{a} inclure dans le corpus impliquerait un effort \'{e}pist\'{e}mologique remarquable. Il s'agirait l\`{a} d'un travail fastidieux, ne serait-ce que pour constituer un corpus cons\'{e}quent. Cependant, ce projet pourrait se r\'{e}v\'{e}ler prometteur. Il permettrait peut-\^{e}tre d'apporter une r\'{e}ponse statistique plus neutre \`{a} ce probl\`{e}me de cat\'{e}gorisation de la litt\'{e}rature de science-ficition, qui semble aujourd'hui avancer davantage gr\^{a}ce aux vues lucides et \`{a} l'\'{e}rudition de certains chercheurs, que gr\^{a}ce \`{a} des m\'{e}thodes stables de classification. 































































































Conclusion





	Pour conclure


\end{document}
